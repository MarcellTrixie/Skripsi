%versi 2 (8-10-2016) 
\chapter{Pendahuluan}
\label{chap:intro}
   
Pada bab ini dijelaskan mengenai latar belakang penulisan skripsi, rumusan masalah, tujuan penulisan skripsi, batasan masalah, metodologi penelitian, dan sistematika penulisan skripsi.

\section{Latar Belakang}
\label{sec:label}

Skripsi merupakan karangan ilmiah yang wajib ditulis oleh mahasiswa sebagai bagian dari persyaratan akhir pendidikan akademiknya di Perguruan Tinggi. Namun dalam penulisannya, peserta skripsi sering melakukan kesalahan kecil yang tidak dapat diabaikan. Kesalahan sering terjadi dalam penggunaan imbuhan, kata keterangan, penulisan kata dan sebagainya. Hal-hal seperti ini seharusnya dapat diperiksa dan diminimalisir oleh diri sendiri. Pada saat bimbingan, waktu dosen pembimbing lebih baik dimanfaatkan untuk membahas konten dibanding memeriksa kesalahan-kesalahan tersebut. 

Dari masalah tersebut dapat dibuat sebuah aplikasi untuk melakukan pemeriksaan pada dokumen skripsi. Kesalahan yang akan diperiksa berasal dari survei yang dilakukan kepada dosen-dosen Informatika Unpar. Hasil dari survei tersebut akan diseleksi untuk diimplementasikan ke dalam aplikasi. Aplikasi sederhana ini dapat dimanfaatkan oleh mahasiswa Informatika Unpar secara mandiri. Aplikasi akan memeriksa dokumen \textit{PDF} skripsi dan menampilkan laporan yang berisi kesalahan-kesalahan yang ditemukan pada dokumen tersebut. 

\section{Rumusan Masalah}
\label{sec:rumusan}
Berdasarkan latar belakang yang sudah ditulis, dapat dirumuskan masalah sebagai berikut:
\begin{enumerate}
	\item Bagaimana cara memeriksa kesalahan yang ada pada dokumen skripsi?
	\item Bagaimana cara membuat perangkat lunak yang dapat memeriksa kesalahan pada dokumen skripsi?
\end{enumerate}

\section{Tujuan}
\label{sec:tujuan}
Tujuan dari skripsi ini adalah sebagai berikut:
\begin{enumerate}
	\item Dapat memeriksa kesalahan yang ada pada dokumen skripsi.
	\item Dapat membangun perangkat lunak untuk memeriksa kesalahan yang ada pada dokumen skripsi.
\end{enumerate}

\section{Batasan Masalah}
\label{sec:batasan}
Batasan masalah skripsi ini adalah sebagai berikut:
\begin{enumerate}
	\item Jenis dokumen yang dapat diperiksa oleh perangkat lunak yang dibuat adalah dokumen dengan ekstensi \textit{PDF}.
	\item Pemeriksaan menggunakan \textit{pattern matching} tanpa analisis gramatikal.
	\item Pemeriksaan kosakata tanpa imbuhan.
\end{enumerate}

\section{Metodologi Penelitian}
\label{sec:metlit}
Metodologi penelitian yang digunakan pada skripsi ini adalah sebagai berikut:
\begin{enumerate}
	\item Melakukan survei kepada dosen-dosen Informatika mengenai kesalahan-kesalahan penulisan yang ditemui dalam dokumen skripsi.
	\item Melakukan studi literatur \textit{Regular Expression} untuk mendeteksi kesalahan-kesalahan dalam file \textit{PDF} skripsi.
	\item Mempelajari \textit{library PDF Parser} untuk mengestraksi file \textit{PDF} skripsi yang akan diperiksa.
	\item Melakukan perancangan perangkat lunak.
	\item Melakukan implementasi perancangan perangkat lunak.
	\item Melakukan pengujian terhadap perancangan perangkat lunak.
	\item Menulis dokumen skripsi.
\end{enumerate}

\section{Sistematika Pembahasan}
\label{sec:sispem}
Sistematika penulisan pada skripsi ini terdiri dari 6 bab, yaitu:

\begin{enumerate}
	\item Bab 1 Pendahuluan
	\\
	Bab 1 akan membahas latar belakang dibuatnya perangkat lunak untuk memeriksa kesalahan dokumen skripsi. Pada bab ini dibahas juga rumusan masalah, tujuan skripsi, batasan masalah dan metodologi penelitian yang digunakan pada skripsi.

	\item Bab 2 Landasan Teori
	\\
	Bab 2 yang merupakan landasan teori akan berisi teori-teori yang menjadi dasar-dasar dalam penulisan skripsi ini. Teori yang akan dibahas pada bab 2, yaitu \textit{Regular Expression}, \textit{library PDF Parser} dan kamus bahasa Indonesia \textit{LibreOffice}.
	
	\item Bab 3 Analisis Masalah
	\\
	Bab 3 berisi analisis masalah yang muncul dalam menyelesaikan masalah tersebut. Pada bab ini akan dianalisa masalah yang ditemukan pada saat melakukan pengamatan beberapa sidang skripsi semester Ganjil 2018/2019 dan survei secara personal kepada dosen-dosen Informatika Unpar. Hasil dari setiap survei akan dipilih mana saja yang dapat diimplementasikan dalam perangkat lunak.
	
	\item Bab 4 Perancangan
	\\
	Bab 4 berisi rancangan perangkat lunak yang akan dibuat, seperti perancangan kelas dan algoritma yang digunakan dalam pembangunan perangkat lunak. Perangkat lunak akan dibuat dengan menggunakan bahasa pemrogramam \textit{PHP}.
	
	\item Bab 5 Implementasi dan Pengujian
	\\
	Bab 5 pada skripsi ini membahas implementasi perangkat lunak dan pengujian yang dilakukan terhadap perangkat lunak tersebut. Bab ini juga menjelaskan tentang spesifikasi perangkat lunak dan pengujian yang dilakukan pada skripsi ini.
	
	\item Bab 6 Kesimpulan dan Saran
	\\
	Bab 6 berisi kesimpulan dari penulisan skripsi ini. Bab ini juga berisi saran untuk pengembangan perangkat lunak agar lebih baik lagi.
\end{enumerate}