%versi 2 (8-10-2016) 
\chapter{Pendahuluan}
\label{chap:intro}
   
Pada bab pendahuluan ini dijelaskan mengenai latar belakang penulisan skripsi, rumusan masalah, tujuan penulisan skripsi, batasan masalah, metodologi penelitian, dan sistematika penulisan skripsi ini.

\section{Latar Belakang}
\label{sec:label}

Skripsi merupakan karangan ilmiah yang wajib ditulis oleh mahasiswa sebagai bagian dari persyaratan akhir pendidikan akademiknya. Namun dalam penulisannya, peserta skripsi sering melakukan kesalahan kecil. Kesalahan sering terjadi dalam penggunaan imbuhan, kata keterangan, penulisan kata dan sebagainya. Hal-hal seperti ini seharusnya dapat diperiksa dan diminimalisir oleh diri sendiri. Pada saat bimbingan, waktu dosen pembimbing lebih baik dimanfaatkan untuk membahas konten dibanding memeriksa kesalahan-kesalahan tersebut.

Dari masalah tersebut dapat dibuat sebuah aplikasi untuk melakukan pemeriksaan pada dokumen skripsi. Kesalahan yang akan diperiksa berasal dari survei yang dilakukan kepada dosen-dosen. Hasil dari survei tersebut akan diimplementasikan ke dalam aplikasi. Aplikasi sederhana ini dapat dimanfaatkan mahasiswa secara mandiri. Aplikasi ini akan menerima masukan berupa file \textit{PDF} dan menampilkan laporan yang berisi kesalahan-kesalahan yang ditemukan pada dokumen.


\section{Rumusan Masalah}
\label{sec:rumusan}
Berdasarkan deskripsi topik yang sudah ditulis, dapat dirumuskan masalah sebagai berikut :
\begin{enumerate}
	\item Bagaimana cara memeriksa kesalahan yang ada pada dokumen skripsi?
	\item Bagaimana cara membuat perangkat lunak yang dapat memeriksa kesalahan pada dokumen skripsi?
\end{enumerate}

\section{Tujuan}
\label{sec:tujuan}
Tujuan dari penulisan topik ini adalah :
\begin{enumerate}
	\item Dapat memeriksa kesalahan yang ada pada dokumen skripsi
	\item Dapat membangun perangkat lunak untuk memeriksa kesalahan yang ada pada dokumen skripsi
dengan algoritma \textit{Bee Colony}
\end{enumerate}

\section{Batasan Masalah}
\label{sec:batasan}
Adapun batasan-batasan masalah pada penelitian ini sebagai berikut :
\begin{enumerate}
	\item Pemeriksaan menggunakan \textit{pattern matching} tanpa analisis gramatikal
	\item Tidak ada pemeriksaan kosakata
\end{enumerate}

\section{Metodologi}
\label{sec:metlit}
Langkah-langkah yang akan dilakukan dalam penelitian ini adalah:
\begin{enumerate}
	\item Melakukan survei kepada seluruh dosen Informatika
	\item Mempelajari \textit{Regular Expression}	
	\item Mempelajari \textit{library PDF Parser}
	\item Mempelajari \textit{library} TCPDF
	\item Melakukan perancangan perangkat lunak
	\item Melakukan implementasi perancangan perangkat lunak
	\item Melakukan pengujian terhadap perancangan perangkat lunak
	\item Menulis dokumen skripsi
\end{enumerate}

\section{Sistematika Pembahasan}
\label{sec:sispem}
Sistematika penulisan pada skripsi ini terdiri dari 6 bab, yaitu:

\begin{enumerate}
	\item Bab I Pendahuluan
	\\
	Bab 1 akan berisi latar belakang masalah, rumusan masalah, tujuan, batasan masalah, metodologi penelitian dan sistematika pembahasan dari skripsi ini.

	\item Bab II Dasar Teori
	\\
	Bab 2 yang merupakan landasan teori akan berisi teori-teori yang menjadi dasar-dasar dalam penulisan skripsi ini.
	
	\item Bab III Analisis
	\\
	Bab 3 berisi analisis masalah yang muncul dalam menyelesaikan masalah tersebut.
	
	\item Bab IV Perancangan
	\\
	Bab 4 berisi rancangan perangkat lunak yang akan dibuat.
	
	\item Bab V Implementasi dan Pengujian
	\\
	Bab 5 berisi implementasi perangkat lunak yang sudah dirancang pada bab sebelumnya. Disini akan diuraikan detail dari perangkat lunak yang telah dibuat. Setelah perangkat lunak dibuat perlu diuji fungsionalitasnya, untuk mengetahui kemampuan program dalam menghadapi masalah yang diberikan. 
	
	\item Bab V1 Kesimpulan
	\\
	Bab 6 berisi kesimpulan dari penulisan skripsi ini. Bab ini juga berisi saran.
\end{enumerate}