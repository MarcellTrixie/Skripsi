\chapter{Analisis Masalah}
\label{chap:analisis}

Pada bab 3 akan diuraikan tentang survei kesalahan umum.

\section{Survei Kesalahan Umum}
\label{sec:survei}

Pada subbab ini, akan diuraikan tentang survei yang dilakukan untuk mengumpulkan informasi yang dibutuhkan dalam pengembangan perangkat lunak. Informasi yang dicari adalah tentang kesalahan-kesalahan umum yang sering terjadi pada penulisan dokumen skripsi. Untuk mengumpulkan informasi tersebut, metode yang dipilih adalah melakukan survei. Dalam pelaksanaannya, survei dibagi menjadi dua, yaitu pengamatan beberapa sidang skripsi dan wawancara secara personal kepada dosen-dosen Informatika Unpar. 

\subsection{Pengamatan Sidang}
Pengamatan dilakukan pada sidang skripsi semester Ganjil 2018/2019, yang berlangsung pada bulan Mei 2019. Tidak semua sidang skripsi yang berlangsung diamati, melainkan dari 42 sidang skripsi hanya diambil 7 sidang skripsi saja. Hal tersebut dilakukan dengan pertimbangan dari ke-7 sidang skripsi tersebut diuji oleh dosen Informatika yang berbeda-beda. Namun ada beberapa dosen Informatika yang tidak masuk dalam pengamatan, karena tidak dapat menghadiri sidang yang diuji oleh dosen tersebut. Berikut adalah rincian dari sidang skripsi yang telah diamati:

\begin{table}[H]
	\caption {Tabel pengamatan sidang skripsi} \label{tab:pengamatan_sidang}
	\begin{center}
		\begin{tabular}{|p{2 cm}|>{\raggedright} p{3.5 cm}| p{4.5 cm}| p{4.5 cm}|}
		\hline
		Tanggal & Mahasiswa & Judul Skripsi & Penguji \\ 
		\hline
		15-05-2019 & Osfaldo Mickael Oktavianus Naibaho & Sistem Informasi Penjualan Barang Pada Apotek & -Vania Natali, S.Kom, M.T. \newline -Elisati Hulu, M.T. \newline \\ 
		\hline
		16-05-2019 & Ricky Wahyudi & Temu Kembali Gambar Menggunakan Fitur Surf dan Warna & -Dr.rer.nat. Cecilia Esti Nugraheni, ST, MT \newline -Dr. Ir. Veronica Sri Moertini, MT. \newline \\ 
		\hline 
		17-05-2019 & Billy Adiwijaya & Pembangkit Timelapse Pengembangan Proyek Perangkat Lunak & -Kristopher David Harjono M.T. \newline -Elisati Hulu M.T. \newline \\ 
		\hline 
		\end{tabular}
	\end{center}
\end{table}

\begin{table}[H]
	\caption {Tabel pengamatan sidang skripsi} \label{tab:pengamatan_sidang}
	\begin{center}
		\begin{tabular}{|p{2 cm}|>{\raggedright} p{3.5 cm}| p{4.5 cm}| p{4.5 cm}|}
		\hline
		Tanggal & Mahasiswa & Judul Skripsi & Penguji \\ 
		\hline
		20-05-2019 & Ihsan Fajari & Sistem Informasi Rekomendasi Pariwisata di Tasikmalaya & -Dra. Rosa de Lima Endang Padmowati, MT \newline -Dr. Ir. Veronica Sri Moertini, MT. \newline \\ 
		\hline 
		22-05-2019 & Muhammad Adrian Putra Zubir & Sistem Informasi Penyediaan Barang Pada Apotek & -Dra. Rosa de Lima Endang Padmowati, MT \newline -Pascal Alfian Nugroho, S.Kom, M.Comp. \newline \\ 
		\hline 
		23-05-2019 & Ellena Angelica & Kolektor Pengumuman Informatika & -Natalia S.Si, M.Si \newline -Dr. Ir. Veronica Sri Moertini, MT. \newline \\ 
		\hline 
		24-05-2019 & Evelyn Wijaya & Open Source Snake 360 & -Candra Wijaya S.T., M.T. \newline -Raymond Chandra Putra, S.T., M.T. \newline \\ 
		\hline 
		\end{tabular}
	\end{center}
\end{table}

Dari ke-7 pengamatan yang telah dilakukan, terdapat beberapa kesalahan-kesalahan yang terjadi dalam penulisan dokumen skripsi. Berikut adalah kesalahan-kesalahan yang disebutkan oleh dosen penguji pada sidang skripsi di atas beserta penjelasannya:


\begin{table}[H]
	\caption {Tabel hasil pengamatan sidang skripsi} \label{tab:hasil_sidang}
	\begin{center}
		\begin{tabular}{|p{1.5 cm}|>{\raggedright} p{5.5 cm}| p{7.5 cm}|}
		\hline
		Kode & Jenis kesalahan & Keterangan \\ 
		\hline
		PS-01 & Penulisan kata & Mahasiswa paling sering melakukan kesalahan dalam penulisan kata, atau yang lebih sering disebut dengan \textit{typo}. Kesalahan-kesalahan kecil seperti ini paling sering terjadi tanpa disadari. \newline \\ 
		\hline 
		PS-02 & Penggunaan imbuhan di- dan kata depan di & Kesalahan ini merupakan kesalahan yang sering terjadi dalam penulisan dokumen skripsi. Penulisan imbuhan di- disatukan antara imbuhan dengan kata dasarnya. Untuk kata depan, penulisannya dipisah antara kata depan dengan kata berikutnya. Pada umumnya diikuti oleh keterangan tempat atau waktu. \newline \\ 
		\hline 
		\end{tabular}
	\end{center}
\end{table}

\begin{table}[H]
	\caption {Tabel hasil pengamatan sidang skripsi} \label{tab:hasil_sidang}
	\begin{center}
		\begin{tabular}{|p{1.5 cm}|>{\raggedright} p{5.5 cm}| p{7.5 cm}|}
		\hline
		Kode & Jenis kesalahan & Keterangan \\ 
		\hline		
		PS-03 & Pemberian spasi sebelum dan setelah tanda baca & Salah satu hal kecil yang sering mengganggu adalah penggunaan spasi sebelum dan setelah tanda baca. Tanda baca yang paling sering dipakai, seperti titik, koma, tanya, dan seru harus diberi spasi setelahnya. Spasi juga digunakan sebelum menggunakan tanda kurung buka. Ada beberapa kesalahan yang masih ditemukan seperti, memberi spasi sebelum tanda tanya ataupun memberi spasi sebelum dan setelah garis miring. \newline \\ 
		\hline
		PS-04 & Pemberian spasi sebelum dan setelah tanda baca & Masalah ini sering ditemukan dalam penulisan dokumen skripsi, biasanya terjadi pada saat menyisipkan gambar atau tabel. Susunan atau ukuran gambar yang tidak tepat dapat mengakibatkan terciptanya ruang kosong yang besar. \newline \\
		\hline 
		PS-05 & Awal kalimat tidak menggunakan huruf kapital & Setiap huruf pertama pada kata pertama dalam sebuah kalimat harus ditulis dengan huruf kapital. \newline \\  
		\hline 
		PS-06 & Pemberian spasi sebelum dan setelah tanda baca & Setiap kata dalam sebuah kalimat dipisahkan dengan jarak 1 spasi agar kalimat dapat dibaca dan dimengerti dengan baik. \newline \\ 
		\hline 
		PS-07 & Pemberian spasi sebelum dan setelah tanda baca & Pada PDF Latex, biasanya kesalahan ini karena mahasiswa tidak memberikan tag kepada gambar tersebut. Hal ini mengakibatkan posisi gambar tidak terletak pada tempat yang seharusnya. \newline \\ 
		\hline 
		PS-08 & Pemberian spasi sebelum dan setelah tanda baca & Dalam penulisan dokumen skripsi, setiap gambar dan tabel perlu diberikan keterangan.
 \newline \\ 
		\hline 
		PS-09 & Pemberian spasi sebelum dan setelah tanda baca & Dalam sebuah bab, biasanya jumlah subbab lebih dari 1. Kesalahan yang sering dilakukan oleh mahasiswa yaitu, hanya terdapat 1 subbab saja pada 1 bab. Apabila dalam bab tersebut hanya terdapat 1 subbab, lebih baik tidak perlu dibuat subbab. \newline \\ 
		\hline 
		\end{tabular}
	\end{center}
\end{table}

\subsection{Wawancara Personal}
Survei dilanjutkan dengan melakukan wawancara secara personal kepada dosen-dosen Informatika Unpar. Berikut adalah hasil wawancara yang sudah dilakukan:

\begin{table}[H]
	\caption {Tabel hasil wawancara dosen} \label{tab:hasil_wawancara}
	\begin{center}
		\begin{tabular}{|p{2 cm}|>{\raggedright} p{3.5 cm}| p{4 cm}| p{5 cm}|}
		\hline
		Tanggal & Narasumber & Hasil Wawancara & Penjelasan \\ 
		\hline
		9-07-2019 & Keenan Adiwijaya Leeman S.T. & KAL-01 \newline Cetak miring untuk bahasa asing & Penggunaan kata dalam bahasa asing harus ditulis menggunakan cetak miring. Mahasiswa sering lupa untuk menulis cetak miring bahasa asing. \newline \\ 
		\hline
		 & & KAL-02 \newline Kalimat pengantar untuk setiap subbab & Setiap penulisan bab dan subbab selalu diikuti dengan kalimat pengantar untuk memulai bab dan subbab tersebut. Kesalahan yang sering terjadi, yaitu mahasiswa seringkali lupa untuk menuliskan kalimat pengantar tersebut. \newline \\ 
		\hline 
		 & & KAL-03 \newline Kelengkapan data skripsi & Data skripsi harus diisi dengan lengkap sebagai bentuk identitas, seperti nama mahasiswa, NPM, dosen pembimbing, judul skripsi dan sebagainya. Hal-hal seperti seringkali lupa diisi karena terlalu fokus dalam mengerjakan konten-konten dalam skripsi. \newline \\
		\hline
		9-07-2019 & Candra Wijaya S.T., M.T. & CHW-01 \newline Letak keterangan untuk gambar dan tabel & Kesalahan yang sering terjadi adalah letak dari penulisan keterangan tersebut. Keterangan pada gambar posisinya ada di bawah gambar, sedangkan keterangan pada tabel posisinya ada di atas tabel. \newline \\ 
		\hline
		 & & CHW-02 \newline Penggunaan bahasa yang benar & KBBI menjadi kaidah dalam penulisan bahasa Indonesia. Mahasiswa terkadang salah memilih kata yang hendak ditulis dalam dokumen, padahal kata tersebut tidak sesuai dengan KBBI. \newline \\ 
		\hline
		\end{tabular}
	\end{center}
\end{table}

\begin{table}[H]
	\caption {Tabel hasil wawancara dosen} \label{tab:hasil_wawancara}
	\begin{center}
		\begin{tabular}{|p{2 cm}|>{\raggedright} p{3.5 cm}| p{4 cm}| p{5 cm}|}
		\hline
		Tanggal & Narasumber & Hasil Wawancara & Penjelasan \\ 
		\hline
		15-07-2019 & Husnul Hakim, S.Kom., M.T. & HUH-01 \newline Rujukan untuk gambar dan tabel & Setiap gambar dan tabel yang dimasukan ke dalam dokumen skripsi, perlu dirujuk dalam sebuah paragraf. Mahasiswa sering lupa atau terlewat untuk merujuk gambar dan tabel tersebut. \newline \\ 
		\hline
		 & & HUH-02 \newline Penulisan pseudocode & Dalam penulisan pseudocode hal-hal yang perlu diperhatikan antara lain nama method, masukan serta keluaran pada method dan no baris pada pseudocode.	 \newline \\ 
		\hline
		 & & HUH-03 \newline Penulisan kata hubung & Kesalahan penggunaan konjungsi akan berakibat tidak jelasnya makna kalimat karena hubungan antar frasa dan antar klausa tidak jelas. \newline \\ 
		\hline
		
		16-07-2019 & Vania Natali, S.Kom, M.T. & VAN-01 \newline Tahun skripsi pada cover skripsi & Penulisan tahun skripsi harus sama dengan tahun dimana mahasiswa mengambil skripsi tersebut. Kesalahan yang pernah terjadi, yaitu mahasiswa salah menuliskan tahun skripsi. Meskipun terlihat sepele, namun hal ini perlu diperhatikan.			 \newline \\ 
		\hline
		 & & VAN-02 \newline Konsistensi penggunaan kata & Mahasiswa harus konsisten dalam penulisan kata, misalnya kata\textit{user} dan pengguna.  Mahasiswa harus memilih antara memakai \textit{user} atau pengguna. \newline \\ 
		\hline
		 & & VAN-02 \newline Penggunaan kata ganti orang & Dalam penulisan dokumen skripsi, tidak boleh ada kata ganti orang. Jika karya non-ilmiah lebih santai karena memakai gaya bahasa non-formal, maka berbeda dengan karya ilmiah. Karya ilmiah memiliki aturan baku dan menggunakan bahasa formal.  \newline \\ 
		\hline
		\end{tabular}
	\end{center}
\end{table}

\begin{table}[H]
	\caption {Tabel hasil wawancara dosen} \label{tab:hasil_wawancara}
	\begin{center}
		\begin{tabular}{|p{2 cm}|>{\raggedright} p{3.5 cm}| p{4 cm}| p{5 cm}|}
		\hline
		Tanggal & Narasumber & Hasil Wawancara & Penjelasan \\ 
		\hline
		16-07-2019 & Natalia S.Si, M.Si & NAT-01 \newline Penulisan daftar referensi & Penulisan daftar referensi dibuat jika dalam tulisan ilmiah tersebut memang menggunakan kutipan atau rujukan dari orang lain. Kesalahan yang sering terjadi, yaitu tidak ditemukannya referensi yang akan digunakan. Pada teks yang akan dirujuk, akan terdapat tanda [?], seharusnya tanda tanya tersebut diisi oleh nomor dari referensi. \newline \\ 
		\hline
		\end{tabular}
	\end{center}
\end{table}

\begin{comment}
\begin{enumerate}
	\item Wawancara pertama dilakukan pada tanggal 9 Juli 2019. Dosen yang menjadi narasumbernya adalah Keenan Adiwijaya Leeman S.T.. Kesalahan penulisan dokumen skripsi yand didapat adalah sebagai berikut:
		
		\begin{itemize}
			\item Cetak miring untuk bahasa asing \\
			Kode implementasi: KAL-01 \\
			Penggunaan kata dalam bahasa asing harus ditulis menggunakan cetak miring. Mahasiswa sering lupa untuk menulis cetak miring bahasa asing.
			
			\item Kalimat pengantar untuk setiap subbab \\
			Kode implementasi: KAL-02 \\			
			Setiap penulisan bab dan subbab selalu diikuti dengan kalimat pengantar untuk memulai bab dan subbab tersebut. Kesalahan yang sering terjadi, yaitu mahasiswa seringkali lupa untuk menuliskan kalimat pengantar tersebut.
			
			\item Kelengkapan data skripsi \\
			Kode implementasi: KAL-03 \\
			Data skripsi harus diisi dengan lengkap sebagai bentuk identitas, seperti nama mahasiswa, NPM, dosen pembimbing, judul skripsi dan sebagainya. Hal-hal seperti seringkali lupa diisi karena terlalu fokus dalam mengerjakan konten-konten dalam skripsi.				
			
		\end{itemize}
		
	\item Wawancara ke-2 dilakukan pada tanggal 9 Juli 2019. Dosen yang menjadi narasumbernya adalah Candra Wijaya S.T., M.T.. Kesalahan penulisan dokumen skripsi yand didapat adalah sebagai berikut:
	
		\begin{itemize}
			\item Letak keterangan untuk gambar dan tabel \\
			Kode implementasi: CHW-01 \\
			Kesalahan yang sering terjadi adalah letak dari penulisan keterangan tersebut. Keterangan pada gambar posisinya ada di bawah gambar, sedangkan keterangan pada tabel posisinya ada di atas tabel.
			
			\item Penggunaan bahasa yang benar \\
			Kode implementasi: CHW-02 \\
			KBBI menjadi kaidah dalam penulisan bahasa Indonesia. Mahasiswa terkadang salah memilih kata yang hendak ditulis dalam dokumen, padahal kata tersebut tidak sesuai dengan KBBI.	
			
		\end{itemize}
		
	\item Wawancara ke-3 dilakukan pada tanggal 15 Juli 2019. Dosen yang menjadi narasumbernya adalah Husnul Hakim, S.Kom., M.T.. Kesalahan penulisan dokumen skripsi yand didapat adalah sebagai berikut:
	
		\begin{itemize}
			\item Rujukan untuk gambar dan tabel \\	
			Kode implementasi: HUH-01 \\
			Setiap gambar dan tabel yang dimasukan ke dalam dokumen skripsi, perlu dirujuk dalam sebuah paragraf. Mahasiswa sering lupa atau terlewat untuk merujuk gambar dan tabel tersebut.			
			
			\item Penulisan pseudocode \\
			Kode implementasi: HUH-02 \\
			Dalam penulisan pseudocode hal-hal yang perlu diperhatikan antara lain nama method, masukan serta keluaran pada method dan no baris pada pseudocode.			
			
			\item Penulisan kata hubung \\
			Kode implementasi: HUH-03 \\
			Kesalahaan penggunaan konjungsi akan berakibat tidak jelasnya makna kalimat karena hubungan antar frasa dan antar klausa tidak jelas.
			
		\end{itemize}
		
	\item Wawancara ke-4 dilakukan pada tanggal 16 Juli 2019. Dosen yang menjadi narasumbernya adalah Vania Natali, S.Kom, M.T. Kesalahan penulisan dokumen skripsi yand didapat adalah sebagai berikut:
	
		\begin{itemize}
			\item Tahun skripsi pada cover skripsi \\
			Kode implementasi: VAN-01 \\
			Penulisan tahun skripsi harus sama dengan tahun dimana mahasiswa mengambil skripsi tersebut. Kesalahan yang pernah terjadi, yaitu mahasiswa salah menuliskan tahun skripsi. Meskipun terlihat sepele, namun hal ini perlu diperhatikan.			
			
			\item Konsistensi penggunaan kata \\
			Kode implementasi: VAN-02 \\
			Mahasiswa harus konsisten dalam penulisan kata, misalnya kata\textit{user} dan pengguna.  Mahasiswa harus memilih antara memakai \textit{user} atau pengguna.
			
			\item Penggunaan kata ganti orang \\
			Kode implementasi: VAN-03 \\
			Dalam penulisan dokumen skripsi, tidak boleh ada kata ganti orang. Jika karya non-ilmiah lebih santai karena memakai gaya bahasa non-formal, maka berbeda dengan karya ilmiah. Karya ilmiah memiliki aturan baku dan menggunakan bahasa formal. 
			
		\end{itemize}
		
	\item Wawancara ke-5 dilakukan pada tanggal 16 Juli 2019.
	
		\begin{itemize}
			\item Penulisan daftar referensi \\
			Kode implementasi: NAT-01 \\ 
			Penulisan daftar referensi dibuat jika dalam tulisan ilmiah tersebut memang menggunakan kutipan atau rujukan dari orang lain. Kesalahan yang sering terjadi, yaitu tidak ditemukannya referensi yang akan digunakan. Pada teks yang akan dirujuk, akan terdapat tanda [?], seharusnya tanda tanya tersebut diisi oleh nomor dari referensi.
			
		\end{itemize}
		
\end{enumerate}
\end{comment}

\section{Keputusan Implementasi Hasil Survei}

Setiap hasil survei yang didapatkan melalui pengamatan sidang skripsi dan wawancara dosen, telah diberikan sebuah kode. Kode tersebut akan digunakan sebagai identitas dari setiap hasil survei. Namun, tidak semua dari hasil survei tersebut dapat diimplementasikan menggunakan \textit{regex}. Metode yang digunakan untuk mendeteksi kesalahan yaitu dengan \textit{pattern matching}, sehingga hal-hal yang bersifat kontekstual tidak dapat dicek dengan \textit{regex}. Berikut ini adalah hasil keputusan yang telah diambil pada setiap hasil survei di atas:

\begin{table}[H]
	\caption {Tabel keputusan implementasi} \label{tab:keputusan}
	\begin{center}
		\begin{tabular}{|p{1.5 cm}|>{\raggedright} p{4 cm}| p{2 cm}| p{7 cm}|}
		\hline
		Kode & Survei & Keputusan & Alasan \\ 
		\hline 
		PS-01 & Penulisan Kata & Ya & Dapat diselesaikan menggunakan \textit{regular expression} \newline \\ 
		\hline 
		PS-02 & Penggunaan imbuhan di- dan kata depan di- & Ya & Dapat diselesaikan menggunakan \textit{regular expression}. Namun untuk penggunaan kata depan di- tidak bisa diimplementasi karena pada kamus Indonesia \textit{LibreOffice} tidak ada fitur untuk membedakan kata sebagai keterangan atau bukan. \newline \\ 
		\hline 
		PS-03 & Pemberian spasi sebelum dan setelah tanda baca & Ya & Dapat diselesaikan menggunakan \textit{regular expression} \newline \\ 
		\hline 
		PS-04 & Terdapat ruang kosong yang besar & Tidak & Tidak dapat diselesaikan menggunakan \textit{regular expression}, karena hasil ekstrak dari PDF menggunakan \textit{PDF Parser} tidak mendeteksi adanya baris kosong. \newline \\ 
		\hline 
		\end{tabular}
	\end{center}
\end{table}

\begin{table}[H]
	\caption {Tabel keputusan implementasi} \label{tab:keputusan}
	\begin{center}
		\begin{tabular}{|p{1.5 cm}|>{\raggedright} p{4 cm}| p{2 cm}| p{7 cm}|}
		\hline
		Kode & Survei & Keputusan & Alasan \\ 
		\hline 
		PS-05 & Awal kalimat tidak menggunakan huruf kapital & Ya & Dapat diselesaikan menggunakan \textit{regular expression} \newline \\ 
		\hline
		PS-06 & Tidak ada spasi antar kata & Ya & Dapat diselesaikan menggunakan \textit{regular expression} \newline \\ 
		\hline 
		PS-07 & Gambar tidak sesuai tempatnya & Tidak & Tidak dapat diselesaikan menggunakan \textit{regular expression}, karena hasil ekstrak PDF dari \textit{PDF Parser} tidak dapat mendeteksi gambar. \newline \\ 
		\hline 
		PS-08 & Tidak ada keterangan untuk gambar dan tabel & Tidak & Tidak dapat diselesaikan menggunakan \textit{regular expression}, karena hasil ekstrak PDF dari \textit{PDF Parser} tidak dapat mendeteksi gambar dan tabel. \newline \\ 
		\hline 
		PS-09 & Jumlah subbab dalam 1 bab tidak boleh hanya 1 & Ya & Dapat diselesaikan menggunakan \textit{regular expression} \newline \\ 
		\hline 
		KAL-01 & Cetak miring untuk bahasa asing & Tidak & Tidak dapat diselesaikan menggunakan \textit{regular expression}, karena membutuhkan kamus bahasa Inggris. Selain itu \textit{PDF Parser} tidak dapat mencocokan teks yang cetak miring. \newline \\ 
		\hline 
		KAL-02 & Kalimat pengantar untuk setiap subbab & Ya & Dapat diselesaikan menggunakan \textit{regular expression}. \newline \\ 
		\hline 
		KAL-03 & Kelengkapan data skripsi & Ya & Dapat diselesaikan menggunakan \textit{regular expression} \newline \\ 
		\hline 
		CHW-01 & Letak keterangan untuk gambar dan tabel & Tidak & Tidak dapat diselesaikan menggunakan \textit{regular expression}, karena hasil ekstrak PDF dari \textit{PDF Parser} tidak dapat mendeteksi gambar dan tabel.\newline \\ 
		\hline 
		CHW-02 & Penggunaan bahasa yang benar & Ya & Dapat diselesaikan menggunakan \textit{regular expression} \newline \\ 
		\hline  
		\end{tabular}
	\end{center}
\end{table}

\begin{table}[H]
	\caption {Tabel keputusan implementasi} \label{tab:keputusan}
	\begin{center}
		\begin{tabular}{|p{1.5 cm}|>{\raggedright} p{4 cm}| p{2 cm}| p{7 cm}|}
		\hline
		Kode & Survei & Keputusan & Alasan \\ 
		\hline
		HUH-01 & Rujukan untuk gambar dan tabel & Tidak & Tidak dapat diselesaikan menggunakan \textit{regular expression}, karena hasil ekstrak PDF dari \textit{PDF Parser} tidak dapat mendeteksi gambar dan tabel. \newline \\ 
		\hline 
		HUH-02 & Penulisan pseudocode & Tidak & Tidak dapat diselesaikan menggunakan \textit{regular expression}, karena hasil ekstrak PDF dari \textit{PDF Parser} tidak dapat mendeteksi pseudocode. \newline \\ 
		\hline 
		HUH-03 & Penulisan kata hubung & Tidak & Tidak dapat diselesaikan menggunakan \textit{regular expression}, karena \textit{regex} tidak dapat memeriksa kata hubung yang digunakan sudah tepat atau belum berdasarkan fungsi dari kata hubung tersebut. \newline \\ 
		\hline 
		VAN-01 & Tahun skripsi pada cover skripsi & Ya & Dapat diselesaikan menggunakan \textit{regular expression} \newline \\ 
		\hline 
		VAN-02 & Konsistensi penggunaan kata & Tidak & Tidak dapat diselesaikan menggunakan \textit{regular expression}, karena \newline tidak dapat membuat padanan kata untuk memeriksa kosistensi penggunaan kata. \\ 
		\hline 
		VAN-03 & Penggunaan kata ganti orang & Ya & Dapat diselesaikan menggunakan \textit{regular expression} \newline \\ 
		\hline 
		NAT-01 & Penulisan daftar referensi & Ya & Dapat diselesaikan menggunakan \textit{regular expression} \newline \\ 
		\hline
		\end{tabular}
	\end{center}
\end{table}

Seperti yang sudah dijabarkan pada tabel 3.7 hingga tabel 3.9, 12 dari 21 hasil survei akan diimplementasikan menjadi fitur dalam perangkat lunak. Keputusan tersebut diambil berdasarkan bisa / tidaknya kesalahan tersebut diperiksa menggunakan \textit{pattern matching regex} dan tingkat kesulitan untuk memeriksa kesalahan tersebut.