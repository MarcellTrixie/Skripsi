\chapter{Analisis Masalah}
\label{chap:analisis}

Pada bab 3 akan diuraikan tentang analisis masalah.

\section{Survei Kesalahan Umum}
\label{sec:survei}
Pada subbab ini, akan diuraikan tentang survei yang dilakukan untuk mengumpulkan informasi yang dibutuhkan dalam pengembangan perangkat lunak. Informasi yang dicari adalah tentang kesalahan-kesalahan umum yang sering terjadi pada penulisan dokumen skripsi. Untuk mengumpulkan informasi tersebut, metode yang dipilih adalah melakukan survei. Dalam pelaksanaannya, survei dibagi menjadi dua, yaitu pengamatan beberapa sidang skripsi dan wawancara secara personal kepada dosen-dosen Informatika Unpar. 

\subsection{Pengamatan Sidang}
Pengamatan dilakukan pada sidang skripsi semester Ganjil 2018/2019, yang berlangsung pada bulan Mei 2019. Tidak semua sidang skripsi yang berlangsung diamati, dari 42 sidang skripsi hanya diambil 7 sidang skripsi saja. Hal tersebut dilakukan dengan pertimbangan dari ke-7 sidang skripsi tersebut diuji oleh dosen Informatika yang berbeda-beda. Namun ada beberapa dosen Informatika yang tidak masuk dalam pengamatan, karena tidak dapat menghadiri sidang yang diuji oleh dosen tersebut. Berikut adalah rincian dari sidang skripsi yang telah diamati:

\begin{enumerate}
	\item Pengamatan pertama dilakukan pada sidang skripsi Osfaldo Mickael Oktavianus Naibaho pada tanggal 15 Mei 2019. Dosen yang menguji sidang ini adalah Bu Vania Natali dan Pak Elisati Hulu.
	
	\item Pengamatan ke-2 dilakukan pada sidang skripsi Ricky Wahyudi pada tanggal 16 Mei 2019. Dosen yang menguji sidang ini adalah Bu Cecilia Esti Nugraheni dan Bu Veronica Sri Moertini.
	
	\item Pengamatan ke-3 dilakukan pada sidang skripsi Billy Adiwijaya pada tanggal 17 Mei 2019. Dosen yang menguji sidang ini adalah Pak Elisati Hulu dan Pak Kristopher David Harjono.
	
	\item Pengamatan ke-4 dilakukan pada sidang skripsi Ihsan Fajari pada tanggal 20 Mei 2019. Dosen yang menguji sidang ini adalah Bu Rosa De Lima Endang Padmowati dan Bu Veronica Sri Moertini.

	\item Pengamatan ke-5 dilakukan pada sidang skripsi Muhammad Adrian Putra Zubir pada tanggal 20 Mei 2019. Dosen yang menguji sidang ini adalah Bu Rosa De Lima Endang Padmowati dan Bu Veronica Sri Moertini.

	\item Pengamatan ke-6 dilakukan pada sidang skripsi Ellena Angelica pada tanggal 23 Mei 2019. Dosen yang menguji sidang ini adalah Bu Natalia dan Bu Veronica Sri Moertini.
	
	\item Pengamatan ke-7 dilakukan pada sidang skripsi Evelyn Wijaya pada tanggal 24 Mei 2019. Dosen yang menguji sidang ini adalah Pak Candra Wijaya dan Pak Raymond Chandra Putra.
\end{enumerate}

Dari ke-7 pengamatan yang dilakukan, didapatkan beberapa kesalahan-kesalahan yang terjadi dalam penulisan dokumen skripsi. Berikut adalah kesalahan-kesalahan yang disebutkan oleh dosen penguji pada sidang skripsi di atas beserta penjelasannya:

\begin{enumerate}
	\item Penulisan Judul
	\item Penggunaan imbuhan di
	\item Referensi tidak dirujuk
	\item Tidak ada spasi setelah tanda baca titik
	\item Awal kalimat menggunakan huruf kapital
	\item Terdapat space kosong
	\item Tidak ada spasi antar kata
	\item Kata tidak sesuai dengan KBBI
	\item Gambar tidak sesuai tempatnya
	\item Tidak ada keterangan untuk gambar dan tabel
	\item Apabila hanya terdapat 1 subsection lebih baik dijadikan 1 section saja
\end{enumerate}

\subsection{Wawancara Personal}
