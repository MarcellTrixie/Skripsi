\chapter{Analisis Masalah}
\label{chap:analisis}

Pada bab 3 akan diuraikan tentang survei kesalahan umum.

\section{Survei Kesalahan Umum}
\label{sec:survei}
Pada subbab ini, akan diuraikan tentang survei yang dilakukan untuk mengumpulkan informasi yang dibutuhkan dalam pengembangan perangkat lunak. Informasi yang dicari adalah tentang kesalahan-kesalahan umum yang sering terjadi pada penulisan dokumen skripsi. Untuk mengumpulkan informasi tersebut, metode yang dipilih adalah melakukan survei. Dalam pelaksanaannya, survei dibagi menjadi dua, yaitu pengamatan beberapa sidang skripsi dan wawancara secara personal kepada dosen-dosen Informatika Unpar. 

\subsection{Pengamatan Sidang}
Pengamatan dilakukan pada sidang skripsi semester Ganjil 2018/2019, yang berlangsung pada bulan Mei 2019. Tidak semua sidang skripsi yang berlangsung diamati, melainkan dari 42 sidang skripsi hanya diambil 7 sidang skripsi saja. Hal tersebut dilakukan dengan pertimbangan dari ke-7 sidang skripsi tersebut diuji oleh dosen Informatika yang berbeda-beda. Namun ada beberapa dosen Informatika yang tidak masuk dalam pengamatan, karena tidak dapat menghadiri sidang yang diuji oleh dosen tersebut. Berikut adalah rincian dari sidang skripsi yang telah diamati:

\begin{enumerate}
	\item Pengamatan tanggal 15 Mei 2019 \\
	Sidang yang diamati adalah sidang skripsi Osfaldo Mickael Oktavianus Naibaho, dengan judul Sistem Informasi Penjualan Barang Pada Apotek. Sidang tersebut diuji oleh Vania Natali, S.Kom, M.T. dan Elisati Hulu, M.T..
	
	\item Pengamatan tanggal 16 Mei 2019 \\
	Sidang yang diamati adalah sidang skripsi Ricky Wahyudi, dengan judul Temu Kembali Gambar Menggunakan Fitur Surf dan Warna. Sidang tersebut diuji oleh Vania Natali, S.Kom, M.T. dan Elisati Hulu, M.T..Dr.rer.nat. Cecilia Esti Nugraheni, ST, MT dan Dr. Ir. Veronica Sri Moertini, MT.
	
	\item Pengamatan tanggal 17 Mei 2019 \\
	Sidang yang diamati adalah sidang skripsi Billy Adiwijaya, dengan judul Pembangkit Timelapse Pengembangan Proyek Perangkat Lunak. Sidang tersebut diuji oleh Kristopher David Harjono M.T. dan Elisati Hulu M.T..
	
	\item Pengamatan tanggal 20 Mei 2019 \\
	Sidang yang diamati adalah sidang skripsi Ihsan Fajari, dengan judul Sistem Informasi Rekomendasi Pariwisata di Tasikmalaya. Sidang tersebut diuji oleh Dra. Rosa de Lima Endang Padmowati, MT dan Dr. Ir. Veronica Sri Moertini, MT.
	
	\item Pengamatan tanggal 22 Mei 2019 \\
	Sidang yang diamati adalah sidang skripsi Muhammad Adrian Putra Zubir, dengan judul Sistem Informasi Penyediaan Barang Pada Apotek. Sidang tersebut diuji oleh Dra. Rosa de Lima Endang Padmowati, MT dan Pascal Alfian Nugroho, S.Kom, M.Comp.
	
	\item Pengamatan tanggal 23 Mei 2019 \\
	Sidang yang diamati adalah sidang skripsi Ellena Angelica, dengan judul Kolektor Pengumuman Informatika. Sidang tersebut diuji oleh Natalia S.Si, M.Si dan Dr. Ir. Veronica Sri Moertini, MT.
	
	\item Pengamatan tanggal 24 Mei 2019 \\
	Sidang yang diamati adalah sidang skripsi Evelyn Wijaya, dengan judul Open Source Snake 360. Sidang tersebut diuji oleh Candra Wijaya S.T., M.T. dan Raymond Chandra Putra, S.T., M.T..
	
\end{enumerate}

Dari ke-7 pengamatan yang telah dilakukan, terdapat beberapa kesalahan-kesalahan yang terjadi dalam penulisan dokumen skripsi. Berikut adalah kesalahan-kesalahan yang disebutkan oleh dosen penguji pada sidang skripsi di atas beserta penjelasannya:

\begin{enumerate}
	\item Penulisan Judul \\
	Kode implementasi: PS-01 \\	
	Dalam penulisan judul, setiap huruf awal pada kata harus menggunakan huruf kapital. Hal ini berlaku untuk hampir semua jenis kata, seperti nama, tempat, sifat dan keterangan. Namun, ada beberapa pengecualian seperti preposisi (kata depan yang diikuti oleh kata lainnya), konjungsi (kata sambung), dan interjeksi (kata yang mengungkapkan seruan perasaan).
	
	\item Penulisan kata \\
	Kode implementasi: PS-02 \\	
	Mahasiswa paling sering melakukan kesalahan dalam penulisan kata, atau yang lebih sering disebut dengan \textit{typo}. Kesalahan-kesalahan kecil seperti ini paling sering terjadi tanpa disadari.
	
	\item Penggunaan imbuhan di- dan kata depan di\\
	Kode implementasi: PS-03 \\	
	Kesalahan ini merupakan kesalahan yang sering terjadi dalam penulisan dokumen skripsi. Penulisan imbuhan di- disatukan antara imbuhan dengan kata dasarnya. Untuk kata depan, penulisannya dipisah antara kata depan dengan kata berikutnya. Pada umumnya diikuti oleh keterangan tempat atau waktu.
	
	\item Tidak ada spasi setelah tanda baca titik \\
	Kode implementasi: PS-04 \\	
	Sebuah kalimat akan diakhiri oleh tanda baca titik. Untuk melanjutkan kalimat baru, setelah tanda baca titik harus ada jarak 1 spasi untuk memisahkan kalimat sebelumnya dengan yang berikutnya. Terkadang mahasiswa lupa untuk memberikan jarak 1 spasi setelah tanda baca titik di akhir kalimat.

	\item Awal kalimat tidak menggunakan huruf kapital \\
	Kode implementasi: PS-05 \\	
	Setiap huruf pertama pada kata pertama dalam sebuah kalimat harus ditulis dengan huruf kapital. 	
	
	\item Terdapat ruang kosong yang besar \\
	Kode implementasi: PS-06 \\	
	Masalah ini sering ditemukan dalam penulisan dokumen skripsi, biasanya terjadi pada saat menyisipkan gambar atau tabel. Susunan atau ukuran gambar yang tidak tepat dapat mengakibatkan terciptanya ruang kosong yang besar.

	\item Tidak ada spasi antar kata \\
	Kode implementasi: PS-07 \\	
	Setiap kata dalam sebuah kalimat dipisahkan dengan jarak 1 spasi agar kalimat dapat dibaca dan dimengerti dengan baik.

	\item Gambar tidak sesuai tempatnya \\
	Kode implementasi: PS-08 \\	
	Pada PDF Latex, biasanya kesalahan ini karena mahasiswa tidak memberikan tag kepada gambar tersebut. Hal ini mengakibatkan posisi gambar tidak terletak pada tempat yang seharusnya.

	\item Tidak ada keterangan untuk gambar dan tabel \\
	Kode implementasi: PS-09 \\	
	Dalam penulisan dokumen skripsi, setiap gambar dan tabel perlu diberikan keterangan.

	\item Jumlah subbab dalam 1 bab tidak boleh hanya 1 \\
	Kode implementasi: PS-10 \\	
	Dalam sebuah bab, biasanya jumlah subbab lebih dari 1. Kesalahan yang sering dilakukan oleh mahasiswa yaitu, hanya terdapat 1 subbab saja pada 1 bab. Apabila dalam bab tersebut hanya terdapat 1 subbab, lebih baik tidak perlu dibuat subbab.
	
\end{enumerate}

\subsection{Wawancara Personal}
Survei dilanjutkan dengan melakukan wawancara secara personal kepada dosen-dosen Informatika Unpar. Berikut adalah hasil wawancara yang sudah dilakukan:

\begin{enumerate}
	\item Wawancara pertama dilakukan pada tanggal 9 Juli 2019. Dosen yang menjadi narasumbernya adalah Keenan Adiwijaya Leeman S.T.. Kesalahan penulisan dokumen skripsi yand didapat adalah sebagai berikut:
		
		\begin{itemize}
			\item Cetak miring untuk bahasa asing \\
			Kode implementasi: KAL-01 \\
			Penggunaan kata dalam bahasa asing harus ditulis menggunakan cetak miring.	Mahasiswa sering lupa untuk menulis cetak miring bahasa asing.
			
			\item Kalimat pengantar untuk setiap subbab \\
			Kode implementasi: KAL-02 \\			
			Setiap penulisan bab dan subbab selalu diikuti dengan kalimat pengantar untuk memulai bab dan subbab tersebut. Kesalahan yang sering terjadi, yaitu mahasiswa seringkali lupa untuk menuliskan kalimat pengantar tersebut.
			
			\item Kelengkapan data skripsi \\
			Kode implementasi: KAL-03 \\
			Data skripsi harus diisi dengan lengkap sebagai bentuk identitas, seperti nama mahasiswa, NPM, dosen pembimbing, judul skripsi dan sebagainya. Hal-hal seperti seringkali lupa diisi karena terlalu fokus dalam mengerjakan konten-konten dalam skripsi.				
			
		\end{itemize}
		
	\item Wawancara ke-2 dilakukan pada tanggal 9 Juli 2019 juga. Dosen yang menjadi narasumbernya adalah Candra Wijaya S.T., M.T.. Kesalahan penulisan dokumen skripsi yand didapat adalah sebagai berikut:
	
		\begin{itemize}
			\item Letak keterangan untuk gambar dan tabel \\
			Kode implementasi: CHW-01 \\
			Kesalahan yang sering terjadi adalah letak dari penulisan keterangan tersebut. Keterangan pada gambar posisinya ada di bawah gambar, sedangkan keterangan pada tabel posisinya ada di atas tabel.
			
			\item Penggunaan bahasa yang benar \\
			Kode implementasi: CHW-02 \\
			KBBI menjadi kaidah dalam penulisan bahasa Indonesia. Mahasiswa terkadang salah memilih kata yang hendak ditulis dalam dokumen, padahal kata tersebut tidak sesuai dengan KBBI.	
			
		\end{itemize}
		
	\item Wawancara ke-3 dilakukan pada tanggal 15 Juli 2019 juga. Dosen yang menjadi narasumbernya adalah Husnul Hakim, S.Kom., M.T.. Kesalahan penulisan dokumen skripsi yand didapat adalah sebagai berikut:
	
		\begin{itemize}
			\item Rujukan untuk gambar dan tabel \\	
			Kode implementasi: HUH-01 \\
			Setiap gambar dan tabel yang dimasukan ke dalam dokumen skripsi, perlu dirujuk dalam sebuah paragraf. Mahasiswa sering lupa atau terlewat untuk merujuk gambar dan tabel tersebut.			
			
			\item Penulisan pseudocode \\
			Kode implementasi: HUH-02 \\
			Dalam penulisan pseudocode hal-hal yang perlu diperhatikan antara lain nama method, masukan serta keluaran pada method dan no baris pada pseudocode.			
			
			\item Penulisan kata hubung \\
			Kode implementasi: HUH-03 \\
			Kesalahaan penggunaan konjungsi akan berakibat tidak jelasnya makna kalimat karena hubungan antar frasa dan antar klausa tidak jelas.
			
		\end{itemize}
		
	\item Wawancara ke-4 dilakukan pada tanggal 16 Juli 2019 juga. Dosen yang menjadi narasumbernya adalah Vania Natali, S.Kom, M.T. Kesalahan penulisan dokumen skripsi yand didapat adalah sebagai berikut:
	
		\begin{itemize}
			\item Tahun skripsi pada cover skripsi \\
			Kode implementasi: VAN-01 \\
			Penulisan tahun skripsi harus sama dengan tahun dimana mahasiswa mengambil skripsi tersebut. Kesalahan yang pernah terjadi, yaitu mahasiswa salah menuliskan tahun skripsi. Meskipun terlihat sepele, namun hal ini perlu diperhatikan.			
			
			\item Konsistensi penggunaan kata \\
			Kode implementasi: VAN-02 \\
			Mahasiswa harus konsisten dalam penulisan kata, misalnya kata\textit{user} dan pengguna.  Mahasiswa harus memilih antara memakai \textit{user} atau pengguna.
			
			\item Penggunaan kata ganti orang \\
			Kode implementasi: VAN-03 \\
			Dalam penulisan dokumen skripsi, tidak boleh ada kata ganti orang. Jika karya nonilmiah lebih santai karena memakai gaya bahasa nonformal, maka berbeda dengan karya ilmiah. Karya ilmiah memiliki aturan baku dan menggunakan bahasa formal. 
			
		\end{itemize}
		
\end{enumerate}