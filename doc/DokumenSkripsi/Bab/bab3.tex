\chapter{Analisis Masalah}
\label{chap:analisis}

Pada bab 3 akan diuraikan tentang survei kesalahan umum.

\section{Survei Kesalahan Umum}
\label{sec:survei}
Pada subbab ini, akan diuraikan tentang survei yang dilakukan untuk mengumpulkan informasi yang dibutuhkan dalam pengembangan perangkat lunak. Informasi yang dicari adalah tentang kesalahan-kesalahan umum yang sering terjadi pada penulisan dokumen skripsi. Untuk mengumpulkan informasi tersebut, metode yang dipilih adalah melakukan survei. Dalam pelaksanaannya, survei dibagi menjadi dua, yaitu pengamatan beberapa sidang skripsi dan wawancara secara personal kepada dosen-dosen Informatika Unpar. 

\subsection{Pengamatan Sidang}
Pengamatan dilakukan pada sidang skripsi semester Ganjil 2018/2019, yang berlangsung pada bulan Mei 2019. Tidak semua sidang skripsi yang berlangsung diamati, dari 42 sidang skripsi hanya diambil 7 sidang skripsi saja. Hal tersebut dilakukan dengan pertimbangan dari ke-7 sidang skripsi tersebut diuji oleh dosen Informatika yang berbeda-beda. Namun ada beberapa dosen Informatika yang tidak masuk dalam pengamatan, karena tidak dapat menghadiri sidang yang diuji oleh dosen tersebut. Berikut adalah rincian dari sidang skripsi yang telah diamati:

\begin{table}[H]
	\caption {Tabel rincian sidang skripsi yang diamati} \label{tab:pengamatan}
	\begin{center}
	\begin{tabular}{|c|c|c|c|c|}
	\hline 	
	No & Tanggal & Mahasiswa & Penguji 1  & Penguji 2 \\ 
	\hline 
	1 & 15 Mei 2019 & Osfaldo Mickael & Bu Vania Natali & Pak Elisati Hulu \\ 
	\hline 
	2 & 16 Mei 2019 & Ricky Wahyudi & Bu Cecilia Esti N. & Bu Veronica Sri M. \\ 
	\hline 
	3 & 17 Mei 2019 & Billy Adiwijaya & Pak Kristopher David H. & Pak Elisati Hulu \\ 
	\hline 
	4 & 20 Mei 2019 & Ihsan Fajari & Bu Rosa De Lima E. P. & Bu Veronica Sri M. \\ 
	\hline 
	5 & 20 Mei 2019 & Muhammad Adrian & Bu Rosa De Lima E. P. & Pak Pascal Alfian N. \\ 
	\hline 
	6 & 23 Mei 2019 & Ellena Angelica & Bu Natalia & Bu Veronica Sri M. \\ 
	\hline 
	7 & 24 Mei 2019 & Evelyn Wijaya & Pak Candra Wijaya & Pak Raymond Chandra P. \\ 
	\hline 
	\end{tabular} 
\end{center} 
\end{table}

Dari ke-7 pengamatan yang dilakukan sesuai dengan tabel 3.1, didapatkan beberapa kesalahan-kesalahan yang terjadi dalam penulisan dokumen skripsi. Berikut adalah kesalahan-kesalahan yang disebutkan oleh dosen penguji pada sidang skripsi di atas beserta penjelasannya:

\begin{enumerate}
	\item Penulisan Judul \\
	Dalam penulisan judul, setiap huruf awal pada kata harus menggunakan huruf kapital. Hal ini berlaku untuk hampir semua jenis kata, seperti nama, tempat, sifat dan keterangan. Namun, ada beberapa pengecualian seperti preposisi (kata depan yang diikuti oleh kata lainnya), konjungsi (kata sambung), dan interjeksi (kata yang mengungkapkan seruan perasaan).
	
	\item Penggunaan imbuhan di- dan kata depan di\\
	Kesalahan ini merupakan kesalahan yang sering terjadi dalam penulisan dokumen skripsi. Penulisan imbuhan di- disatukan antara imbuhan dengan kata dasarnya. Untuk kata depan, penulisannya dipisah antara kata depan dengan kata berikutnya. Pada umumnya diikuti oleh keterangan tempat atau waktu.
	
	\item Referensi tidak dirujuk dengan baik \\
	
	\item Tidak ada spasi setelah tanda baca titik \\
	Sebuah kalimat akan diakhiri oleh tanda baca titik. Untuk melanjutkan kalimat baru, setelah tanda baca titik harus ada jarak 1 spasi untuk memisahkan kalimat sebelumnya dengan yang berikutnya. Terkadang mahasiswa lupa untuk memberikan jarak 1 spasi setelah tanda baca titik di akhir kalimat.

	\item Awal kalimat tidak menggunakan huruf kapital \\
	Setiap huruf pertama pada kata pertama dalam sebuah kalimat harus ditulis dengan huruf kapital. 	
	
	\item Terdapat ruang kosong yang besar \\
	Masalah ini sering ditemukan dalam penulisan dokumen skripsi, biasanya terjadi pada saat menyisipkan gambar atau tabel. Susunan atau ukuran gambar yang tidak tepat dapat mengakibatkan terciptanya ruang kosong yang besar.

	\item Tidak ada spasi antar kata \\
	Setiap kata dalam sebuah kalimat dipisahkan dengan jarak 1 spasi agar kalimat dapat dibaca dan dimengerti dengan baik.
	
	\item Kata tidak sesuai dengan KBBI \\
	KBBI menjadi kaidah dalam penulisan bahasa Indonesia. Mahasiswa terkadang salah memilih kata yang hendak ditulis dalam dokumen, padahal kata tersebut tidak sesuai dengan KBBI.

	\item Gambar tidak sesuai tempatnya \\
	Pada PDF Latex, biasanya kesalahan ini karena mahasiswa tidak memberikan tag kepada gambar tersebut. Hal ini mengakibatkan posisi gambar tidak terletak pada tempat yang seharusnya.

	\item Penulisan keterangan untuk gambar dan tabel \\
	Dalam penulisan dokumen skripsi, setiap gambar dan tabel perlu diberikan keterangan. Kesalahan yang sering terjadi adalah letak dari penulisan keterangan tersebut. Keterangan pada gambar posisinya ada di bawah gambar, sedangkan keterangan pada tabel posisinya ada di atas tabel.

	\item Jumlah subbab dalam 1 bab tidak boleh hanya 1 \\
	Dalam sebuah bab, biasanya jumlah subbab lebih dari 1. Kesalahan yang sering dilakukan oleh mahasiswa yaitu, hanya terdapat 1 subbab saja pada 1 bab. Apabila dalam bab tersebut hanya terdapat 1 subbab, lebih baik tidak perlu dibuat subbab.
	
\end{enumerate}

\subsection{Wawancara Personal}
Survei dilanjutkan dengan melakukan wawancara secara personal kepada dosen-dosen Informatika Unpar.

\begin{enumerate}
	\item Wawancara pertama dilakukan pada tanggal 9 Juli 2019. Dosen yang menjadi narasumbernya adalah Keenan. Hasil wawancara:
		
		\begin{itemize}
			\item Penggunaan cetak miring untuk kata Bahasa Inggris
			\item Kalimat pengantar untuk setiap subbab
			\item Kelengkapan data skripsi
		\end{itemize}
		
	\item Wawancara ke-2 dilakukan pada tanggal 9 Juli 2019 juga. Dosen yang menjadi narasumbernya adalah Pak Chandra Wijaya. Hasil wawancara:
	
		\begin{itemize}
			\item Letak keterangan untuk gambar dan tabel
			\item Kalimat pengantar untuk setia
			\item Kelengkapan data skripsi
		\end{itemize}
		
	\item Wawancara ke-3 dilakukan pada tanggal 15 Juli 2019 juga. Dosen yang menjadi narasumbernya adalah Pak Husnul Hakim. Hasil wawancara:
	
		\begin{itemize}
			\item Rujukan untuk gambar dan tabel
			\item Penulisan pseudocode
			\item Penulisan kata hubung
			\item No baris pada pseudocode
			\item Nama method serta masukan dan keluaran pada pseudocode
		\end{itemize}
		
	\item Wawancara ke-4 dilakukan pada tanggal 16 Juli 2019 juga. Dosen yang menjadi narasumbernya adalah Bu Vania Natali. Hasil wawancara:
	
		\begin{itemize}
			\item Tahun skripsi pada cover skripsi
			\item Konsistensi penggunaan kata
			\item Penggunaan kata ganti orang
		\end{itemize}
	
	\item Wawancara ke-5 dilakukan pada tanggal 16 Juli 2019 juga. Dosen yang menjadi narasumbernya adalah Bu Natalia. Hasil wawancara:
	
		\begin{itemize}
			\item Penulisan \textit{equation}
			\item Penulisan catatan kaki
			\item Penulisan daftar pustaka
		\end{itemize}
		
\end{enumerate}