\lstdefinelanguage{plaintext}{
  sensitive=false,
  comment=[l]{//},
  morecomment=[s]{/*}{*/},
  identifierstyle=\color{black},
  morestring=[b]',
  morestring=[b]"
}

\lstset
{ 
    language=plaintext,
    basicstyle=\footnotesize,
    numbers=left,
    stepnumber=1,
    showstringspaces=false,
    tabsize=1,
    breaklines=true,
    breakatwhitespace=false,
    frame=leftline
}

\chapter{Implementasi dan Pengujian}
\label{chap:implementasiDanPengujian}

Pada bab ini dibahas mengenai implementasi perangkat lunak dan pengujian perangkat lunak. Bagian implementasi berisi tentang lingkungan implementasi dan hasil implementasi. Bagian pengujian berisi tentang pengujian fungsional.

\section{Implementasi}
Pada bagian ini akan dijelaskan mengenai lingkungan yang digunakan untuk membangun perangkat lunak.

\subsection{Lingkungan Implementasi}
Berikut spesifikasi perangkat keras dan perangkat lunak yang digunakan dalam pengembangan aplikasi:

\begin{enumerate}
	\item Spesifikasi Perangkat Keras
	
		\begin{itemize}
			\item Perangkat: Laptop
			\item Processor: AMD Bristol Ridge Quad Core FX-9830P 3GHz
			\item RAM: 8GB
			\item GPU: Radeon RX 460
			\item Storage: Harddisk 1TB
		\end{itemize}		

	\item Spesifikasi Perangkat Lunak

		\begin{itemize}
			\item Sistem Operasi Windows 10 64-bit
			\item PHP 7.3.5 (cli)
			\item Composer versi 1.8.5
			\item Sublime Text versi 3.2.1
		\end{itemize}	
	
\end{enumerate}

\subsection{Hasil Implementasi}
Hasil dari implementasi adalah sebuah aplikasi yang dibangun dengan bahasa pemrograman \textit{PHP} dan dijalankan melalui terminal. Aplikasi ini akan memeriksa dokumen skripsi dan mengeluarkan laporan kesalahan yang didapatkan. 

\begin{lstlisting}[caption={Perintah untuk menjalankan aplikasi}			\label{lst:command},language=plaintext,xleftmargin=.15\textwidth] 
D:\Skripsi\src\pdf_checker>php main.php ../res/skripsi.pdf
\end{lstlisting}

\section{Pengujian}
Pada bagian ini akan dijelaskan mengenai pengujian yang dilakukan pada perangkat lunak. Pengujian yang dilakukan meliputi pengujian fungsional dan eksperimental.

\subsection{Pengujian Fungsional}

\subsection{Pengujian Eksperimental}
