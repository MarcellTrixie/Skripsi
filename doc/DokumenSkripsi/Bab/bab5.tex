\lstdefinelanguage{plaintext}{
  sensitive=false,
  comment=[l]{//},
  morecomment=[s]{/*}{*/},
  identifierstyle=\color{black},
  morestring=[b]',
  morestring=[b]"
}

\lstset
{ 
    language=plaintext,
    basicstyle=\footnotesize,
    numbers=left,
    stepnumber=1,
    showstringspaces=false,
    tabsize=1,
    breaklines=true,
    breakatwhitespace=false,
    frame=leftline,
}

\chapter{Implementasi dan Pengujian}
\label{chap:implementasiDanPengujian}

Pada bab ini dibahas mengenai implementasi perangkat lunak dan pengujian yang dilakukan terhadap perangkat lunak tersebut. Lingkungan implementasi, yang meliputi perangkat keras dan perangkat lunak, serta hasil implementasi akan dijelaskan pada bab ini. Selain Pengujian yang dilakukan pada skripsi ini, yang meliputi pengujian fungsional dan eksperimental akan dijelaskan pada bab ini.

\section{Implementasi}
Pada bagian ini akan dijelaskan mengenai lingkungan yang digunakan untuk membangun perangkat lunak beserta hasil implementasinya.

\subsection{Lingkungan Implementasi}
Berikut spesifikasi perangkat keras dan perangkat lunak yang digunakan dalam pembangunan pada skripsi ini:

\begin{enumerate}
	\item Spesifikasi Perangkat Keras
	
		\begin{itemize}
			\item Perangkat: Laptop
			\item Processor: AMD Bristol Ridge Quad Core FX-9830P 3GHz
			\item RAM: 8GB
			\item GPU: Radeon RX 460
			\item Storage: Harddisk 1TB
		\end{itemize}		

	\item Spesifikasi Perangkat Lunak

		\begin{itemize}
			\item Sistem Operasi Windows 10 64-bit
			\item PHP 7.3.5 (cli)
			\item Composer versi 1.8.5
			\item Sublime Text versi 3.2.1
		\end{itemize}	
	
\end{enumerate}

\subsection{Hasil Implementasi}
Perangkat lunak dibangun menggunakan bahasa pemrograman \textit{PHP} dan \textit{library PdfParser}. Perangkat lunak tidak memiliki \textit{Graphical User Interface}, sehingga seluruh kegiatan dilakukan melalui terminal. Perangkat lunak akan menerima input berupa file PDF skripsi yang disimpan pada folder yang telah disediakan, dan mengeluarkan laporan kesalahan pada terminal.

\begin{lstlisting}[caption={Perintah yang digunakan untuk menjalankan perangkat lunak}	\label{lst:command},language=plaintext,xleftmargin=.3\textwidth] 
php main.php ../res/nama_file.pdf
\end{lstlisting}
\medskip

Listing \ref{lst:command} merupakan perintah yang perlu dituliskan pada terminal, untuk menjalankan perangkat lunak. Kelas Main menjadi kelas yang digunakan untuk menjalankan seluruh proses yang berjalan dalam perangkat lunak. File PDF skripsi yang akan diperiksa harus berada di folder yang telah disediakan, yaitu pada folder Skripsi$\backslash$src$\backslash$res. Nama file yang digunakan pada umumnya sesuai dengan \textit{template} skripsi yang diberikan, yaitu ''skripsi.pdf''. Namun pengguna juga dapat menggunakan nama yang berbeda, yang paling utama file tersebut memiliki ekstensi PDF.

\begin{lstlisting}[caption={Perintah yang digunakan untuk menjalankan perangkat lunak}	\label{lst:error_report},language=plaintext,xleftmargin=.05\textwidth] 
ERROR 1
==================
Error Code: PS-03
Note: Perhatikan spasi setelah tanda baca.
Excerpt: Namun dalam penulisannya, peserta skripsi sering melakukan kesalahan 
kecil yang tidak dapat diabaikan.kesalahan sering terjadi dalam penggunaan 
imbuhan, kata keterangan, penulisan kata dan sebagainya

ERROR 2
==================
Error Code: PS-03
Note: Perhatikan spasi setelah tanda baca.
Excerpt: Regex biasanya dimanfaatkan untuk memverifikasi kecocokan antara input 
dengan pola teks,untuk menemukan teks yang cocok dengan pola dalam teks yang 
lebih besar, untuk mengganti teks yang cocok dengan pola dengan teks lain atau 
menyusun ulang bit dari teks yang cocok dan untuk membagi sebuah blok teks
menjadi beberapa subteks

ERROR 3
==================
Error Code: PS-05
Note: Huruf pertama pada kalimat ini tidak menggunakan huruf kapital
Excerpt: Namun dalam penulisannya, peserta skripsi sering melakukan kesalahan
kecil yang tidak dapat diabaikan.kesalahan sering terjadi dalam penggunaan 
imbuhan, kata keterangan, penulisan kata dan sebagainya
\end{lstlisting}
\medskip

Listing \ref{lst:error_report} merupakan hasil laporan yang dikeluarkan oleh perangkat lunak melalui \textit{terminal windows}. Informasi yang diberikan oleh laporan tersebut yaitu, kode kesalahan, jenis kesalahan yang ditemukan dan kesalahan yang ditemukan. Laporan kesalahan yang dikeluarkan sudah diurutkan dari fitur pertama hingga terakhir, yaitu dari fitur PS-01 hingga NAT-03.

\section{Pengujian Fungsional}
\label{PengFung}
Pengujian fungsional bertujuan untuk menguji fungsionalitas perangkat lunak. Perangkat lunak memiliki 8 fitur yang telah diimplementasikan. Fitur-fitur tersebut akan diuji untuk melihat kebenaran dan kesesuaian fitur tersebut dengan yang diharapkan. Untuk melakukan pengujian ini, perangkat lunak akan dijalankan sebanyak jumlah fitur yang ada. Setiap pengujian yang dilakukan, fitur yang diaktifkan hanya 1 saja secara bergantian. Hal ini dilakukan hingga seluruh fitur telah diuji.

Pada pengujian ini, perangkat lunak akan diuji dengan 2 buah kasus uji. Kedua kasus uji tersebut menggunakan \textit{template} dokumen skripsi Informatika Unpar. Mode dokumen yang digunakan adalah mode \textit{final}. Kedua \textit{test case} tersebut diberi nama 'TC\_PF\_01.pdf'' dan ''TC\_PF\_02.pdf''. Isi dari kedua file tersebut serupa, namun pada file ''TC\_PF\_02.pdf'' sudah disisipkan kesalahan-kesalahan yang dapat dideteksi oleh setiap fitur yang ada. Berikut ini adalah rincian dari kesalahan-kesalahan yang dimasukan ke dalam kasus uji tersebut.

\begin{enumerate}
	\item Pada halaman cover bahasa Indonesia dan bahasa Inggris, data yang meliputi judul, nama mahasiswa, NPM mahasiswa, dan yang lainnya tidak diisi. Hal ini dilakukan untuk menguji fitur pemeriksa kelengkapan data skripsi (KAL-03).
	
	\item Untuk menguji fitur pemeriksa jumlah sub bab atau sub sub bab (PS-09), pada beberapa bab hanya memiliki sebuah sub bab atau sebuah sub sub bab.
	
	\item Pada bab 1, beberapa karakter pertama dalam awal kalimat menggunakan huruf kecil. Hal ini dilakukan untuk menguji fitur pemeriksa huruf kapital (PS-05).
	
	\item Pada bab 2, terdapat teori yang referensinya tidak dirujuk dengan baik. File referensi.bib tidak diikutsertakan pada saat file latex dieksekusi. Hal ini dilakukan untuk menguji fitur pemeriksa referensi (NAT-01).
	
	\item Pada bab 1 dan 2, terdapat beberapa kata yang tidak diberikan karakter spasi sebelum ataupun setelah tanda baca. Hal ini dilakukan untuk menguji fitur pemeriksa karakter spasi sebelumm atau setelah tanda baca (PS-03).	

	\item Pada bab 3, terdapat beberapa kata yang diketik tidak sesuai dengan kamus bahasa Indonesia. Hal ini dilakukan untuk menguji fitur pemeriksa kata (PS-01). Selain itu pada bab ini tidak diberikan kata pengantar sebelum memulai subbab, untuk menguji fitur pemeriksa kata pengantar pada bab (KAL-02).
		
	\item Pada bab 6, terdapat kalimat yang disisipkan kata ganti orang , untuk menguji fitur pemeriksa kata ganti orang (VAN-03).
\end{enumerate}

\subsection{Menguji fitur PS-01}
Hasil yang diharapkan dari pengujian fitur ini adalah perangkat lunak dapat menemukan kata-kata yang tidak terdapat pada kamus bahasa Indonesia \textit{LibreOffice}. Berikut ini adalah hasil dari pengujian yang telah dilakukan:

\begin{enumerate}

	\item Kasus uji TC\_PF\_01.pdf
	
\begin{lstlisting}[caption={Laporan Kesalahan Kasus Uji TC\_PF\_01}	\label{lst:ps-01-1},language=php,xleftmargin=.0\textwidth]
ERROR 1
==================
Error Code: PS-01
Note: Ditemukan penulisan kata yang tidak sesuai dengan kamus
Excerpt: 

PENDAHULUANPada, dijelaskan, mengenai, penulisan, rumusan, tujuan, batasan, 
penelitian, merupakan, karangan, ditulis, sebagai, bagian, persya, ratan,
pendidikan, akademiknya, penulisannya, peserta, melakukan, kesalahan,
diabaikan, Kesalahan, terjadi, penggunaan, imbuhan, keterangan, 
sebagainya, seharusnya, diperiksa, diminimalisir, bimbingan, pembimbing, 
dimanfaatkan, membahas, konten, dibanding, memeriksa, tersebut, dibuat, 
sebuah, pemeriksaan, berasal, dilakukan, Unpar, diseleksi, 
diimplementasikan, secara, dijalankan, melalui, command, Windows, 
menerima, masukan, berupa, le, PDF, menampilkan, laporan, berisi, 
ditemukan, Rumusan, Berdasarkan, dirumuskan, berikut, membuat, Tujuan, 
adalah, membangun, LANDASAN, TEORIPada, dibahas, landasan, expression, 
library, LibreO, Expression, regex, tertentu, digunakan, pemrograman, 
Regex, biasanya, memveri, kecocokan, input, menemukan, mengganti, 
menyusun, membagi, menjadi, subteks, pencocokan, misalnya, validasi, 
string, username, password, mail, IP, Pemanfaatan, menyederhanakan, 
pemrosesan, kehidupan, sehari, mencerminkan, disebut, keteraturan, 
menggunakan, Deterministic, Finite, DFA, nite, state, machineyang, 
backtracking, berbagai, satunya, Perl, Compa, tible, PCRE, serangkaian, 
menerapkan, sintaks, perbedaan, Perlversi, Metakarakter, dibedakan, 
berdasarkan, posisinya, metakarakter, outside, square, brackets, 
fungsinya, berbeda, terdapat, sedangkan, inside, bracketsterdapat, 
Karakter, karakter, memiliki, spesi, k, dikelompokan, setiap, didalam, 
bracket, mendukung, POSIX, penggunaannya, diantara, Sebagai, alnum, 
keputusan, UmumPada, mengumpulkan, dibutuhkan, pengembangan, dicari, 
dipilih, pelaksanaannya, dibagi, pengamatan, Pengamatan, berlangsung, Mei, 
diamati, melainkan, diambil, pertimbangan, diuji, menghadiri, disajikan, 
Penguji, Osfaldo, Mickael, Oktavianus, Naibaho, Penjualan, Vania, Natali, 
S, M, T, Elisati, Ricky, Wahyudi, Menggunakan, Fitur, Surf, Dr, rer, nat, 
Cecilia, Esti, Nugraheni, ST, MT, Ir, Veronica, Moer, tini, selanjutnya, 
diminta, PERANCANGANPada, perancangan, dibangun, meliputi, algoritma, 
pengecekan, Perancangan, rancangan, Rancangan, ditunjukan, Pemeriksa, 
PENGUJIANPada, pengujian, terhadap, Lingkungan, Pengujian, lingkungan, 
beserta, implementasinya, Berikut, pembangunan, Spesi, Processor, AMD, 
Bristol, Ridge, Quad, Core, FX, P, GHz, GB, GPU, Radeon, RX, Storage, 
Harddisk, TB, PHP, cli, Composer, Sublime, Text, PdfParser, Graphical, 
User, Interface, kegiatan, disimpan, disediakan, mengeluarkan, Listing, 
menjalankan, phpmain, php, r, s, n, m, f, i, l, p, d, KESIMPULAN, 
SARANPada, kesimpulan, Kesimpulan
\end{lstlisting}
	
	\item Kasus uji TC\_PF\_02.pdf

\begin{lstlisting}[caption={Laporan Kesalahan Kasus Uji TC\_PF\_02}	\label{lst:ps-01-2},language=php,xleftmargin=.0\textwidth]
ERROR 1
==================
Error Code: PS-01
Note: Ditemukan penulisan kata yang tidak sesuai dengan kamus
Excerpt: 

PENDAHULUANpada, dijelaskan, mengenai, penulisan, rumusan, tujuan, 
batasan, penelitian, merupakan, karangan, ditulis, sebagai, bagian, 
persya, ratan, pendidikan, akademiknya, penulisannya, peserta, melakukan, 
kesalahan, diabaikan, terjadi, penggunaan, imbuhan, keterangan, 
sebagainya, seharusnya, diperiksa, diminimalisir, bimbingan, pembimbing, 
dimanfaatkan, membahas, konten, dibanding, memeriksa, tersebut, dibuat, 
sebuah, pemeriksaan, berasal, dilakukan, Unpar, diseleksi, 
diimplementasikan, secara, dijalankan, melalui, command, Windows, 
menerima, masukan, berupa, le, PDF, menampilkan, laporan, berisi, 
ditemukan, Rumusan, Berdasarkan, dirumuskan, berikut, membuat, Tujuan, 
adalah, membangun, LANDASAN, TEORIPada, dibahas, landasan, expression, 
library, LibreO, Expression, regex, tertentu, digunakan, pemrograman, 
Regex, biasanya, memveri, kecocokan, input, menemukan, mengganti, 
menyusun, membagi, menjadi, subteks, pencocokan, misalnya, validasi, 
string, username, password, mail, IP, Pemanfaatan, menyederhanakan, 
pemrosesan, kehidupan, sehari, mencerminkan, disebut, keteraturan, 
menggunakan, Deterministic, Finite, DFA, nite, state, machineyang, 
backtracking, berbagai, satunya, Perl, Compa, tible, PCRE, serangkaian, 
menerapkan, sintaks, perbedaan, Perlversi, Metakarakter, dibedakan, 
berdasarkan, posisinya, metakarakter, outside, square, brackets, 
fungsinya, berbeda, terdapat, sedangkan, inside, bracketsterdapat, 
Kesalahan, UmumPda, baggian, mengumpulkan, dibutuhkan, pengembangan, 
Infrmasi, yng, dicari, ksalahan, umumm, yaang, serring, terjaadi, paada, 
metod, ang, dipilih, pelaksanaannya, dibagi, menjhadi, pengamatan, 
Pengamatan, berlangsung, Mei, diamati, melainkan, diambil, pertimbangan, 
diuji, menghadiri, disajikan, Penguji, Osfaldo, Mickael, Oktavianus, 
Naibaho, Penjualan, Vania, Natali, S, M, T, Elisati, Ricky, Wahyudi, 
Menggunakan, Fitur, Surf, Dr, rer, nat, Cecilia, Esti, Nugraheni, ST, MT, 
Ir, Veronica, Moer, tini, PERANCANGANPada, perancangan, dibangun, 
meliputi, algoritma, pengecekan, Perancangan, rancangan, Rancangan, 
ditunjukan, Pemeriksa, PENGUJIANPada, pengujian, terhadap, Lingkungan, 
Pengujian, lingkungan, beserta, implementasinya, Berikut, spesi, 
pembangunan, Spesi, Processor, AMD, Bristol, Ridge, Quad, Core, FX, P, 
GHz, GB, GPU, Radeon, RX, Storage, Harddisk, TB, PHP, cli, Composer, 
Sublime, Text, KESIMPULAN, SARANPada, kesimpulan, Kesimpulan, berikan
\end{lstlisting}
\end{enumerate}

Pada kasus uji TC\_PF\_02.pdf disisipkan 10 buah kesalahan penulisan kata. Kata-kata yang disisipkan dalam kasus uji tersebut, yaitu: ''Pda'', ''baggian'', ''Infrmasi'', ''yng'', ''ksalahan'', ''umumm'', ''yaang'', ''serring'', ''terjaadi'', ''paada'', ''metod'', ''ang'' dan ''menjadhi''. Seluruh kesalahan yang disisipkan berhasil diperiksa dan dikeluarkan pada laporan yang ditunjukan oleh Listing \ref{lst:ps-01-2}. Tingkat akurasi untuk fitur ini adalah 100\%. Namun, ada beberapa hal yang belum dapat dilakukan fitur ini, berikut adalah contohnya:

\begin{itemize}
	\item Mendeteksi nama orang. \newline
	Pada listing \ref{lst:ps-01-2} terdapat beberapa nama orang yang termasuk ke dalam kesalahan, contohnya Osfaldo, Mickael, Ricky dan yang lainnya.
	
	\item Mendeteksi kata yang menggunakan bahasa asing \newline
	Pada listing \ref{lst:ps-01-2} terdapat beberapa bahasa asing yang termasuk ke dalam kesalahan, contohnya \textit{outside}, \textit{square}, \textit{brackets} dan yang lainnya.
	
	\item Mendeteksi kata yang memiliki imbuhan \newline
	Pada listing \ref{lst:ps-01-2} terdapat beberapa kata berimbuhan yang termasuk ke dalam kesalahan, contohnya menggunakan, pengecekan, perancangan dan yang lainnya.
	
	\item Mendeteksi kata yang saling berdempetan \newline
	Pada listing \ref{lst:ps-01-2} terdapat beberapa kata berdempetan yang termasuk ke dalam kesalahan, contohnya PERANCANGANPada, UmumPda dan yang lainnya.
	
	\item Mendeteksi kata yang tidak utuh \newline
	Pada listing \ref{lst:ps-01-2} terdapat beberapa kata yang tidak utuh (terpotong oleh ''-'') yang termasuk ke dalam kesalahan, contohnya persya-ratan menjadi persya dan ratan.
\end{itemize}

\subsection{Menguji fitur PS-03}
Hasil yang diharapkan dari pengujian fitur ini adalah perangkat lunak dapat menemukan kata-kata yang tidak diberi spasi setelah tanda baca. Berikut ini adalah hasil dari pengujian yang telah dilakukan:

\begin{enumerate}
	\item Kasus uji TC\_PF\_01.pdf
	
\begin{lstlisting}[caption={PLaporan Kesalahan Kasus Uji TC\_PF\_01}	\label{lst:ps-03-1},language=php,xleftmargin=.0\textwidth]
ERROR 1
==================
Error Code: PS-03
Note: Perhatikan spasi setelah tanda baca.
Excerpt: Listing 5.1: Perintah yang digunakan untuk menjalankan perangkat lunak
1phpmain.php
\end{lstlisting}
	
	\item Kasus uji TC\_PF\_02.pdf
	
\begin{lstlisting}[caption={Laporan Kesalahan Kasus Uji TC\_PF\_02}	\label{lst:ps-03-2},language=php,xleftmargin=.0\textwidth]
ERROR 1
==================
Error Code: PS-03
Note: Perhatikan spasi setelah tanda baca.
Excerpt: Namun dalam penulisannya, peserta skripsi sering melakukan kesalahan 
kecil yang tidak dapat diabaikan.kesalahan sering terjadi dalam penggunaan 
imbuhan, kata keterangan, penulisan kata dan sebagainya

ERROR 2
==================
Error Code: PS-03
Note: Perhatikan spasi setelah tanda baca.
Excerpt: Regex biasanya dimanfaatkan untuk memverifikasi kecocokan antara input 
dengan pola teks,untuk menemukan teks yang cocok dengan pola dalam teks yang 
lebih besar, untuk mengganti teks yang cocok dengan pola dengan teks lain atau 
menyusun ulang bit dari teks yang cocok dan untuk membagi sebuah blok teks 
menjadi beberapa subteks
\end{lstlisting}
\end{enumerate}

Pada kasus uji TC\_PF\_02.pdf disisipkan 2 buah kesalahan dalam penggunaan spasi setelah tanda baca. Kesalahan tersebut terdapat pada sub bab 1.1 baris ke-3 dan sub bab 2.1 baris ke-3. Kedua kesalahan tersebut berhasil diperiksa dan dikeluarkan pada laporan yang ditunjukan oleh Listing \ref{lst:ps-03-2}. Tingkat akurasi untuk fitur ini adalah 100\%. Namun, ada beberapa hal yang belum dapat dilakukan fitur ini, berikut adalah contohnya:

\begin{itemize}
	\item Fitur ini masih belum bisa membedakan penggunaan tanda titik pada gelar pendidikan, perangkat lunak mendeteksi hal tersebut menjadi sebuah kesalahan.
	
	\item Pada listing \ref{lst:ps-03-1}, fitur ini juga belum dapat membedakan penggunaan tanda titik pada nama file. Hal ini juga berlaku untuk kata yang merujuk tabel, gambar dan lain-lain. Misalnya Tabel 4.1, Gambar 2.5 dan lain-lain.
\end{itemize}

\subsection{Menguji fitur PS-05}
Hasil yang diharapkan dari pengujian fitur ini adalah perangkat lunak dapat menemukan karakter pertama yang tidak menggunakan huruf kapital pada sebuah kalimat. Berikut ini adalah hasil dari pengujian yang telah dilakukan:

\begin{enumerate}
	\item Kasus uji TC\_PF\_01.pdf \newline
	Pada kasus uji ini, perangkat lunak tidak menemukan kesalahan dalam dokumen skripsi.
	
	\item Kasus uji TC\_PF\_02.pdf
	
\begin{lstlisting}[caption={Laporan Kesalahan Kasus Uji TC\_PF\_02}	\label{lst:ps-05-2},language=php,xleftmargin=.0\textwidth]
ERROR 1
==================
Error Code: PS-05
Note: Huruf pertama pada kalimat ini tidak menggunakan huruf kapital
Excerpt: Namun dalam penulisannya, peserta skripsi sering melakukan kesalahan 
kecil yang tidak dapat diabaikan.kesalahan sering terjadi dalam penggunaan 
imbuhan, kata keterangan, penulisan kata dan sebagainya

ERROR 2
==================
Error Code: PS-05
Note: Huruf pertama pada kalimat ini tidak menggunakan huruf kapital
Excerpt: pada saat bimbingan, waktu dosen pembimbing lebih baik dimanfaatkan 
untuk membahas konten dibanding memeriksa kesalahan-kesalahan tersebut

ERROR 3
==================
Error Code: PS-05
Note: Huruf pertama pada kalimat ini tidak menggunakan huruf kapital
Excerpt: dari masalah tersebut dapat dibuat sebuah aplikasi untuk melakukan 
pemeriksaan pada dokumen skripsi

ERROR 4
==================
Error Code: PS-05
Note: Huruf pertama pada kalimat ini tidak menggunakan huruf kapital
Excerpt: kesalahan yang akan diperiksa berasal dari survei yang dilakukan 
kepada dosen-dosen Informatika Unpar

ERROR 5
==================
Error Code: PS-05
Note: Huruf pertama pada kalimat ini tidak menggunakan huruf kapital
Excerpt: hasil dari survei tersebut akan diseleksi untuk diimplementasikan ke 
dalam aplikasi

ERROR 6
==================
Error Code: PS-05
Note: Huruf pertama pada kalimat ini tidak menggunakan huruf kapital
Excerpt: aplikasi sederhana ini dapat dimanfaatkan oleh mahasiswa Informatika 
Unpar secara mandiri

ERROR 7
==================
Error Code: PS-05
Note: Huruf pertama pada kalimat ini tidak menggunakan huruf kapital
Excerpt: aplikasi ini dijalankan melalui melalui terminal command Windows 

ERROR 8
==================
Error Code: PS-05
Note: Huruf pertama pada kalimat ini tidak menggunakan huruf kapital
Excerpt: aplikasi menerima masukan berupa file PDF skripsi dan menampilkan 
laporan yang berisi kesalahan-kesalahan yang ditemukan pada dokumen skripsi
\end{lstlisting}
\end{enumerate}

Pada kasus uji TC\_PF\_02.pdf disisipkan 9 buah kesalahan dalam penggunaan huruf kapital pada awal kalimat. Kesalahan tersebut disisipkan pada 9 kalimat yang ada pada bab 1. Listing \ref{lst:ps-05-2} hanya menemukan 8 buah kesalahan. Satu kesalahan yang tidak ditemukan berada pada kata pengantar setelah penulisan bab dan judulnya. Kesalahan ini tidak terdeteksi karena hasil ekstrak judul bab dan kata pertama pada kalimat pengantar tergabung menjadi satu. Tingkat akurasi untuk fitur ini adalah 88,89\%.

\subsection{Menguji fitur PS-09}
Hasil yang diharapkan dari pengujian fitur ini adalah perangkat lunak dapat menemukan bab atau sub bab yang hanya memiliki satu sub bab atau sub sub bab. Berikut ini adalah hasil dari pengujian yang telah dilakukan:

\begin{enumerate}
	\item Kasus uji TC\_PF\_01.pdf \newline
	Pada kasus uji ini, perangkat lunak tidak menemukan kesalahan dalam dokumen skripsi.
	
	\item Kasus uji TC\_PF\_02.pdf
	
\begin{lstlisting}[caption={Laporan Kesalahan Kasus Uji TC\_PF\_02}	\label{lst:ps-09-2},language=php,xleftmargin=.0\textwidth]
ERROR 1
==================
Error Code: PS-09
Note: Bab/Subbab 2.1 ini hanya terdapat 1 sub bab/sub sub bab

ERROR 2
==================
Error Code: PS-09
Note: Bab/Subbab 2.1.1 ini hanya terdapat 1 sub bab/sub sub bab

ERROR 3
==================
Error Code: PS-09
Note: Bab/Subbab 3.1 ini hanya terdapat 1 sub bab/sub sub bab

ERROR 4
==================
Error Code: PS-09
Note: Bab/Subbab 3.1.1 ini hanya terdapat 1 sub bab/sub sub bab

ERROR 5
==================
Error Code: PS-09
Note: Bab/Subbab 4.1 ini hanya terdapat 1 sub bab/sub sub bab

ERROR 6
==================
Error Code: PS-09
Note: Bab/Subbab 5.1 ini hanya terdapat 1 sub bab/sub sub bab

ERROR 7
==================
Error Code: PS-09
Note: Bab/Subbab 5.1.1 ini hanya terdapat 1 sub bab/sub sub bab
\end{lstlisting}
\end{enumerate}

Laporan yang ditunjukan pada Listing \ref{lst:ps-09-2} sudah sesuai dengan yang diharapkan.

Pada kasus uji TC\_PF\_02.pdf disisipkan 7 buah kesalahan. Kesalahan tersebut berkaitan dengan jumlah sub bab atau sub sub bab yang ada. Ada 4 bab yang hanya memiliki 1 sub bab, yaitu bab 2, 3, 4 dan 5. Selain itu ada 3 sub bab yang hanya memiliki 1 sub sub bab, yaitu sub bab 2.1, 3.1 dan 5.1. Pada Listing \ref{lst:ps-05-2} seluruh kesalahan berhasil ditemukan. Tingkat akurasi untuk fitur ini adalah 100\%.

\subsection{Menguji fitur KAL-02}
Hasil yang diharapkan dari pengujian fitur ini adalah perangkat lunak dapat menemukan bab yang tidak diberikan kata pengantar. Berikut ini adalah hasil dari pengujian yang telah dilakukan:

\begin{enumerate}
	\item Kasus uji TC\_PF\_01.pdf \newline
	Pada kasus uji ini, perangkat lunak tidak menemukan kesalahan dalam dokumen skripsi.
	
	\item Kasus uji TC\_PF\_02.pdf
	
\begin{lstlisting}[caption={Laporan Kesalahan Kasus Uji TC\_PF\_02}	\label{lst:kal-02-2},language=php,xleftmargin=.0\textwidth]
ERROR 1
==================
Error Code: KAL-02
Note: Berilah kata pengantar untuk setiap bab
Excerpt: BAB 3 ANALISIS MASALAH 3.1Survei Kesalahan UmumPda baggian ini akan 
dijelaskan tentang survei yang dilakukan untuk mengumpulkan informasi yang 
dibutuhkan dalam pengembangan perangkat lunak
\end{lstlisting}
\end{enumerate}

Pada kasus uji TC\_PF\_02.pdf disisipkan sebuah kesalahan. Kesalahan tersebut disisipkan pada bab 3. Pada bab 3 setelah penulisan bab dan judul bab, tidak diberikan kata pengantar. Pada Listing \ref{lst:kal-02-2} kesalahan tersebut berhasil ditemukan. Tingkat akurasi untuk fitur ini adalah 100\%.

\subsection{Menguji fitur KAL-03}
Hasil yang diharapkan dari pengujian fitur ini adalah perangkat lunak dapat menemukan data skripsi yang belum diisi. Berikut ini adalah hasil dari pengujian yang telah dilakukan:

\begin{enumerate}
	\item Kasus uji TC\_PF\_01.pdf \newline
	Pada kasus uji ini, perangkat lunak tidak menemukan kesalahan dalam dokumen skripsi.
	
	\item Kasus uji TC\_PF\_02.pdf
	
\begin{lstlisting}[caption={Laporan Kesalahan Kasus Uji TC\_PF\_02}	\label{lst:kal-03-2},language=php,xleftmargin=.0\textwidth]
ERROR 1
==================
Error Code: KAL-03
Note: Ada data skripsi yang belum dilengkapi pada halaman cover
Excerpt: SKRIPSI/TUGAS AKHIR JUDUL BAHASA INDONESIA Nama Lengkap NPM: 10 digit 
NPM UNPAR PROGRAM STUDI MATEMATIKA/FISIKA/TEKNIK INFORMATIKA FAKULTAS 
TEKNOLOGI INFORMASI DAN SAINS UNIVERSITAS KATOLIK PARAHYANGAN tahun FINAL 
PROJECT/UNDERGRADUATE THESIS JUDUL BAHASA INGGRIS Nama Lengkap NPM: 10 digit 
NPM UNPAR DEPARTMENT OF MATHEMATICS/PHYSICS/INFORMATICS FACULTY OF INFORMATION 
TECHNOLOGY AND SCIENCES PARAHYANGAN CATHOLIC UNIVERSITY tahun 
\end{lstlisting}
\end{enumerate}

Pada kasus uji TC\_PF\_02.pdf disisipkan dua buah kesalahan. Kesalahan tersebut disisipkan pada halaman cover Bahasa Indonesia dan Bahasa Inggris. Isi dari file data.tex tidak diisi satupun, sehingga masih berisi kode \textit{template} yang diberikan. Pada Listing \ref{lst:kal-03-2} kesalahan tersebut berhasil ditemukan. Tingkat akurasi untuk fitur ini adalah 100\%.

\subsection{Menguji fitur NAT-01}
Hasil yang diharapkan dari pengujian fitur ini adalah perangkat lunak dapat menemukan referensi yang tidak dirujuk dengan baik. Berikut ini adalah hasil dari pengujian yang telah dilakukan:

\begin{enumerate}
	\item Kasus uji TC\_PF\_01.pdf \newline
	Pada kasus uji ini, perangkat lunak tidak menemukan kesalahan dalam dokumen skripsi.
	
	\item Kasus uji TC\_PF\_02.pdf
	
\begin{lstlisting}[caption={Laporan Kesalahan Kasus Uji TC\_PF\_02}	\label{lst:nat-01-2},language=php,xleftmargin=.0\textwidth]
ERROR 1
==================
Error Code: NAT-01
Note: Referensi tidak dirujuk dengan baik, lakukan perintah PDFLatex->BibTex->
PDFLatex->PDFLatex untuk memperbaikinya
Excerpt: 2.1 Regular Expression Regular expression (regex ) [ ?] adalah jenis 
pola teks tertentu yang dapat digunakan pada banyak aplikasi modern dan bahasa 
pemrograman


ERROR 2
==================
Error Code: NAT-01
Note: Referensi tidak dirujuk dengan baik, lakukan perintah PDFLatex->BibTex->
PDFLatex->PDFLatex untuk memperbaikinya
Excerpt: PCRE [ ?] adalah serangkaian fungsi yang menerapkan pencocokan pola 
regex dengan menggunakan sintaks dan semantik yang sama dengan bahasa 
pemrograman Perl 5, meskipun ada beberapa sedikit perbedaan
\end{lstlisting}
\end{enumerate}

Pada kasus uji TC\_PF\_02.pdf disisipkan dua buah kesalahan. Kesalahan tersebut disisipkan pada bab 2. Pada bab 2, terdapat 2 buah landasan teori yang referensinya tidak muncul. Pada Listing \ref{lst:nat-01-2} kedua kesalahan tersebut berhasil ditemukan. Tingkat akurasi untuk fitur ini adalah 100\%.

\subsection{Menguji fitur VAN-03}
Hasil yang diharapkan dari pengujian fitur ini adalah perangkat lunak dapat menemukan kata ganti orang pada kalimat. Berikut ini adalah hasil dari pengujian yang telah dilakukan:

\begin{enumerate}
	\item Kasus uji TC\_PF\_01.pdf \newline
	Pada kasus uji ini, perangkat lunak tidak menemukan kesalahan dalam dokumen skripsi.
	
	\item Kasus uji TC\_PF\_02.pdf
	
\begin{lstlisting}[caption={Laporan Kesalahan Kasus Uji TC\_PF\_02}	\label{lst:van-03-2},language=php,xleftmargin=.0\textwidth]
ERROR 1
==================
Error Code: VAN-03
Note: Kalimat ini mengandung kata ganti orang
Excerpt: 6.1Kesimpulan Kesimpulan yang saya dapat dari pengembangan perangkat 
lunak ini adalah sebagai berikut

ERROR 2
==================
Error Code: VAN-03
Note: Kalimat ini mengandung kata ganti orang
Excerpt: 6.2Saran Saran yang saya berikan untuk pengembangan perangkat lunak 
ini adalah sebagai berikut
\end{lstlisting}
\end{enumerate}

Pada kasus uji TC\_PF\_02.pdf disisipkan dua buah kesalahan. Kesalahan tersebut disisipkan pada bab 6. Pada bab 6, terdapat 2 sub bab untuk kesimpulan dan saran. Masing-masing sub bab disisipkan kata ganti orang sebanyak 1 kata. Pada Listing \ref{lst:van-03-2} kedua kesalahan tersebut berhasil ditemukan. Tingkat akurasi untuk fitur ini adalah 100\%.

\subsection{Kesimpulan Pengujian Fungsional}

Pengujian fungsional telah dilakukan dengan menggunakan 2 kasus uji, berupa dokumen skripsi yang dibuat menggunakan \textit{template} skripsi FTIS UNPAR. Pada pengujian ini dapat ditarik beberapa kesimpulan, yaitu sebagai berikut:

\begin{itemize}
	\item Fitur-fitur yang ada pada perangkat lunak dapat mengeluarkan laporan kesalahan sesuai dengan hasil yang diharapkan. Hal tersebut dapat dilihat dari presentase keakuratan dari setiap fitur yang sangat tinggi.
	
	\item Dari 8 fitur yang diimplementasi, masih ada 2 fitur yang perlu disempurnakan lagi. Masih ada yang belum bisa ditangani seperti yang sudah dijelaskan pada sub bab \ref{PengFung}.
	
\end{itemize}

\section{Pengujian Eksperimental}
Pada pengujian eksperimental, perangkat lunak akan diuji dengan 5 buah kasus uji dokumen skripsi yang diambil dari \textit{Github} FTIS Unpar \footnote{https://github.com/ftisunpar/Skripsi}. Dokumen yang digunakan sebagai pengujian, yaitu:

\begin{enumerate}
	\item Skripsi yang ditulis oleh Cornelius David Herianto, dengan judul Pengelompokan Dokumen Berbasis Algoritma Genetika. Dokumen ini diberi nama TC\_PE\_01.pdf~\cite{pe01}
	
	\item Skripsi yang ditulis oleh Stillmen Vallian, dengan judul Kustomisasi Sharif Judge untuk Kebutuhan Program Studi Teknik Informatika. Dokumen ini diberi nama TC\_PE\_02.pdf~\cite{pe02}
	
	\item Skripsi yang ditulis oleh Nancy Valentina, dengan judul Aplikasi Pratinjau 3 Dimensi Berbasis Web. Dokumen ini diberi nama TC\_PE\_03.pdf~\cite{pe03}
	
	\item Skripsi yang ditulis oleh Evelyn Wijaya, dengan judul Pembangunan Gim Snake 360 Berbasis Web dengan Kode Terbuka. Dokumen ini diberi nama TC\_PE\_04.pdf~\cite{pe04}
	
	\item Skripsi yang ditulis oleh Ellena Angelica, dengan judul Kolektor Pengumuman Informatika. Dokumen ini diberi nama TC\_PE\_05.pdf~\cite{pe05} 
	
\end{enumerate}

Pada pengujian ini, ke-5 kasus uji tersebut akan dijalankan pada perangkat lunak. Berikut adalah hasil yang dikeluarkan oleh perangkat lunak dari setiap kasus uji yang digunakan:

\begin{enumerate}
	\item TC\_PE\_01.pdf
	
\begin{lstlisting}[caption={Pesan Kesalahan Secured PDF}	\label{lst:secured-pe01},language=php,xleftmargin=.0\textwidth]
Fatal error: Uncaught Exception: Object list not found. Possible secured file. 
in D:\Skripsi\src\vendor\smalot\pdfparser\src\Smalot\PdfParser\Parser.php:105
Stack trace:
#0 D:\Skripsi\src\vendor\smalot\pdfparser\src\Smalot\PdfParser\Parser.php(81): 
Smalot\PdfParser\Parser->parseContent('%PDF-1.5\n%\xD0\xD4\xC5\xD8\n...')
#1 D:\Skripsi\src\pdf_checker\SkripsiExtract.php(26): Smalot\PdfParser\Parser->
parseFile('D:\\Skripsi\\src\\...')
#2 D:\Skripsi\src\pdf_checker\Main.php(9): SkripsiExtract->extractFromPDF('../
res/TC_PE_01...')
#3 {main}
thrown in D:\Skripsi\src\vendor\smalot\pdfparser\src\Smalot\PdfParser\Parser.
php on line 105
\end{lstlisting}

	Pada kasus uji ini, perangkat lunak mendeteksi bahwa file ini merupakan file yang \textit{secured}. Sebagaimana telah dijelaskan pada sub bab \ref{sec:pdfparser}, bahwa \textit{library} ini tidak dapat melakukan ekstrak dokumen yang \textit{secured}. Namun masalah ini dapat diatasi dan akan dibahas lebih lanjut pada sub sub bab \ref{secured}. Berikut ini adalah laporan kesalahan yang ditemukan:

\begin{lstlisting}[caption={Laporan Kesalahan Kasus Uji TC\_PE\_01}	\label{lst:pe01},language=php,xleftmargin=.0\textwidth]

ERROR 1
==================
Error Code: PS-01
Note: Ditemukan penulisan kata yang tidak sesuai dengan kamus
Excerpt: 

PENDAHULUAN, BelakangPengelompokan, clustering, merupakan, mencari, 
kumpulan, melibatkan, pemilihan, cluster, dibandingkan, berada, 
Clustering, berguna, mereduksi, karakteristik, tertentu, mengembangkan, 
klasi, dikenal, sebagai, memberikan, masukkan, dukungan, terhadap, 
mengenai, pembelajaran, terarah, unsupervised, learning, Pembagian, 
berdasarkan, diketahui, sebelumnya, melainkan, kesamaan, menurut, ukuran, 
Document, pengelompokan, lakukan, Pengelompokan, diterapkan, penambangan, 
web, pencari, search, engine, information, retrieval, dilakukan, adalah, 
mengukur, kemiripan, similarity, mengelompokan, serupa, terdiri, tulisan, 
algoritma, digunakan, K, means, membagi, tersebut, dibentuk, meminimalkan, 
centroid, setiap, dicari, menggunakan, mean, dimodelkan, n, banyaknya, 
terbukti, melakukan, apapun, memiliki, kekurangan, terjebak, local, 
optima, tergantung, centroidawal, ditangani, Genetic, Algorithm, GA, 
menyelesaikan, pencarian, optimasi, heuristik, menirukan, secara, terjadi, 
survival, of, the, ttest, News, disediakan, menjadi, Karakteristik, 
datasetini, Terdiri, berasal, website, Newsdari, ditulis, Terbagi, 
business, entertainment, politics, dantech, sportterdapat, techterdapat, 
plain, textyang, le, TXT, diproses, terlebih, membersihkan, mengambil, 
berupa, karakter, kapitalisasi, diabaikan, diubah, tech, dilampirkan, 
Lampiran, menemukan, direpresentasikan, diambil, informasinya, diolah, 
merepresentasikan, PERANCANGANPada, perancangan, rancangan, pengguna, 
Kebutuhan, kebutuhan, masukan, diakomodasi, Graphical, User, Interface, 
Rincian, melalui, berformat, CSV, dibuka, pengolah, spreadsheet, 
Microsoft, Excel, File, laporan, dibutuhkan, keluaran, leCSV, dilihat, 
penamaan, YYYY, MM, DD, hh, mm, ss, csv, Januari, KMeans, PENGUJIAN, 
Pengujian, spesi, berikut, processor, R, Core, TM, i, HQ, CPU, GHz, MB, 
Windows, Build, diteliti, antaranya, terikat, tujuan, mengetahui, 
hubungan, Banyaknya, individu, pembobotan, penghitungan, TF, IDF, 
terjadinya, pembentukan, keturunan, Individu, elitisme, tnessterbaik, 
selanjutnya, selama, terakhir, nya, KESIMPULAN, Kesimpulan, Berdasarkan, 
penge, lompokan, diperlukan, sebelum, mengelompokkan, tasi, 
Merepresentasikan, tersusun, Kcentroid, tnessyang, intracluster, berhasil, 
dibuat, purity, sebesar, mengatasi, membutuhkan, disebabkan, komputasi, 
alasan, berjalan, Mulai, kedua, menyusun, bersifat, sparse, bernilai, 
Centroid, memperlambat, perhitungan, cosine, seharusnya, mengabaikan, 
berbobot, merepre, sentasikan, berbanding, kemungkinannya, semakin, 
sejauh, Kemungkinan, jaraknya

ERROR 2
==================
Error Code: PS-03
Note: Perhatikan spasi setelah tanda baca.
Excerpt: Format penamaan dari file ini adalah "algoritma-YYYY.MM.DD hh_mm_ss.
csv"

ERROR 3
==================
Error Code: KAL-02
Note: Berilah kata pengantar untuk setiap bab
Excerpt: BAB 1 PENDAHULUAN 1.1Latar BelakangPengelompokan ( clustering ) 
merupakan prosedur untuk mencari struktur alami dari suatu kumpulan data

ERROR 4
==================
Error Code: KAL-02
Note: Berilah kata pengantar untuk setiap bab
Excerpt: BAB 5 PENGUJIAN DAN EKSPERIMEN 5.1Skenario Pengujian 
Eksperimental Eksperimen dilakukan dengan menggunakan spesifikasi komputer 
sebagai berikut: 1.Tipe processor : Intel(R) Core(TM) i7-4720HQ CPU 
@2.60GHz 2.Memori: 12288MB RAM 3.Sistem operasi: Windows 10 Pro 64-bit 
(10.0, Build 17763)Pada suatu penelitian, ada beberapa jenis variabel yang 
diteliti di antaranya variabel bebas, variabel terikat, dan variabel 
kontrol

ERROR 5
==================
Error Code: KAL-02
Note: Berilah kata pengantar untuk setiap bab
Excerpt: BAB 6 KESIMPULAN DAN SARAN 6.1Kesimpulan Kesimpulan yang dapat diambil 
dari penelitian ini adalah sebagai berikut: 1.Berdasarkan dataset yang telah 
digunakan, algoritma genetika dapat digunakan dalam penge- lompokan dokumen

ERROR 6
==================
Error Code: NAT-01
Note: Referensi tidak dirujuk dengan baik, lakukan perintah PDFLatex->BibTex->
PDFLatex->PDFLatex untuk memperbaikinya
Excerpt: Pembagian kelompok dalam clustering tidak berdasarkan sesuatu yang 
telah diketahui sebelumnya, melainkan berdasarkan kesamaan tertentu menurut 
suatu ukuran tertentu [ ?]

ERROR 7
==================
Error Code: NAT-01
Note: Referensi tidak dirujuk dengan baik, lakukan perintah PDFLatex->BibTex->
PDFLatex->PDFLatex untuk memperbaikinya
Excerpt: Pengelompokan dokumen diterapkan dalam beberapa bidang seperti 
penambangan web, mesin pencari ( search engine ), dan temu kembali informasi 
(information retrieval ) [ ?]

ERROR 8
==================
Error Code: NAT-01
Note: Referensi tidak dirujuk dengan baik, lakukan perintah PDFLatex->BibTex->
PDFLatex->PDFLatex untuk memperbaikinya
Excerpt: GA merupakan teknik pencarian heuristik tingkat tinggi yang menirukan 
proses evolusi yang secara alami terjadi [ ?] berdasarkan prinsip survival of
the fittest 

ERROR 9
==================
Error Code: NAT-01
Note: Referensi tidak dirujuk dengan baik, lakukan perintah PDFLatex->BibTex->
PDFLatex->PDFLatex untuk memperbaikinya
Excerpt: Algoritma ini dinamakan demikian karena menggunakan konsep-konsep 
dalam genetika sebagai model pemecahan masalahnya [ ?]

ERROR 10
==================
Error Code: NAT-01
Note: Referensi tidak dirujuk dengan baik, lakukan perintah PDFLatex->BibTex->
PDFLatex->PDFLatex untuk memperbaikinya
Excerpt: 2.1Pengelompokan 2.1.1Definisi Pengelompokan Pengelompokan ( 
clustering ) merupakan sebuah metode untuk menggabungkan himpunan objek ke 
dalam kelompok-kelompok sedemikian rupa sehingga objek dalam satu kelompok 
( cluster ) lebih mirip (karena suatu hal) satu sama lain daripada objek di 
kelompok lain [ ?]
\end{lstlisting}
	
	\item TC\_PE\_02.pdf \newline
	Perangkat lunak tidak mengeluarkan laporan kesalahan yang ditemukan pada dokumen. Pada terminal juga tidak diberikan pesan error yang menandakan bahwa terdapat kesalahan pada perangkat lunak. Namun masalah ini dapat diatasi dan akan dibahas lebih lanjut pada sub sub bab \ref{kosong}. Berikut ini adalah laporan kesalahan yang ditemukan:

\begin{lstlisting}[caption={Laporan Kesalahan Kasus Uji TC\_PE\_02}	\label{lst:pe02},language=php,xleftmargin=.0\textwidth]
ERROR 1
==================
Error Code: PS-01
Note: Ditemukan penulisan kata yang tidak sesuai dengan kamus
Excerpt: 

PENDAHULUAN, BelakangPengumuman, jurusan, UNPAR, umumnya, dilakukan, 
Pengumuman, pengumuman, dijamin, dituju, setelah, dikirim, layanan, 
memiliki, terorganisir, Berbagai, tercampur, menyulitkan, pemilik, 
mencari, mengakibatkan, terbaca, secara, Penelitian, membuat, kekurangan, 
tersebut, BlueTape, adalah, berfungsi, membantu, urusan, paper, based, 
FTIS, menjadi, paperless, Fitur, mengumpulkan, berisi, menampilkannya, 
penerima, memanfaatkan, LINE, Line, Corporation, memungkinkan, disebut, 
mengirim, pengikut, bersamaan, Penerima, diminta, mengikuti, menerima, 
noti, diakses, melalui, internet, Heroku, cloud, ketiga, penggunanya, 
membangun, menjalankan, mengoperasikan, Rumusan, memodi, berjalan, 
mengimplementasikan, menguji, Tujuan, Mempelajari, Memodi, Menguji, 
LANDASAN, landasan, dipakai, menjalan, mendukung, pemrograman, meliputi, 
Ruby, Node, js, Java, Python, Clojure, Scala, Go, PHP, menyebarkan, 
mengelola, ditulis, didukung, mende, nisikan, sebagai, gabungan, source, 
code, dependencyyang, leProc, le, Dependency, dibangun, dijalankan, 
membutuhkan, dependency, Aturan, peletakkan, berbeda, menuliskan, 
lepackage, json, Dyno, berbasis, Unix, web, dyno, disebar, dikelompokkan, 
Web, process, type, satunya, HTTP, routerHeroku, Worker, PERANCANGANBab, 
membahas, perancangan, Pembahasan, dibagi, bagian, Perancangan, Class, 
merupakan, pembangunan, Penjelasan, diberikan, control, view, PENGUJIAN, 
pengujian, ImplementasiBagian, terlebih, deploy, dalamnya, diberi, 
Shadowtape, shadowtape, herokuapp, Setelah, dimodi, peneliti, Gmail, 
shadowbluetape, gmail, bernama, Lingkungan, Pengembangan, Berikut, spesi, 
Spesi, Processor, R, Celeron, CPU, U, GHz, x, Graphics, Ivybridge, Mobile, 
GB, Harddisk, SATA, Ubuntu, LTS, Code, version, apache, v, cli, Composer, 
pgAdmin, Application, Desktop, psql, PostgreSQL, heroku, linux, node, 
KESIMPULAN, Kesimpulan, kesimpulan, diambil, penelitian, comsetiap, 
diidenti, ditampilkan, memunculkan, teruji, Mengadaptasi, kon, gurasi, O, 
cial, Account, Meneliti, loginyang, ditemukan, Meningkatkan, usabilitas, 
menggunakannya

ERROR 2
==================
Error Code: PS-03
Note: Perhatikan spasi setelah tanda baca.
Excerpt: Heroku mendukung beberapa bahasa pemrograman, meliputi: Ruby, Node.js, 
Java, Python, Clojure, Scala, Go, dan PHP

ERROR 
==================
Error Code: PS-03
Note: Perhatikan spasi setelah tanda baca.
Excerpt: Contoh: Aplikasi dengan bahasa Node.js menuliskan deskripsi dependency 
padafilepackage.json 

ERROR 4
==================
Error Code: PS-03
Note: Perhatikan spasi setelah tanda baca.
Excerpt: azurewebsites.net/ (Gambar3.1)

ERROR 5
==================
Error Code: PS-03
Note: Perhatikan spasi setelah tanda baca.
Excerpt: Kode program ini dapat diakses di https://github.com/ftisunpar/
BlueTape 

ERROR 6
==================
Error Code: PS-03
Note: Perhatikan spasi setelah tanda baca.
Excerpt: MVC (Model-View-Control ler ) adalah sebuah metode untuk membuat 
perangkat lunak menjadi tiga1 https://github.com/ftisunpar/BlueTape BAB 4 
PERANCANGANBab ini membahas perancangan fitur kolektor pengumuman

ERROR 7
==================
Error Code: PS-03
Note: Perhatikan spasi setelah tanda baca.
Excerpt: Aplikasi web tersebut diberi nama Shadowtape dan dapat diakses di 
https://shadowtape.herokuapp.com/ 

ERROR 8
==================
Error Code: PS-03
Note: Perhatikan spasi setelah tanda baca.
Excerpt: Akun email baru menggunakan layanan email Gmail dengan alamat 
shadowbluetape@gmail.com

ERROR 9
==================
Error Code: PS-03
Note: Perhatikan spasi setelah tanda baca.
Excerpt: BlueTape versi skripsi ini diberi nama shadowtape dan dapat diakses di 
https://shadowtape.herokuapp.com/

ERROR 10
==================
Error Code: PS-03
Note: Perhatikan spasi setelah tanda baca.
Excerpt: Fitur ini dapat melakukan sinkronisasi kotak masuk untuk alamat 
email shadowbluetape@gmail.comsetiap jam

ERROR 11
==================
Error Code: PS-03
Note: Perhatikan spasi setelah tanda baca.
Excerpt: Email yang diidentifikasi sebagai pengumuman dapat ditampilkan di 
https://shadowtape.herokuapp

ERROR 12
==================
Error Code: PS-09
Note: Bab/Subbab 2.1 ini hanya terdapat 1 sub bab/sub sub bab

ERROR 13
==================
Error Code: PS-09
Note: Bab/Subbab 3.1 ini hanya terdapat 1 sub bab/sub sub bab

ERROR 14
==================
Error Code: PS-09
Note: Bab/Subbab 3.1.1 ini hanya terdapat 1 sub bab/sub sub bab

ERROR 15
==================
Error Code: KAL-02
Note: Berilah kata pengantar untuk setiap bab
Excerpt: BAB 1 PENDAHULUAN 1.1Latar BelakangPengumuman di jurusan Teknik 
Informatika UNPAR pada umumnya dilakukan lewat email 
\end{lstlisting}
	
	\item TC\_PE\_03.pdf \newline
	Perangkat lunak tidak mengeluarkan laporan kesalahan yang ditemukan pada dokumen. Pada terminal juga tidak diberikan pesan error yang menandakan bahwa terdapat kesalahan pada perangkat lunak. Namun masalah ini dapat diatasi dan akan dibahas lebih lanjut pada sub sub bab \ref{kosong}. Berikut ini adalah laporan kesalahan yang ditemukan:

\begin{lstlisting}[caption={Laporan Kesalahan Kasus Uji TC\_PE\_03}	\label{lst:pe03},language=php,xleftmargin=.0\textwidth]
ERROR 1
==================
Error Code: PS-01
Note: Ditemukan penulisan kata yang tidak sesuai dengan kamus
Excerpt: 

PENDAHULUAN, BelakangSnake, merupakan, sebuah, permainan, dibuat, 
Peter, Trefonas, Snake, berasal, Blockade, Awalnya, dimainkan, Nokia, 
bermain, adalah, pemain, menggerakan, tersebut, mendapatkan, makanan, 
sebanyak, banyaknya, menabrak, Setiap, memakan, memanjang, semakin, meng, 
gerakan, menutupi, HTML, Hyper, Text, Markup, Language, digunakan, 
membuat, web, penerus, XHTML, memiliki, satunya, Canvas, menggambar, 
pixel, ditulis, gunakan, pemrograman, JavaScript, Javascript, menjadi, 
jQuery, library, traversal, penanganan, event, Ajax, de, ngan, 
Application, Programming, Interface, GitHub, layanan, hosting, bersama, 
pengembangan, menggunakan, version, control, Git, adanya, programmer, 
etahui, perubahan, termakan, ditempatkan, secara, Permainan, berakhir, 
tubuhnya,  jumlahnya, warnanya, bermacam, menambah, Silther, 
PERANCANGANPada, mengenai, perancangan, dibangun, Perancangan, dilakukan, 
meliputi, sequence, state, tampilan, Rancangan, Sequence, ditunjukan, 
dijelaskan, Source, sequenceyang, memuat, sequenceuntuk, memilih, Berikut, 
penjelasan, menerima, input, mencari, le, Filelabirin, ditemukan, 
memanggil, method, Maze, lelabirin, Setelah, DrawingObject, digambar, 
PENGUJIANPada, pengujian, linkini, generaldevilx, github, lingkungan, 
Lingkungan, pembangunan, Processor, Core, i, U, GHz, GB, Card, GeForce, 
MX, Storage, TB, mobile, berbasis, Android, desktop, SM, J, G, Exynos, 
Octa, MHz, Cortex, KESIMPULAN, SARANPada, berisi, kesimpulan, Kesimpulan, 
diambil, diantaranya, Pembangunan, berhasil, berbelok, smartphone, 
diharapkan, menekan, Play, Menambahkan, berdasarkan, penguji, buatan, 
Menyimpan, disimpan, dimuat, Berdasarkan, dipaparkan, mengurangi, 
kesalahan, pembuatan, terutama, memposisikan, Seharusnya, melainkan, 
dituliskan, Sebelum, sebaiknya, preview, terlebih, melihat, dipilih

ERROR 2
==================
Error Code: PS-03
Note: Perhatikan spasi setelah tanda baca.
Excerpt: 1.2Rumusan Masalah Rumusan dari masalah yang akan dibahas pada skripsi 
ini adalah sebagai berikut: Bagaimana membangun permainan Snakemenggunakan 
HTML5 Canvas? Bagaimana cara menyimpan labirin pada fileeksternal? 1 
https://en.wikipedia.org/wiki/Snake_(video_ game_ genre) BAB 2 LANDASAN 
TEORI 
2.1 Snake GameSnake Game merupakan permainan mengendalikan ular untuk 
mendapatkan makanan yang terdapat pada labirin

ERROR 3
==================
Error Code: PS-03
Note: Perhatikan spasi setelah tanda baca.
Excerpt: Contoh singleplayer game Snake adalah Snake pada telepon genggam Nokia 
yang dapat dilihat pada Gambar2.1 1dan contoh multiplayer game Snake adalah 
Slither.io yang dapat dilihat Gambar2.2 2 

ERROR 4
==================
Error Code: PS-03
Note: Perhatikan spasi setelah tanda baca.
Excerpt: Gambar 2.1: Permainan Snake pada telepon genggam NokiaGambar 2.2: 
Permainan Slither.iopadaAndroid 1 https://en.wikipedia.org/wiki/
Snake_(video_game_genre) 2 https://play.google.com/store/apps/details?
id=air.com.hypah.io.slither BAB 3 ANALISIS 3.1Analisis Permainan Snakeyang 
Sudah AdaPermainan Snake yang akan dianalisis adalah Slither.io dan Snake 
pada telepon genggam Nokia 

ERROR 5
==================
Error Code: PS-03
Note: Perhatikan spasi setelah tanda baca.
Excerpt: Slither.io adalah permainan web yang dapat dimainkan oleh lebih dari 1 
pemain( multiplayer )

ERROR 6
==================
Error Code: PS-03
Note: Perhatikan spasi setelah tanda baca.
Excerpt: 3.1.1Ular dan Makanan Ular pada Slither.io dibentuk dengan menggunakan 
sekumpulan lingkaran yang saling berdempetan satu sama lain seperti pada 
Gambar3.1

ERROR 7
==================
Error Code: PS-03
Note: Perhatikan spasi setelah tanda baca.
Excerpt: Makanan pada Slither.io berbentuk lingkaran

ERROR 8
==================
Error Code: PS-03
Note: Perhatikan spasi setelah tanda baca.
Excerpt: Gambar 3.1: Ular pada Silther.io BAB 4 PERANCANGANPada bab ini akan 
dibahas mengenai perancangan permainan yang dibangun

ERROR 9
==================
Error Code: PS-03
Note: Perhatikan spasi setelah tanda baca.
Excerpt: Permainan ini dapat dimainkan pada linkini : https://generaldevilx.
github.io/Snake360/ 5.1Implementasi Pada bagian ini akan dijelaskan mengenai 
lingkungan yang digunakan untuk membangun dan implementasi antarmuka dari Open 
Source Snake 360

ERROR 10
==================
Error Code: KAL-02
Note: Berilah kata pengantar untuk setiap bab
Excerpt: BAB 1 PENDAHULUAN 1.1Latar BelakangSnake merupakan sebuah permainan 
yang pertama kali dibuat oleh Peter Trefonas pada tahun 1978
\end{lstlisting}	
	
	\item TC\_PE\_04.pdf

\begin{lstlisting}[caption={Pesan Kesalahan Secured PDF}	\label{lst:secured-pe04},language=php,xleftmargin=.0\textwidth]
Fatal error: Uncaught Exception: Object list not found. Possible secured file. 
in D:\Skripsi\src\vendor\smalot\pdfparser\src\Smalot\PdfParser\Parser.php:105
Stack trace:
#0 D:\Skripsi\src\vendor\smalot\pdfparser\src\Smalot\PdfParser\Parser.php(81): 
Smalot\PdfParser\Parser->parseContent('%PDF-1.5\n%\xD0\xD4\xC5\xD8\n...')
#1 D:\Skripsi\src\pdf_checker\SkripsiExtract.php(26): Smalot\PdfParser\Parser->
parseFile('D:\\Skripsi\\src\\...')
#2 D:\Skripsi\src\pdf_checker\Main.php(9): SkripsiExtract->extractFromPDF('../
res/TC_PE_04...')
#3 {main}
thrown in D:\Skripsi\src\vendor\smalot\pdfparser\src\Smalot\PdfParser\Parser.
php on line 105
\end{lstlisting}

	Pada kasus uji ini, perangkat lunak mendeteksi bahwa file ini merupakan file yang \textit{secured}. Hal ini sama dengan yang terjadi pada kasus uji TC\_PE\_01.pdf. Namun masalah ini dapat diatasi dan akan dibahas lebih lanjut pada sub sub bab \ref{secured}. Berikut ini adalah laporan kesalahan yang ditemukan:

\begin{lstlisting}[caption={Laporan Kesalahan Kasus Uji TC\_PE\_04}	\label{lst:pe04},language=php,xleftmargin=.0\textwidth]
ERROR 1
==================
Error Code: PS-01
Note: Ditemukan penulisan kata yang tidak sesuai dengan kamus
Excerpt:

PENDAHULUAN, BelakangAplikasi, merupakan, sebuah, membantu, penggunanya, 
meninjau, dihasilkan, secara, sebelum, pengguna, tersebut, melakukan, 
pembuatan, Kelebihan, adalah, peninjauan, berbagai, memaksimalkan, 
memungkinkan, merubah, bertujuan, memutuskan, dasarnya, terhindar, 
ekspektasi, Penggunaan, web, memudahkan, mela, kukan, menggunakan, 
berbasis, lingkungan, Windows, Linux, Mac, OS, membatasi, cakupan, dibuat, 
kustomisasi, mengajar, perkuliahan, Melalui, diharapkan, memiliki, 
gambaran, mengenai, ruangan, komposisi, memanfaatkan, WebGL, Three, js, 
Application, Programming, Interface, gra, s, berdasar, OpenGL, terbuka, 
ECMAScript, melalui, canvas, HTML, membuat, digunakan, sebagai, Ruangan, 
dilengkapi, peralatan, menunjang, pengajaran, internet, learning, 
menjamin, kenyamanan, selama, pendingin, Rumusan, Berikut, dibahas, 
pendukung, lainnya, direpsentasikan, tampilan, OrbitControls, berkaitan, 
index, html, HyperText, Markup, Language, mem, json, JSON, constant, 
diinisialisasi, imported, repsentasi, ujian, berlangsung, img, jpg, 
pilihan, textureatap, PENGUJIANBab, membahas, pengujian, rancangan, 
Lingkungan, Pengujian, Sedangkan, spesi, Processor, GHz, Core, Graphics, 
Processing, GPU, HD, MB, Random, Access, Memory, GB, MHz, DDR, Storage, 
SSD, macOS, Sierra, Pemrograman, CSS, Text, Code, MAMP, version, Google, 
Chrome, Ver, sion, O, cial, Build, sebelumnya, Pengimplementasian, pemro, 
graman, dijabarkan, KESIMPULAN, SARANBab, kesimpulan, pengembangan, 
Kesimpulan, Berdasarkan, Prantijau, diperoleh, tujuan, merepresentasikan, 
Parahyangan, Fitur, dikembangkan, Mengganti, Unggah, Object, Notation, 
mengubah, Menghasilkan, cetakan, dikarenakan, keterbatasan, ngembangan, 
Mengimplementasi, penyembunyian, menutupi, penyediaan, kemudahan, 
mengunggah

ERROR 2
==================
Error Code: PS-03
Note: Perhatikan spasi setelah tanda baca.
Excerpt: Perangkat lunak akan dibuat dengan memanfaatkan WebGL dan pustaka 
Three.js

ERROR 3
==================
Error Code: PS-03
Note: Perhatikan spasi setelah tanda baca.
Excerpt: Sementara itu pustaka Three.js bertujuan membuat pustaka 3 dimensi 
yang mudah dan ringan untuk digunakan

ERROR 4
==================
Error Code: PS-03
Note: Perhatikan spasi setelah tanda baca.
Excerpt: 1.2Rumusan Masalah Berikut ini masalah-masalah yang dibahas dalam 
skripsi ini: Bagaimana ruangan kelas dan objek pendukung lainnya dapat 
direpsentasikan dalam WebGL? Bagaimana membuat tampilan responsif pada 
aplikasi agar terlihat bagus saat dicetak? BAB 2 LANDASAN TEORIBab ini 
berisi penjelasan mengenai teori-teori yang menjadi dasar penelitian ini, 
seperti WebGL dan Three.js library

ERROR 5
==================
Error Code: PS-03
Note: Perhatikan spasi setelah tanda baca.
Excerpt: BAB 3 ANALISISBab ini berisi analisis pemodelan properti kelas, 
analisis pemanfaatan pustaka Three.js, dan analisis penggunaan WebGL

ERROR 6
==================
Error Code: PS-03
Note: Perhatikan spasi setelah tanda baca.
Excerpt: 3.2Analisis pemanfaatan pustaka Three.js Pada pengimplementasian 
Aplikasi Pratinjau 3 Dimensi Berbasis Web digunakan banyak fitur yang telah 
disediakan oleh pustaka Three.js

ERROR 7
==================
Error Code: PS-03
Note: Perhatikan spasi setelah tanda baca.
Excerpt: Berikut ini merupakan daftar fitur pustaka Three.js yang digunakan 
pada implementasi Aplikasi Pratinjau 3 Dimensi Berbasis Web: 3.2.1Panggung 
Panggung ( scene ) merupakan sebuah wadah untuk menempatkan sesuatu pada 
pustaka Three.js (subbab2.2.1)

ERROR 8
==================
Error Code: PS-03
Note: Perhatikan spasi setelah tanda baca.
Excerpt: 1 https://www.blender.org/ BAB 4 PERANCANGANPada bab ini dibahas 
mengenai perancangan perangkat lunak yang meliputi: perancangan struk- tur 
web, perancangan antarmuka, dan perancangan fitur yang diimplementasikan 
pada Aplikasi Pratinjau 3 Dimensi Berbasis

ERROR 9
==================
Error Code: PS-03
Note: Perhatikan spasi setelah tanda baca.
Excerpt: Berkas yang ada pada folder ini hanya satu yaitu custom.css

ERROR 10
==================
Error Code: PS-03
Note: Perhatikan spasi setelah tanda baca.
Excerpt: Terdapat berbagai file JavaScript di dalam folder ini, daftar 
berkasnya adalah sebagai berikut: Main.js , berkas JavaScript ini berisi 
berbagai fungsi utama yang digunakan untuk membangun Aplikasi Pratinjau 3 
Dimensi Berbasis Web

ERROR 11
==================
Error Code: PS-03
Note: Perhatikan spasi setelah tanda baca.
Excerpt: three.js , berkas JavaScript ini berisi berbagai fungsi yang 
disediakan oleh pustaka Three.js

ERROR 12
==================
Error Code: PS-03
Note: Perhatikan spasi setelah tanda baca.
Excerpt: Nantinya fungsi yang terdapat di dalam berkas ini akan dipanggil oleh 
Main.js OrbitControls.js , berkas JavaScript ini berisi berbagai fungsi yang 
juga disediakan oleh pustaka Three.js

ERROR 13
==================
Error Code: PS-03
Note: Perhatikan spasi setelah tanda baca.
Excerpt: berkas index.html , merupakan berkas HyperText Markup Language (HTML) 
yang mem- bentuk web untuk aplikasi pratinjau ini

ERROR 14
==================
Error Code: PS-03
Note: Perhatikan spasi setelah tanda baca.
Excerpt: Terdapat 2 berkas json di dalam folder ini yaitu sebagai berikut: 
constant.json , berkas ini berisi berbagai informasi awal untuk 
diinisialisasi ke aplikasi sehingga dapat menampilkan gambaran awal dari 
ruangan kelas Fakultas Teknologi Informasi dan Sains

ERROR 15
==================
Error Code: PS-03
Note: Perhatikan spasi setelah tanda baca.
Excerpt: imported.json , berkas ini berisi informasi untuk repsentasi ruangan 
kelas saat ujian sedang berlangsung di Fakultas Teknologi Informasi dan Sains

ERROR 16
==================
Error Code: PS-03
Note: Perhatikan spasi setelah tanda baca.
Excerpt: Berikut ini merupakan daftar berkas yang ada pada folder ini: 
textureatap.jpg, merupakan tekstur untuk bagian atap ruangan kelas

ERROR 17
==================
Error Code: PS-05
Note: Huruf pertama pada kalimat ini tidak menggunakan huruf kapital
Excerpt: t y p e d e f u n s i g n e d l o n g GLenum ; t y p e d e f b o 
o l e a n G L b o o l e a n ; t y p e d e f u n s i g n e d l o n g G L b 
i t f i e l d ; t y p e d e f b y t e GLbyte ; t y p e d e f s h o r t G L 
s h o r t ; t y p e d e f l o n g G L i n t ; t y p e d e f l o n g G L s 
i z e i ; t y p e d e f l o n g l o n g G L i n t p t r ; t y p e d e f l 
o n g l o n g G L s i z e i p t r ; t y p e d e f o c t e t GLubyte ; t y 
p e d e f u n s i g n e d s h o r t G L u s h o r t ; t y p e d e f u n s 
i g n e d l o n g G L u i n t ; t y p e d e f u n r e s t r i c t e d f l 
o a t G L f l o a t ; t y p e d e f u n r e s t r i c t e d f l o a t G L 
c l a m p f ; Listing 2.1: Alias untuk tipe pada WebGL

ERROR 18
==================
Error Code: PS-05
Note: Huruf pertama pada kalimat ini tidak menggunakan huruf kapital
Excerpt: Berkas yang ada pada folder ini hanya satu yaitu custom.css

ERROR 19
==================
Error Code: PS-09
Note: Bab/Subbab 4.2.1 ini hanya terdapat 1 sub bab/sub sub bab

ERROR 20
==================
Error Code: KAL-02
Note: Berilah kata pengantar untuk setiap bab
Excerpt: BAB 1 PENDAHULUAN 1.1Latar BelakangAplikasi pratinjau 3 dimensi 
merupakan sebuah perangkat lunak yang membantu penggunanya untuk meninjau 
kembali desain dari produk yang ingin dihasilkan secara 3 dimensi, sebelum 
pengguna tersebut melakukan implementasi pembuatan produk

ERROR 21
==================
Error Code: KAL-03
Note: Ada data skripsi yang belum dilengkapi pada halaman cover
Excerpt: SKRIPSI APLIKASI PRATINJAU 3 DIMENSI BERBASIS WEB Nancy Valentina 
NPM: 2014730049 PROGRAM STUDI TEKNIK INFORMATIKA FAKULTAS TEKNOLOGI 
INFORMASI DAN SAINS UNIVERSITAS KATOLIK PARAHYANGAN tahun UNDERGRADUATE 
THESIS JUDUL BAHASA INGGRIS Nancy Valentina NPM: 2014730049 DEPARTMENT OF 
INFORMATICS FACULTY OF INFORMATION TECHNOLOGY AND SCIENCES PARAHYANGAN 
CATHOLIC UNIVERSITY tahun

ERROR 22
==================
Error Code: NAT-01
Note: Referensi tidak dirujuk dengan baik, lakukan perintah PDFLatex->BibTex->
PDFLatex->PDFLatex untuk memperbaikinya
Excerpt: Canvas pada HTML menyediakan suatu destinasi untuk pembangunan objek 
secara programatik pada halaman web dan memungkinkan menampilkan objek yang 
sedang dibangun menggunakan API pembangun objek yang berbeda [ ?]
\end{lstlisting}
	
	\item TC\_PE\_05.pdf \newline
	Perangkat lunak tidak mengeluarkan laporan kesalahan yang ditemukan pada dokumen. Pada terminal juga tidak diberikan pesan error yang menandakan bahwa terdapat kesalahan pada perangkat lunak. Namun masalah ini dapat diatasi dan akan dibahas lebih lanjut pada sub sub bab \ref{kosong}. Berikut ini adalah laporan kesalahan yang ditemukan:

\begin{lstlisting}[caption={Laporan Kesalahan Kasus Uji TC\_PE\_05}	\label{lst:pe05},language=php,xleftmargin=.0\textwidth]
ERROR 1
==================
Error Code: PS-01
Note: Ditemukan penulisan kata yang tidak sesuai dengan kamus
Excerpt: 

PENDAHULUAN, BelakangOnline, judge, merupakan, sebuah, online, dirancang, 
mengevaluasi, algoritma, dikumpulkan, pengguna, mengcompile, mengeksekusi, 
mengujinya, lingkungan, menggunakan, disediakan, dijalankan, batasan, 
time, memory, keamanan, sebagainya, Penggunaan, ditemukan, pemrograman, 
Sharif, Judge, adalah, C, Java, Python, diciptakan, Mohammad, Javad, 
Naderi, bersifat, source, ditulis, PHP, framework, CodeIgniter, 
danbackend, BASH, menilai, jawaban, peserta, secara, penilaian, dimulai, 
membuat, assignment, dibutuhkan, mulai, berhenti, tambahan, mengunduh, 
mengerjakan, tersebut, Peserta, mengumpulkan, menjalankan, membandingkan, 
Setelah, menyesuaikan, keluaran, digunakan, Parahyangan, Algoritma, 
membantu, memanfaatkan, memberikan, ujian, melihat, setelah, memperbaiki, 
dikerjakan, melewati, pengumpulan, prakteknya, terkini, pengembangan, 
Pengembangan, memiliki, kebutuhan, spesi, k, Kebutuhan, login, 
terintegrasi, server, membatasi, pengaksesan, lainnya, terakhir, dicommit, 
GitHub, Juli, bug, diperbaiki, menyebabkan, memenuhi, dikemnbangkan, 
sebutkan, disebutkan,  Flow, Chart, menunjukan, mengalir, ANALISISBab, 
ditawarkan, didapat, issue, repository, dicatat, Google, Sheets,
didiskusikan, pembimbing, dilakukan, menganalisis, setiap, terbuka, 
terdapat, diikuti, urutan, dibuat, analisa, didapatkan, pertanyaan, 
usulan, Bebera, dijadikan, pertimbangan, melalui, diwawancarai, Husnul, 
Claudio, Franciscus, Vania, Natali, Luciana, Abednego, Joanna, Helga, 
PERANCANGANBab, perancangan, diimplementasi, Mengganti, Method, shel, l, 
exec, rm, Menjadiunlink, menghapus, le, control, Assignment, diubah, 
method, unlink, Listing, perubahan, syntax, Perubahan, 
classAssignmentsextendsCI, Controller, Upload, Tests, zip, file, shell, f, 
assignments, root, config, array, upload, path, allowed, types, else, 
foreach, old, pdf, files, name, this, messages, type, success, text, PDF, 
uploaded, successfully, Chrome, Version, O, cial, Build, Firefox, Quantum, 
Microsoft, Excel, Kedua, KESIMPULAN, SARANBab, kesimpulan, selanjutnya, 
Kesimpulan, Berdasarkan, diperoleh, sebagai, berikut, dikembangkan, 
mengimplementasi, dilihat, padaTabel, Membatasi, Mensupport, ledengan, 
TXT, Display, Name, pendaftaran, Lock, Student, Archived, Logs, Fame, 
loginkeserver, Branding, Mengubah, PHPExcel, menjadilibrary, 
PHPSpreadsheet, menjadiunlink, methodrekoneksi, berhasil, ditentukan, 
Fitur, JAR, dikarenakan, keterbatasan, pembaruan, Mengimplementasi, multi, 
class, masukan, padasub, Round, Robin, menggunakanCactiuntuk

ERROR 2
==================
Error Code: PS-03
Note: Perhatikan spasi setelah tanda baca.
Excerpt: Contoh perbandingan URL biasa dengan Clean URL s: URL biasa: \example.
com\index.php?page=news ,Clean URLs :\ example.com ews 

ERROR 3
==================
Error Code: PS-03
Note: Perhatikan spasi setelah tanda baca.
Excerpt: 4.1Mengganti Method shel l_exec("rm ...")Menjadiunlink() Method shel 
l_exec("rm ...") yang memiliki fungsi untuk menghapus sebuah file terdapat 
pada kelas control ler Assignment.php 

ERROR 4
==================
Error Code: PS-03
Note: Perhatikan spasi setelah tanda baca.
Excerpt: Listing 4.1: Perubahan kode program pada Assignment.php@@-433,8+433,7
@@classAssignmentsextendsCI_Controller // Upload Tests (zip file) -
shell_exec('rm-f'.$assignments_root.'/*.zip'); 
+unlink($assignments_root.'/*.zip'); $config = array( 'upload_path' => 
$assignments_root, 'allowed_types' => 'zip', 
@@-482,7+481,7@@classAssignmentsextendsCI_Controller else { 
foreach($old_pdf_files as $old_name) -shell_exec("rm-f$old_name"); 
+unlink($old_name); $this->messages[] = array( 'type' => 'success', 'text' 
=> 'PDF file uploaded successfully.' 4.2Menambahkan Method Rekoneksi ke 
Database Method rekoneksi ke database ditambahkan pada kelas control ler 
Queueprocess.php .Method yang digunakan yaitu $this->db->reconnect() 

ERROR 5
==================
Error Code: PS-03
Note: Perhatikan spasi setelah tanda baca.
Excerpt: Listing4.2menunjukan perubahan kode program yang dilakukan 
diQueueprocess.php Listing 4.2: Perubahan kode program pada Queueprocess.
php1 GNU Operating System, "Comparing and Merging Files" terakhir diubah 6 
Mei 2017

ERROR 6
==================
Error Code: PS-03
Note: Perhatikan spasi setelah tanda baca.
Excerpt: https://www.gnu.org/software/diutils/manual/diutils.html#Detailed-
Unified BAB 5 IMPLEMENTASI DAN PENGUJIANBab ini membahas tentang 
implementasi dan pengujian perangkat lunak berdasarkan rancangan yang sudah 
dibuat

ERROR 7
==================
Error Code: PS-05
Note: Huruf pertama pada kalimat ini tidak menggunakan huruf kapital
Excerpt: Contoh perbandingan URL biasa dengan Clean URL s: URL biasa: \example.
com\index.php?page=news ,Clean URLs :\ example.com ews 

ERROR 8
==================
Error Code: PS-09
Note: Bab/Subbab 3.1 ini hanya terdapat 1 sub bab/sub sub bab

ERROR 9
==================
Error Code: KAL-02
Note: Berilah kata pengantar untuk setiap bab
Excerpt: BAB 1 PENDAHULUAN 1.1Latar BelakangOnline judge merupakan sebuah 
sistem online yang dirancang untuk mengevaluasi kode algoritma yang 
dikumpulkan oleh pengguna
\end{lstlisting}
	
\end{enumerate}

\subsection{Perangkat Lunak Tidak Mengeluarkan Laporan Kesalahan}
\label{kosong}
Hipotesis yang didapatkan dari laporan yang dikeluarkan saat perangkat lunak menggunakan kasus uji TC\_PF\_02.pdf, TC\_PF\_03.pdf dan TC\_PF\_05.pdf adalah ada kesalahan yang perlu ditelusuri pada \textit{library PdfParser}.

\begin{enumerate}
	\item Pertama melakukan pengecekan terhadap method extractFromPDF(\$input) yang ada pada kelas SkripsiExtract. Langkah yang dilakukan untuk mencari kesalahan tersebut adalah dengan menambahkan kode print start dan stop di dalam method tersebut pada awal dan akhir sebelum \textit{return value}. Kode print stop akan bergeser ke atas hingga bertemu dengan sumber masalah. 
	
\begin{lstlisting}[caption={Potongan kode pada \textit{method extractFromPDF}}	\label{lst:checkKode},language=php,xleftmargin=.0\textwidth]
$directory = getcwd();
$this->parser = new \Smalot\PdfParser\Parser();
$this->file = $directory . '/' . $input;
echo "start";
$pdf = $this->parser->parseFile($this->file);
echo "stop";
$pages  = $pdf->getPages();
\end{lstlisting}

Pada saat perangkat lunak dijalankan, hasil yang dikeluarkan hanya tulisan \textit{start} saja. Pada listing \ref{lst:checkKode}, sumber masalah ditemukan pada baris ke-5 (baris ke-26 pada \textit{source code}). Setelah sumber ditemukan, masalah yang ditemui berasal dari \textit{library PdfParser}.
	
	\item Langkah ke-2 yaitu menelusuri kelas Parser dari \textit{library PdfParser}. Langkah yang dilakukan sama dengan langkah yang pertama. Pada kelas tersebut sumber masalah berada pada \textit{method parseContent()}.
	
\begin{lstlisting}[caption={Potongan kode pada \textit{method parseContent}}	\label{lst:checkKode2},language=php,xleftmargin=.0\textwidth]
ob_start();
echo "start";
@$parser = new \TCPDF_PARSER(ltrim($content));
echo "stop";
list($xref, $data) = $parser->getParsedData();
unset($parser);
ob_end_clean();
\end{lstlisting}
	
	Pada listing \ref{lst:checkKode2}, sumber masalah berada pada baris ke-3 (baris ke-93 pada \textit{source code}). Baris tersebut merujuk pada kelas TCPDF\_Parser.	
	
	\item Langkah ke-3 yaitu menelusuri kelas TCPDF\_Parser. Pada kelas ini terdapat beberapa \textit{method}, sehingga \textit{method} yang pertama kali dicek adalah konstruktornya.
	
\begin{lstlisting}[caption={Potongan kode pada konstruktor kelas TCPDF\_Parser}	\label{lst:checkKode3},language=php,xleftmargin=.0\textwidth]
foreach ($this->xref['xref'] as $obj => $offset) {
	if (!isset($this->objects[$obj]) AND ($offset > 0)) {
		// decode objects with positive offset
		echo "start";
		$this->objects[$obj] = $this->getIndirectObject($obj, $offset, true);
		echo "stop";
	}
}
\end{lstlisting}

	Pada listing \ref{lst:checkKode3}, sumber masalah berada pada baris ke-5 (baris ke-123 pada \textit{source code}). Baris tersebut merujuk pada method \textit{getIndirectObject()} yang ada pada kelas yang sama.
	
\begin{lstlisting}[caption={Potongan kode pada \textit{method getIndirectObject()}}	\label{lst:checkKode4},language=php,xleftmargin=.0\textwidth]
if ($decoding AND ($element[0] == 'stream') AND (isset($objdata[($i - 1)][0])) AND ($objdata[($i - 1)][0] == '<<')) {
	echo "start";	
	$element[3] = $this->decodeStream($objdata[($i - 1)][1], $element[1]);
	echo "stop";
}
\end{lstlisting}

	Pada listing \ref{lst:checkKode4}, sumber masalah berada pada baris ke-4 (baris ke-702 pada \textit{source code}). Baris tersebut merujuk pada \textit{method decodeStream()} yang ada pada kelas yang sama.
	
\begin{lstlisting}[caption={Potongan kode pada \textit{method decodeStream()}}	\label{lst:checkKode5},language=php,xleftmargin=.0\textwidth]
try {
	echo "start";
	$stream = TCPDF_FILTERS::decodeFilter($filter, $stream);
	echo "stop";
} catch (Exception $e) {
	$emsg = $e->getMessage();
	if ((($emsg[0] == '~') AND !$this >cfg['ignore_missing_filter_decoders']) OR (($emsg[0] != '~') AND !$this->cfg['ignore_filter_decoding_errors'])) {
		$this->Error($e->getMessage());
	}
}
\end{lstlisting}
	
	Pada listing \ref{lst:checkKode5}, sumber masalah berada pada baris ke-3 (baris ke-781 pada \textit{source code}). Baris tersebut merujuk pada kelas TCPDF\_Filters.
	
	\item Langkah terakhir yaitu melakukan pengecekan pada kelas TCPDF\_Filters. Pada kelas ini, \textit{method} yang diperiksa adalah \textit{decodeFilter()}.
	
\begin{lstlisting}[caption={Potongan kode pada \textit{method decodeFilter()}}	\label{lst:checkKode6},language=php,xleftmargin=.0\textwidth]
case 'FlateDecode': {
	return self::decodeFilterFlateDecode($data);
	break;
}
\end{lstlisting}
	
	Pada listing \ref{lst:checkKode6}, sumber masalah ada pada baris ke-2 (baris ke-94 pada \textit{source code}). Baris tersebut merujuk pada method \textit{decodeFilterFlateCode()}.
	
\begin{lstlisting}[caption={Potongan kode pada \textit{method decodeFilterFlateCode()}}	\label{lst:checkKode7},language=php,xleftmargin=.0\textwidth]
// initialize string to return
echo "start";
$decoded = @gzuncompress($data);
echo "stop";
if ($decoded === false) {
	self::Error('decodeFilterFlateDecode: invalid code');
}
return $decoded;
\end{lstlisting}

	Pada listing \ref{lst:checkKode7}, sumber dari masalah utamanya berada pada baris ke-2 (baris ke-357 pada \textit{source code}). Dari hasil pengujian yang dilakukan, alasan bahwa perangkat lunak tidak mengeluarkan pesan kesalahan karena pada baris ke-2, terdapat karakter ''@'' sebelum method \textit{gzuncompress()}. Fungsi karakter tersebut adalah untuk membungkam seluruh pesan kesalahan, sehingga pengguna tidak mengetahui kesalahan yang ada. Untuk melihat pesan kesalahan, karakter ''@'' dihapuskan dan menambahkan sebuah kode untuk melaporkan kesalahan.
	
\begin{lstlisting}[caption={Potongan kode pada \textit{method decodeFilterFlateCode()}}	\label{lst:checkKode8},language=php,xleftmargin=.0\textwidth]
// initialize string to return
error_reporting(E_ALL);
$decoded = gzuncompress($data);
if ($decoded === false) {
	self::Error('decodeFilterFlateDecode: invalid code');
}
return $decoded;
\end{lstlisting}

	Dengan memakai kode pada baris ke-2 di listing \ref{lst:checkKode9}, perangkat lunak berhasil mengeluarkan laporan kesalahan.
	
\begin{lstlisting}[caption={Pesan kesalahan memory limit}	\label{lst:checkKode9},language=php,xleftmargin=.0\textwidth]
Fatal error: Allowed memory size of 134217728 bytes exhausted (tried to allocate .......(size tergantung file) bytes) in D:\Skripsi\src\vendor\tecnickcom\tcpdf\include\tcpdf_filters.php on line 357
\end{lstlisting}
	
	\item Untuk mengatasi masalah yang ada pada listing \ref{lst:checkKode9}, maka ada beberapa hal yang perlu ditambahkan pada perintah \ref{lst:command} untuk menjalankan perangkat lunak.

\begin{lstlisting}[caption={Perintah untuk mengatasi masalah \textit{memory limit}}	\label{lst:checkKode10},language=php,xleftmargin=.1\textwidth]
php -d memory_limit=(ukuran sesuai kebutuhan)M main.php ../res/nama_file.pdf
\end{lstlisting}

	Dengan menggunakan perintah pada \ref{lst:checkKode10}, perangkat lunak dapat mengeluarkan pesan kesalahan dalam dokumen skripsi. Proses untuk menyelesaikan masalah, bahwa perangkat lunak tidak mengeluarkan pesan pun telah berhasil diselesaikan.

\end{enumerate}

\subsection{Perangkat Lunak Tidak Dapat Memeriksa Dokumen Skripsi}
\label{secured}

Pada beberapa kasus uji, perangkat lunak mengeluarkan kesalahan yang mengatakan bahwa file yang digunakan termasuk \textit{secured document}. File yang terkait yaitu ''TC\_PE\_01.pdf'' dan ''TC\_PE\_04.pdf''. Hipotesis yang didapatkan dari laporan adalah kemungkinan bahwa dokumen yang diunggah ke \textit{Github} akan menjadi \textit{secured document}.

Berdasarkan hipotesis yang telah disampaikan, ada sebuah percobaan yang dilakukan pada kasus uji yang terkait. Hal yang dilakukan adalah mengunduh seluruh data skripsi dari file yang terkait. Setelah itu dilakukan proses \textit{compile} ulang, untuk mendapatkan PDF dokumen skripsi yang baru. Dokumen hasil \textit{compile} ulang diberi nama TC\_PE\_01\_v2.pdf dan TC\_PE\_04\_v2.pdf. 

Pengujian file yang sebelumnya bermasalah dilakukan dengan menggunakan file hasil \textit{compile} ulang. Perangkat lunak berhasil mengeluarkan laporan kesalahan yang ada pada dokumen skripsi. Permasalahan yang berkaitan dengan \textit{secured} PDF berhasil diatasi.

\subsection{Kesimpulan Pengujian Eksperimental}

Pengujian eksperimental telah dilakukan dengan menggunakan 5 kasus uji, berupa dokumen skripsi milik mahasiswa Informatika UNPAR. Pada pengujian ini dapat ditarik beberapa kesimpulan, yaitu sebagai berikut:

\begin{itemize}
	\item Dokumen PDF skripsi yang ada pada \textit{Github} FTIS UNPAR, tidak dapat digunakan sebagai kasus uji. Ada faktor bahwa file dokumen PDF yang telah diunggah pada Github akan menjadi \textit{secured document}. \textit{Library PdfParser} tidak dapat mengesktrak dokumen tersebut. Untuk dapat mengatasi hal ini, pengguna perangkat lunak memerlukan file latex dari dokumen tersebut, lalu dicompile ulang. Hal ini sudah dibuktikan pada pengujian eksperimental, dan perangkat lunak dapat mengeluarkan laporan kesalahan dari dokumen tersebut.
	
	\item Kasus dimana perangkat lunak tidak mengeluarkan laporan kesalahan, terjadi karena adanya \textit{memory limit}. \textit{Memory limit} yang diatur pada konfigurasi \textit{php.ini} adalah 128Mb. Masalah ini dapat diselesaikan dengan menambahkan memori seperti yang sudah dilakukan pada pengujian eksperimental.
	
	\item Dari pengujian ini, dapat dilihat bahwa ada sebuah \textit{library Pdf Parser} yang tidak ditangani dan kesalahan tersebut tidak dimunculkan pada \textit{exception}. Pada pengujian ini telah dijelaskan bahwa pesan kesalahan dari sistem disembunyikan dengan menggunakan tanda ''@''.
\end{itemize}