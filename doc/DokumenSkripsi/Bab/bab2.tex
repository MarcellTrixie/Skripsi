%versi 2 (8-10-2016)
\chapter{Landasan Teori}
\label{chap:teori}

Pada bab 2 akan diuraikan tentang landasan teori yang membahas \textit{regular expression} dan \textit{library PDF Parser}.

\section{\textit{Regular Expression}}
\label{sec:regex} 
 
\textit{Regular expression} (\textit{regex})~\cite{jan:12:cookbook} adalah jenis pola teks tertentu yang dapat digunakan pada banyak aplikasi modern dan bahasa pemrograman. \textit{Regex} biasanya dimanfaatkan untuk memverifikasi kecocokan antara input dengan pola teks, untuk menemukan teks yang cocok dengan pola dalam teks yang lebih besar, untuk mengganti teks yang cocok dengan pola dengan teks lain atau menyusun ulang bit dari teks yang cocok dan untuk membagi sebuah blok teks menjadi beberapa subteks. \textit{Regex}sudah banyak digunakan dalam pencocokan pola, misalnya untuk validasi beberapa string seperti \textit{username} dan \textit{password}, alamat \textit{e-mail}, alamat \textit{IP} ataupun nomor telepon. Pemanfaatan \textit{regex} dengan baik, dapat menyederhanakan banyak tugas pemrograman dan pemrosesan teks dalam kehidupan sehari-hari.

Istilah \textit{regex} berasal dari teori matematika dan komputer sains, yang mencerminkan sifat ekspresi dalam matematika yang disebut keteraturan. Ekspresi tersebut dapat diimplementasikan dalam perangkat lunak, dengan menggunakan \textit{Deterministic Finite Automaton} (DFA). DFA adalah \textit{finite state machine} yang tidak menggunakan \textit{backtracking}.

Regex dapat digunakan dalam berbagai bahasa pemrograman, salah satunya yaitu, \textit{Perl Compatible Regular Expression} (PCRE). PCRE~\cite{pcre} adalah serangkaian fungsi yang menerapkan pencocokan pola \textit{regex} dengan menggunakan sintaks dan semantik yang sama dengan bahasa pemrograman \textit{Perl 5}, meskipun ada beberapa sedikit perbedaan. Pada saat ini, implementasi yang digunakan sesuai dengan \textit{Perl} versi 5.005.

\subsection{Metakarakter}

Metakarakter pada \textit{regex} dibedakan menjadi 2 jenis berdasarkan dari posisinya, yaitu metakarakter \textit{outside square brackets} dan metakarakter \textit{outside square brackets}. Meskipun ada beberapa simbol metakarakter yang sama, namun fungsinya agak berbeda. Pada metakarakter \textit{outside square brackets} terdapat 14 simbol, sedangkan metakarakter \textit{inside square brackets} terdapat 3 simbol. Rincian dari ke-2 metakarakter tersebut akan dijelaskan sebagai berikut:
	
\begin{table}[H]
	\caption {Tabel metakarakter \textit{outside square brackets}} \label{tab:metacharacters-outside}
	\begin{center}
		\begin{tabular}{|c|l|}
		\hline 
		Simbol & Nama \\ 
		\hline 
		$\backslash$ & Backslash \\ 
		\hline 
		$\hat{}$ & Circumflex Anchor\\ 
		\hline 
		\$ & Dollar Anchor\\ 
		\hline 
		. & Dot \\ 
		\hline 
		[ & Left Square Bracket \\ 
		\hline 
		] & Right Square Bracket \\ 
		\hline 
		$\vert$ & Vertical Bar \\ 
		\hline 
		( & Left Parenthesis \\ 
		\hline 
		) & Right Parenthesis \\ 
		\hline 
		? & Question Mark \\ 
		\hline 
		* & Asterisk \\ 
		\hline 
		+ & Plus \\ 
		\hline 
		$\lbrace$ & Left Curly Brace \\ 
		\hline 
		$\rbrace$ & Right Curly Brace \\ 
		\hline 
		\end{tabular} 
	\end{center}
\end{table}

Pada tabel 2.1 telah disebutkan simbol-simbol yang digunakan pada metakarakter \textit{outside square brackets}. Berikut ini adalah penjelasan fungsi dari setiap metakarakter:

\begin{enumerate}
	\item \textit{Backslash} \\
	\textit{Backslash} yang berada di luar kelas karakter memiliki beberapa fungsi, yaitu:
	\begin{itemize}
		\item Membuat karakter lepas\\
		Apabila diikuti oleh karakter non-alfanumerik, metakarakter ini dapat menghilangkan makna khusus yang dimiliki oleh karakter tersebut. Misalnya pada saat ingin mencocokan karakater ''*'', dengan menggunakan metakarkter \textit{backslash} dapat dituliskan dengan ''$\backslash$*''. Jadi karakter \textit{asterisk} akan terbaca sebagai karakter bukan sebagai metakarakter.
				
		\item Menggunakan karakter yang tidak dapat ditulis dalam pola, seperti $\backslash$a (\textit{alarm}), $\backslash$cx (\textit{control-x}, $\backslash$e (\textit{escape}) dan lain-lain.
		
		\item Menentukan jenis karakter dalam pola, seperti $\backslash$d (angka desimal), $\backslash$D (non-angka desimal), $\backslash$w (karakter kata), $\backslash$W (non-karakter kata) dan lain-lain.
		
		\item Menggunakan pernyataan sederhana tertentu, seperti $\backslash$b (\textit{word boundary}), $\backslash$B (\textit{non-word boundary} dan lain-lain.
		
	\end{itemize}
	
	\item \textit{Anchor}\\
	\textit{Anchor} merupakan metakarakter yang terdiri dari simbol \textit{circumflex} ($\hat{}$) dan \textit{dollar} (\$). \textit{Circumflex} digunakan sebagai tanda awal yang memulai suatu teks string, sedangkan \textit{dollar} digunakan sebagai tanda akhir dari suatu teks pola.
	
	\item \textit{Dot} \\
	\textit{Dot} akan cocok dengan karakter apapun, keculai karakter \textit{newline}.
	
	\item \textit{Square brackets} \\
	\textit{Square brackets} terdiri dari pasangan ''['' dan '']'', memiliki fungsi untuk mendefinisikan kelas karakter. Kelas karakter akan berada di antara 2 karakter tersebut.	Contohnya kelas karakter numerik [0-9] sama dengan [0123456789].
	
	\item \textit{Vertical bar} \\
	\textit{Vertical bar} berfungsi untuk memisahkan beberapa kondisi pola alternatif yang berbeda. Sebagai contohnya untuk pola A | B, maka hasilnya akan mencocokan ''A'' atau ''B''.
	
	\item \textit{Parenthesis} \\
	\textit{Parenthesis} terdiri dari pasangan ''('' dan '')'', memiliki fungsi untuk mengelompokan subpola dalam \textit{regex}.	
	
	\item \textit{Quantifiers} \\
	\textit{Quantifiers} berfungsi untuk menunjukkan berapa banyak instance karakter, set karakter, atau kelas karakter yang harus dicocokkan. Pada metakarakter ini terdapat 3 jenis, yaitu:
	\begin{itemize}
		\item \textit{question mark} (?), dengan jumlah kuantifier antara 0 hingga 1 {0,1}.
		\item \textit{asterisk} (*), dengan jumlah kuantifier 0 atau lebih {0,}.
		\item \textit{plus} (+), dengan jumlah kuantifier 1 atau lebih {1,}.
	\end{itemize}	 

	\item \textit{Curly Brackets} \\
	\textit{Curly Brackets} terdiri dari pasangan ''{'' dan ''}'', memiliki fungsi untuk mendefinisikan angka minimal dan maksimum dari kuantifiers yang akan digunakan. Misalnya pola a{1,5}, akan cocok dengan ''a'', ''aa'', ''aaa'', ''aaaa'' atau ''aaaaa''.
	
\end{enumerate}
		
Metakarakter \textit{inside square brackets} merupakan bagian dari kelas karakter. Pada bagian kelas  karakter ini hanya terdapat 3 macam metakarakter saja. Berikut adalah metakarakter yang digunakan:
		
\begin{table}[H]
	\caption {Tabel metakarakter \textit{inside square brackets}} \label{tab:metacharacters-inside}
	\begin{center}
		\begin{tabular}{|p{1.5cm} |p{2.5cm}|}
		\hline 
		Simbol & Nama \\ 
		\hline 
		$\backslash$ & Backslash \\ 
		\hline 
		$\hat{}$ & Circumflex \\ 
		\hline 
		- & Hyphen \\ 
		\hline
		\end{tabular}
	\end{center}
\end{table}

Pada tabel 2.2 telah disebutkan macam-macam metakarakter \textit{inside square brackets}. Dari ke-3 metakarakter tersebut ada beberapa yang memiliki simbol yang sama dengan metakarakter \textit{outside square brackets}, namun fungsinya berbeda. Berikut adalah penjelasan dari fungsi metakarakter dari tabel 2.2:

\begin{enumerate}
	\item \textit{Backslash} \\
	Metakarakter ini fungsinya sama dengan \textit{backslash} yang ada pada \textit{outside square brackets}. Namun dari ke-4 fungsi tersebut hanya 3 fungsi saja yang digunakan, yaitu membuat karakter lepas, Menggunakan karakter yang tidak dapat ditulis dalam pola dan menentukan jenis karakter dalam pola.
	
	\item \textit{Circumflex} \\
	Metakarakter ini memiliki fungsi yang berbeda dengan yang digunakan pada \textit{outside square brackets}. Fungsinya untuk membuat negasi, namun hanya berlaku untuk karakter pertamanya saja. Contohnya [$\hat{}$0] akan cocok dengan semua karakter kecuali karakter 0.
	
	\item \textit{Hyphen} \\
	Metakarakter ini berfungsi untuk menentukan jangkauan dari sebuah karakter dalam kelas karakter, seperti [0-9] yang menandakan jangkauan karakter dari angka 0 hingga 9.
	
\end{enumerate}
	
\subsection{Kelas Karakter}

Kelas karakter adalah karakter yang memiliki atribut yang spesifik yang dikelompokan dalam sebuah kelas. Karakter tersebut dapat berbeda di setiap negara. Kelas karakter hanya valid digunakan pada \textit{regex} didalam tanda kurung siku pada \textit{bracket expression}. Perl mendukung notasi POSIX yang digunakan untuk kelas karakter. Dalam penggunaannya, kelas-kelas tersebut ditulis diantara ''[:'' dan '':]''. PCRE juga mendukung penggunaan notasi ini. Sebagai contoh untuk kelas alfanumerik, penulisannya yaitu [:alnum:]. Berikut ini akan dijelaskan macam-macam kelas karakter yang digunakan:

\begin{table}[H]
	\caption {Tabel kelas karakter} \label{tab:character classes}
	\begin{center}
		\begin{tabular}{|p{2 cm}|>{\raggedright} p{5 cm}| p{7.5 cm}|}
		\hline
		Kelas & Deskripsi & Keterangan \\ 
		\hline 
		alnum & Alfanumerik & Kelas yang berisi dengan karakter alfanumerik, meliputi angka dan huruf. Karakter-karakter yang termasuk yaitu, huruf a-Z atau A-Z dan angka 0-9. Simbol atau karakter khusus tidak termasuk dalam kelas ini. \newline \\ 
		\hline 
		alpha & Huruf & Kelas yang berisi dengan karakter huruf. Karakter-karakter yang termasuk yaitu, huruf a-Z atau A-Z.	Simbol atau karakter khusus tidak termasuk dalam kelas ini. \newline \\ 
		\hline 
		ascii & Kode karakter & Kelas yang merepresentasikan kode karakter dari 0-127. \newline \\ 
		\hline 
		blank & Spasi dan Tab & Karakter-karakter yang termasuk yaitu, TAB dan spasi. \newline \\ 
		\hline 
		cntrl & Karakter kontrol & Karakter kontrol adalah karakter yang tidak merepresentasikan simbol tetapi merepresentasikan character encoding. \newline \\ 
		\hline 
		digit & Angka desimal & Kelas yang berisi dengan karakter numerik. Karakter-karakter yang termasuk yaitu, angka 0-9. \newline \\ 
		\hline 
		\end{tabular} 
	\end{center}
\end{table}

\begin{table}[H]
	\caption {Tabel kelas karakter} \label{tab:character classes}
	\begin{center}
		\begin{tabular}{|p{2 cm}|>{\raggedright} p{5 cm}| p{7.5 cm}|}
		\hline
		Kelas & Deskripsi & Keterangan \\ 
		\hline 
		graph & Karakter cetak (kecuali spasi) & Kelas yang berisi karakter yang dapat dicetak dan tampak. Contoh karakter yang dapat dicetak namun tidak tampak adalah karakter spasi dan TAB. Jadi pada kelas ini karakter spasi dan tab tidak termasuk. \newline \\ 
		\hline 
		lower & Huruf kecil & Kelas yang berisi karakter dengan huruf kecil. \newline \\ 
		\hline 
		print & Karakter cetak (termasuk spasi) & Kelas yang berisi karakter yang dapat dicetak. Karakter yang termasuk kelas ini adalah spasi. \newline \\ 
		\hline 
		punct & Karakter cetak (kecuali alfanumerik) & Kelas yang berisi karakter yang dapat dicetak, namun tidak termasuk alfanumerik. \newline \\ 
		\hline 
		space & Ruang putih & Karakter yang termasuk kelas ini adalah spasi dan TAB. \newline \\ 
		\hline 
		upper & Huruf kapital & Kelas yang berisi karakter dengan huruf kapital. \newline \\ 
		\hline 
		word & Karakter ''word'' & Kelas yang berisi karakter kata. \newline \\ 
		\hline 
		xdigit & Heksadesimal & Kelas yang merepresentasikan bilangan hexadesimal, a-f atau A-F dan 0-9. \newline \\ 
		\hline 
		\end{tabular} 
	\end{center}
\end{table}

\section{PDF Parser}
\label{sec:pdfparser}

\textit{PDF Parser}~\cite{pdfparser} adalah sebuah \textit{PHP library} yang digunakan untuk mengekstrak data yang ada dalam sebuah file PDF. Fitur-fitur yang terdapat dalam \textit{PDF Parser} adalah sebagai berikut:

\begin{enumerate}
	\item Memuat / mengurai objek dan header
	\item Menampilkan data meta, seperti nama penulis, deskripsi dan sebagainya
	\item Menampilkan isi dari teks PDF
	\item Dapat digunakan untuk kompresi PDF
	\item Mendukung \textit{MAC OS Roman charset encoding}
	\item Menangani \textit{encoding} heksa dan oktal pada teks

Dari fitur yang sudah ada, masih ada yang belum bisa ditangani oleh \textit{library} ini. \textit{PDF Parser} belum dapat mengekstrak dokumen yang diamankan. Selain itu, \textit{library} ini tidak dapat mendeteksi jenis teks yang dicetak miring, tebal dan bergaris bawah. Hingga saat ini, pengembangan \textit{library PDF parser} masih terus berjalan. 

\textit{Library} ini dapat digunakan pada \textit{PHP} dengan versi 5.3 ke atas. Namun, sebelum menggunakannya, ada beberapa hal yang perlu dilakukan terlebih dahulu. Ada 2 macam cara untuk memasangnya, yaitu:

\begin{itemize}
	\item Menggunakan \textit{composer} \\
	\item Dijadikan sebagai \textit{standalone library} \\
\end{itemize}
	
\end{enumerate}

\section{Kamus Indonesia \textit{LibreOffice}}
\label{sec:kamusindo}

LibreOffice adalah sebuah paket aplikasi perkantoran berfitur lengkap yang tersedia secara gratis. LibreOffice menggunakan \textit{Open Document Format} (ODF) sebagai format aslinya untuk menyimpan dokumen. ODF menjadi format standar terbuka yang sedang diadopsi sebagai format file yang dibutuhkan untuk penerbitan dan penerimaan dokumen. LibreOffice juga dapat membuka dan menyimpan dokumen dalam format lainnya. Salah satunya format yang digunakan oleh beberapa versi dari Microsoft Office.

LibreOffice memiliki ekstensi untuk kamus Indonesia. Ekstensi ini sudah mengalami beberapa perkembangan, mulai dari versi 1.0 yang rilis pada tanggal 19 Mei 2012. Versi 1.0 merupakan hasil unggahan kembali dari ekstensi \textit{OpenOffice} yang terakhir diperbaharui pada tahun 2009. Pada tanggal 17 Mei 2014 versi 1.1 dirilis, pada versi ini LibreOffice versi 4.0 dapat menggunakan ekstensi ini. Selanjutnya, pada tanggal 15 Juli 2014 versi 2.0 dirilis. Pada versi yang paling baru ini digunakan metode baru dengan sirkumfiks yang kemudian mengubah daftar kata, sehingga memuat semua lema dari Kamus Besar Indonesia.