\lstdefinelanguage{plaintext}{
  sensitive=false,
  comment=[l]{//},
  morecomment=[s]{/*}{*/},
  identifierstyle=\color{black},
  morestring=[b]',
  morestring=[b]"
}

\lstset
{ 
    language=plaintext,
    basicstyle=\footnotesize,
    numbers=left,
    stepnumber=1,
    showstringspaces=false,
    tabsize=1,
    breaklines=true,
    breakatwhitespace=false,
    frame=leftline
}

%versi 2 (8-10-2016)
\chapter{Landasan Teori}
\label{chap:teori}

Pada bab ini akan dibahas mengenai landasan teori yang membahas \textit{regular expression}, \textit{library PdfParser} dan kamus bahasa Indonesia \textit{LibreOffice}.

\section{\textit{Regular Expression}}
\label{sec:regex} 
 
\textit{Regular expression} (\textit{regex})~\cite{jan:12:cookbook} adalah jenis pola teks tertentu yang dapat digunakan pada banyak aplikasi modern dan bahasa pemrograman. \textit{Regex} biasanya dimanfaatkan untuk memverifikasi kecocokan antara input dengan pola teks, untuk menemukan teks yang cocok dengan pola dalam teks yang lebih besar, untuk mengganti teks yang cocok dengan pola dengan teks lain atau menyusun ulang bit dari teks yang cocok dan untuk membagi sebuah blok teks menjadi beberapa subteks.

\textit{Regex} sudah banyak digunakan dalam pencocokan pola, misalnya untuk validasi beberapa string seperti \textit{username} dan \textit{password}, alamat \textit{e-mail}, alamat \textit{IP} ataupun nomor telepon. Pemanfaatan \textit{regex} dengan baik, dapat menyederhanakan banyak tugas pemrograman dan pemrosesan teks dalam kehidupan sehari-hari. Istilah \textit{regex} berasal dari teori matematika dan komputer sains, yang mencerminkan sifat ekspresi dalam matematika yang disebut keteraturan. Ekspresi tersebut dapat diimplementasikan dalam perangkat lunak, dengan menggunakan \textit{Deterministic Finite Automaton} (DFA). DFA adalah \textit{finite state machine} yang tidak menggunakan \textit{backtracking}.

\textit{Regex} dapat digunakan dalam berbagai bahasa pemrograman, salah satunya yaitu, \textit{Perl Compatible Regular Expression} (PCRE). PCRE~\cite{pcre} adalah serangkaian fungsi yang menerapkan pencocokan pola \textit{regex} dengan menggunakan sintaks dan semantik yang sama dengan bahasa pemrograman \textit{Perl 5}, meskipun ada beberapa sedikit perbedaan. Pada saat ini, implementasi yang digunakan sesuai dengan \textit{Perl} versi 5.005.

\subsection{Metakarakter}

Karakter meta pada \textit{regex} dibedakan menjadi 2 jenis berdasarkan dari posisinya, yaitu karakter meta \textit{inside square brackets} dan karakter meta \textit{outside square brackets}. Meskipun ada beberapa simbol karakter meta yang sama, namun fungsinya agak berbeda. Karakter meta \textit{outside square brackets} memiliki 14 jenis karakter, sedangkan karakter meta \textit{inside square brackets} hanya memiliki 3 jenis karakter. Rincian dari ke-2 karakter meta tersebut akan dijelaskan sebagai berikut:

\begin{table}[H]
	\renewcommand{\arraystretch}{1.5}
	\caption {Tabel metakarakter \textit{outside square brackets}} 
	\label{tab:metacharacters-outside}
	\begin{center}
		\begin{tabular}{|c|l|}
		\hline 
		Simbol & Nama \\ 
		\hline 
		$\backslash$ & Backslash \\ 
		\hline 
		$\hat{}$ & Circumflex Anchor\\ 
		\hline 
		\$ & Dollar Anchor\\ 
		\hline 
		\end{tabular} 
	\end{center}
\end{table}
	
\begin{table}[H]
	\renewcommand{\arraystretch}{1.5}
	\caption {Tabel metakarakter \textit{outside square brackets}} 
	\label{tab:metacharacters-outside2}
	\begin{center}
		\begin{tabular}{|c|l|}
		\hline 
		Simbol & Nama \\ 
		\hline 
		. & Dot \\ 
		\hline 
		[ & Left Square Bracket \\ 
		\hline
		] & Right Square Bracket \\ 
		\hline 
		$\vert$ & Vertical Bar \\ 
		\hline 
		( & Left Parenthesis \\ 
		\hline 
		) & Right Parenthesis \\ 
		\hline 
		$\lbrace$ & Left Curly Bracket \\ 
		\hline
		$\rbrace$ & Right Curly Bracket \\ 
		\hline 
		? & Question Mark \\ 
		\hline 
		* & Asterisk \\ 
		\hline 
		+ & Plus \\ 
		\hline 
		\end{tabular} 
	\end{center}
\end{table}

Pada Tabel \ref{tab:metacharacters-outside} dan \ref{tab:metacharacters-outside2}, terdapat 14 karakter yang termasuk karakter meta \textit{outside square brackets}. Setiap karakter yang ada pada tabel tersebut memiliki fungsi yang berbeda-beda. Berikut ini adalah penjelasan fungsi dari setiap karakter meta.

\begin{enumerate}
	\item \textit{Backslash} \\
	Karakter meta \textit{backslash} yang berada di luar kelas karakter memiliki beberapa fungsi, yaitu:
	\begin{itemize}
		\item Membuat karakter lepas\\
		Apabila diikuti oleh karakter non-alfanumerik, karakter meta ini dapat menghilangkan makna khusus yang dimiliki oleh karakter tersebut. Misalnya pada saat ingin mencocokan karakter ''*'', dengan menggunakan karakter meta \textit{backslash} dapat dituliskan dengan ''$\backslash$*''. Jadi karakter \textit{asterisk} akan terbaca sebagai karakter biasa bukan sebagai karakter meta.
				
		\item Menggunakan karakter yang tidak dapat ditulis dalam pola, seperti $\backslash$n (\textit{newline}), $\backslash$R (\textit{line break}, $\backslash$t (\textit{tab}) dan lain-lain.
		
		\item Menentukan jenis karakter dalam pola, seperti $\backslash$d (angka desimal), $\backslash$D (non-angka desimal), $\backslash$w (karakter kata), $\backslash$W (non-karakter kata) dan lain-lain.
		
		\item Menggunakan \textit{assertions}, seperti $\backslash$b (\textit{word boundary}), $\backslash$B (\textit{non-word boundary} dan lain-lain.
		
	\end{itemize}
	
	\item \textit{Anchor}\\
	\textit{Anchor} merupakan karakter meta yang terdiri dari simbol \textit{circumflex} ( $\hat{}$ ) dan \textit{dollar} (\$). Simbol \textit{Circumflex} digunakan untuk menandai awal dari pola pencarian, sedangkan simbol \textit{dollar} digunakan untuk menandai akhir dari pola pencarian. Contohnya pola ''/( $\hat{}$ [0-9].*[A-Za-z]\$ )/'', yang artinya pola tersebut dimulai dengan karakter numerik dan diakhiri dengan karakter huruf.
	
	\item \textit{Dot} \\
	\textit{Dot} akan cocok dengan karakter apapun, keculai karakter \textit{line break}. Contohnya untuk pola ''/HelloW.rld/'', pola tersebut dapat cocok dengan HelloWarld, HelloWirld, HelloWorld, dan seterusnya.
	
	\item \textit{Square brackets} \\
	\textit{Square brackets} terdiri dari pasangan ''['' dan '']'', memiliki fungsi untuk mendefinisikan kelas karakter. Kelas karakter akan berada di antara 2 karakter tersebut. Contohnya kelas karakter numerik ''/[0-9]/'' sama dengan ''/[0123456789]/''.
	
	\item \textit{Vertical bar} \\
	\textit{Vertical bar} atau garis tegak lurus, berfungsi untuk memisahkan beberapa kondisi pola alternatif yang berbeda. Sebagai contohnya untuk pola ''/( A | B )/'', maka hasilnya akan mencocokan karakter ''A'' atau karakter ''B''.
	
	\item \textit{Parenthesis} \\
	\textit{Parenthesis} atau kurung biasa, merupakan karakter meta yang terdiri dari pasangan ''('' dan '')''. Karakter meta ini memiliki fungsi untuk mengelompokan suatu pola dalam \textit{regex}. Pola yang berada di dalam kurung paling dalam akan diolah terlebih dahulu, setelah itu baru mengolah tanda kurung yang ada di luarnya.
	
	\item \textit{Curly Brackets} \\
	\textit{Curly Brackets} atau kurung kurawal, merupakan karakter meta terdiri dari pasangan ''$\lbrace$'' dan ''$\rbrace$''. Karakter meta ini memiliki fungsi untuk memberi informasi jumlah karakter atau pola di dalam kurung siku yang harus ada. Kurung kurawal biasanya diletakan setelah kurung siku. Misalnya pola ''/[a]$\lbrace$1,5$\rbrace$/'', akan cocok dengan ''a'', ''aa'', ''aaa'', ''aaaa'' atau ''aaaaa''.	
	
	\item \textit{Quantifiers} \\
	\textit{Quantifiers} berfungsi untuk menunjukkan berapa banyak instance karakter, set karakter, atau kelas karakter yang harus dicocokkan. Pada karakter meta ini terdapat 3 jenis, yaitu:
	\begin{itemize}
		\item \textit{Question mark} (?)\\
		Pengulangan yang dapat dilakukan oleh kuantifier ini yaitu $\lbrace$ 0,1 $\rbrace$ (0 hingga 1). Contohnya pola ''/(Hello)?World/'' akan cocok dengan ''World'' atau ''HelloWorld''.
		
		\item \textit{Asterisk} (*) \\
		Pengulangan yang dapat dilakukan oleh kuantifier ini yaitu $\lbrace$ 0, $\rbrace$ (0 atau lebih). Contohnya pola ''/(He)*lloWorld/'' akan cocok dengan ''lloWorld'', ''HelloWorld'' atau ''HeHelloWorld''.
		 
		\item \textit{Plus} (+) \\
		Pengulangan yang dapat dilakukan oleh kuantifier ini yaitu $\lbrace$ 1, $\rbrace$ (1 atau lebih). Contohnya pola ''/(HelloWorld)+/'' akan cocok dengan ''HelloWorld'', ''HelloWorldHelloWorld'' dan seterusnya.
	\end{itemize}	 
	
\end{enumerate}
		
Karakter meta \textit{inside square brackets} merupakan bagian dari kelas karakter. Pada bagian kelas karakter ini hanya terdapat 3 macam karakter meta saja. Jenis-jenis karakter meta tersebut akan dijelaskan pada Tabel \ref{tab:metacharacters-inside}.
		
\begin{table}[H]
	\renewcommand{\arraystretch}{1.5}
	\caption {Tabel metakarakter \textit{inside square brackets}} 
	\label{tab:metacharacters-inside}
	\begin{center}
		\begin{tabular}{|c|l|}
		\hline 
		Simbol & Nama \\ 
		\hline 
		$\backslash$ & Backslash \\ 
		\hline 
		$\hat{}$ & Circumflex \\ 
		\hline 
		- & Hyphen \\ 
		\hline
		\end{tabular}
	\end{center}
\end{table}

Pada Tabel \ref{tab:metacharacters-inside} telah disebutkan macam-macam karakter meta \textit{inside square brackets}. Dari ke-3 metakarakter tersebut ada beberapa yang memiliki simbol yang sama dengan karakter meta \textit{outside square brackets}, namun fungsinya berbeda. Fungsi dari setiap karakter meta tersebut akan dijelaskan sebagai berikut.

\begin{enumerate}
	\item \textit{Backslash} \\
	Karakter meta ini fungsinya sama dengan \textit{backslash} yang ada pada \textit{outside square brackets}. Namun dari ke-4 fungsi tersebut hanya 3 fungsi saja yang digunakan, yaitu membuat karakter lepas, Menggunakan karakter yang tidak dapat ditulis dalam pola dan menentukan jenis karakter dalam pola.
	
	\item \textit{Circumflex} \\
	Karakter meta ini memiliki fungsi yang berbeda dengan yang digunakan pada \textit{outside square brackets}. Fungsinya untuk membuat negasi, namun hanya berlaku untuk karakter pertamanya saja. Contohnya [ $\hat{}$ 0 ] akan cocok dengan semua karakter kecuali karakter 0.
	
	\item \textit{Hyphen} \\
	Karakter meta ini berfungsi untuk menentukan jangkauan dari sebuah karakter dalam kelas karakter, seperti [ 0-9 ] yang menandakan jangkauan karakter dari angka 0 hingga 9.
	
\end{enumerate}
	
\subsection{Kelas Karakter}

Kelas karakter adalah karakter yang memiliki atribut yang spesifik yang dikelompokan dalam sebuah kelas. Karakter tersebut dapat berbeda di setiap negara. Kelas karakter hanya valid digunakan pada \textit{regex} didalam tanda kurung siku pada \textit{bracket expression}. Perl mendukung notasi POSIX yang digunakan untuk kelas karakter. Dalam penggunaannya, kelas-kelas tersebut ditulis diantara ''[:'' dan '':]''. PCRE juga mendukung penggunaan notasi ini. Sebagai contoh untuk kelas alfanumerik, penulisannya yaitu [:alnum:]. Berikut ini akan dijelaskan macam-macam kelas karakter yang digunakan pada Tabel \ref{tab:character classes1} dan \ref{tab:character classes2}.

\begin{table}[H]
	\renewcommand{\arraystretch}{1.5}
	\caption {Tabel kelas karakter} \label{tab:character classes1}
	\begin{center}
		\begin{tabular}{|p{2 cm}|>{\raggedright} p{5 cm}| p{7.5 cm}|}
		\hline
		Kelas & Deskripsi & Keterangan \\ 
		\hline 
		alnum & Alfanumerik & Kelas yang berisi dengan karakter alfanumerik, meliputi angka dan huruf. Karakter-karakter yang termasuk yaitu, huruf a-Z, A-Z dan angka 0-9. Simbol atau karakter khusus tidak termasuk dalam kelas ini. \newline \\ 
		\hline 
		alpha & Huruf & Kelas yang berisi dengan karakter huruf. Karakter-karakter yang termasuk yaitu, huruf a-Z atau A-Z.	Simbol atau karakter khusus tidak termasuk dalam kelas ini. \newline \\ 
		\hline 
		ascii & Kode karakter & Kelas yang merepresentasikan kode karakter dari 0-127. \newline \\ 
		\hline 
		blank & Spasi dan Tab & Karakter-karakter yang termasuk yaitu, TAB dan spasi. \newline \\ 
		\hline
		\end{tabular} 
	\end{center}
\end{table}

\begin{table}[H]
	\renewcommand{\arraystretch}{1.5}
	\caption {Tabel kelas karakter} \label{tab:character classes2}
	\begin{center}
		\begin{tabular}{|p{2 cm}|>{\raggedright} p{5 cm}| p{7.5 cm}|}
		\hline
		Kelas & Deskripsi & Keterangan \\ 
		\hline 
		blank & Spasi dan Tab & Karakter-karakter yang termasuk yaitu, TAB dan spasi. \newline \\ 
		\hline
		cntrl & Karakter kontrol & Karakter kontrol adalah karakter yang tidak merepresentasikan simbol tetapi merepresentasikan character encoding. \newline \\ 		
		\hline 
		digit & Angka desimal & Kelas yang berisi dengan karakter numerik. Karakter-karakter yang termasuk yaitu, angka 0-9. \newline \\ 		
		\hline 
		graph & Karakter cetak (kecuali spasi) & Kelas yang berisi karakter yang dapat dicetak dan tampak. Contoh karakter yang dapat dicetak namun tidak tampak adalah karakter spasi dan TAB. Jadi pada kelas ini karakter spasi dan tab tidak termasuk. \newline \\ 
		\hline 
		lower & Huruf kecil & Kelas yang berisi karakter dengan huruf kecil. Karakter yang termasuk kelas ini adalah huruf a-z. \newline \\ 
		\hline
		print & Karakter cetak (termasuk spasi) & Kelas yang berisi karakter yang dapat dicetak. Karakter yang termasuk kelas ini adalah spasi. \newline \\ 
		\hline   
		punct & Karakter cetak (kecuali alfanumerik) & Kelas yang berisi karakter yang dapat dicetak, namun tidak termasuk alfanumerik. \newline \\ 
		\hline		
		space & Ruang putih & Karakter yang termasuk kelas ini adalah spasi dan TAB. \newline \\ 		
		\hline
		upper & Huruf kapital & Kelas yang berisi karakter dengan huruf kapital. \newline \\ 
		\hline 
		word & Karakter ''word'' & Kelas yang berisi karakter kata. \newline \\ 
		\hline 
		xdigit & Heksadesimal & Kelas yang merepresentasikan bilangan hexadesimal, a-f atau A-F dan 0-9. \newline \\ 
		\hline
		\end{tabular} 
	\end{center}
\end{table}

\section{PdfParser}
\label{sec:pdfparser}

\textit{PdfParser}~\cite{pdfparser} adalah sebuah \textit{library} yang digunakan untuk mengekstrak data yang ada dalam sebuah file PDF. \textit{Library} ini digunakan untuk kebutuhan pengguna yang menggunakan bahasa pemrograman PHP. \textit{PDF Parser} dapat digunakan pada \textit{PHP} dengan versi 5.3 ke atas. Terdapat beberapa fitur yang dimiliki oleh \textit{library} ini. Berikut adalah fitur yang sudah dapat digunakan:

\begin{itemize}
	\item Memuat / mengurai objek dan header.
	\item Menampilkan data meta, seperti nama penulis, deskripsi dan sebagainya.
	\item Menampilkan isi dari teks PDF.
	\item Dapat digunakan untuk kompresi PDF.
	\item Mendukung \textit{MAC OS Roman charset encoding}.
	\item Menangani \textit{encoding} heksa dan oktal pada teks.
\end{itemize}

Dari fitur yang sudah ada, masih ada beberapa hal yang masih belum bisa ditangani oleh \textit{library} ini. Berikut adalah beberapa kekurangan yang dimiliki oleh \textit{PdfParser} berdasarkan pengamatan:

\begin{itemize}
	\item Tidak dapat mengekstrak PDF yang termasuk dokumen \textit{secured}.
	\item Tidak dapat mendeteksi jenis teks (cetak miring, tebal dan garis bawah)
	\item Tidak dapat mengesktrak gambar
	\item Hasil ekstrak tabel hanya berupa string saja (tidak dalam bentuk tabel)
\end{itemize}

\section{Kamus Indonesia \textit{LibreOffice}}
\label{sec:kamusindo}

LibreOffice~\cite{libreoffice} adalah sebuah paket aplikasi perkantoran berfitur lengkap yang tersedia secara gratis. LibreOffice menggunakan \textit{Open Document Format} (ODF) sebagai format aslinya untuk menyimpan dokumen. ODF menjadi format standar terbuka yang sedang diadopsi sebagai format file yang dibutuhkan untuk penerbitan dan penerimaan dokumen. LibreOffice juga dapat membuka dan menyimpan dokumen dalam format lainnya. Salah satunya format yang digunakan oleh beberapa versi dari Microsoft Office.

LibreOffice memiliki ekstensi kamus Indonesia yang digunakan pada LibreOffice Writer, untuk melakukan pemeriksaan kata yang dituliskan oleh pengguna. Ekstensi ini sudah mengalami beberapa perkembangan, mulai dari versi 1.0 yang rilis pada tanggal 19 Mei 2012. Versi 1.0 merupakan hasil unggahan kembali dari ekstensi \textit{OpenOffice} yang terakhir diperbaharui pada tahun 2009. Pada tanggal 17 Mei 2014 versi 1.1 dirilis, pada versi ini LibreOffice versi 4.0 dapat menggunakan ekstensi ini. Selanjutnya, pada tanggal 15 Juli 2014 versi 2.0 dirilis. Pada versi yang paling baru ini digunakan metode baru dengan sirkumfiks yang kemudian mengubah daftar kata, sehingga memuat semua lema dari Kamus Besar Indonesia.

Ekstensi kamus Indonesia ini dapat diunduh secara gratis oleh para pengguna melalui halaman resmi dari \textit{LibreOffice}. Ekstensi yang dapat diunduh tersebut memiliki nama \textit{id\_id.oxt}. Dalam ekstensi tersebut, terdapat beberapa file yang berisi informasi tentang kamus Indonesia, yaitu \textit{id\_ID.aff} dan \textit{id\_ID.dic}.

\subsection{File \textit{id\_ID.dic}}

File ini berisi kata-kata yang akan digunakan dalam kamus bahasa Indonesia. File ini dapat dibuka dengan menggunakan aplikasi \textit{notepad} atau teks editor lainnya. Setelah dibuka menggunakan teks editor, akan terlihat 31129 baris yang berisi kata-kata dasar yang ada dalam kamus bahasa Indonesia \textit{LibreOffice}. Untuk menjelaskan hal tersebut, akan diambil beberapa potong baris dari file tersebut.
	
	\begin{lstlisting}[caption={Potongan kode untuk file \textit{id\_ID.dic}}			\label{lst:idDic},language=plaintext,xleftmargin=.35\textwidth] 
31128
a
ab
aba
aba-aba
abad/i0
abadi/DkMkO0k0nl
abadiah
abah/DkMk
abah-abah
	\end{lstlisting}
\medskip
	
Pada Listing \ref{lst:idDic}, baris pertama pada file ini menyatakan banyaknya kata yang ada pada kamus. Kata yang ada dalam kamus tersebut berjumlah 31128 buah. Baris ke-2 hingga ke-31129 merupakan kata-kata yang ada pada kamus. Dalam kamus ini belum dapat pengelompokan kata berdasarkan jenisnya, misalkan kata benda, kata sifat, dan sebagainya. Namun ada beberapa kata tersebut ada yang terlihat berbeda, seperti pada baris 6, 7 dan 9. Di sebelah kiri dari kata tersebut terdapat garis miring yang disertai sebuah kode. Kode tersebut berarti bahwa kata tersebut dapat ditambahkan oleh imbuhan tertentu. Penjelasan tentang kode tersebut akan diuraikan lebih lanjut pada pembahasan file \textit{id\_ID.aff}. 

\subsection{File \textit{id\_ID.aff}}

File ini berisi daftar prefiks, sufiks dan konfiks, untuk kamus Indonesia. Prefiks akan disimbolkan dengan huruf kapital, sedangkan sufiks akan disimbolkan dengan huruf kecil. Selain ke-2 hal tesebut, ada juga yang merupakan gabungan dari prefiks dan sufiks, yaitu konfiks. Setiap imbuhan memiliki kodenya masing-masing, dan sebuah kata dapat memiliki lebih dari 1 kombinasi imbuhan. Berikut adalah penjelasan dari imbuhan yang telah disebutkan:

\begin{enumerate}
	\item \textit{Prefiks} \\
	\textit{Prefiks} atau awalan adalah imbuhan yang ditambahkan pada bagian awal sebuah kata dasar atau bentuk dasar, misalnya ber-, pe-, me- dan lain-lain. Penulisan kode awalan pada file ini akan disimbolkan dengan huruf kapital.
	
	\begin{lstlisting}[caption={Potongan kode untuk prefiks}			\label{lst:prefiks},language=plaintext,xleftmargin=.35\textwidth] 
PFX B0 Y 2	# ber-
PFX B0 0 ber  	[^r]
PFX B0 0 be    	r
	\end{lstlisting}
\medskip
	
	Listing \ref{lst:prefiks} merupakan salah satu contoh dari awalan yang terdapat pada kamus bahasa Indonesia. Berikut ini adalah penjelasan dari baris kode tersebut:	

	\begin{itemize}
		\item PFX\\
		Menandakan bahwa kode tersebut merupakan awalan
		
		\item B0\\
		Huruf B menunjukan awalan yang dimulai dengan huruf B, dan angka 0 menunjukan bentuk awalan asli
		
		\item 2\\
		Menandakan bahwa ada 2 jenis awalan yang dapat digunakan, yaitu ber- dan be-
		\item -ber\\
		Menandakan bahwa kode tersebut merupakan awalan ber-
	\end{itemize}	
		
	\item \textit{Sufiks} \\	
	\textit{Sufiks} atau akhiran adalah imbuhan yang ditambahkan pada bagian belakang kata dasar, misalnya -an, -kan, -i dan lain-lain. Penulisan kode akhiran pada file ini akan disimbolkan dengan huruf kecil.
	
	\begin{lstlisting}[caption={Potongan kode untuk sufiks}			\label{lst:sufiks},language=plaintext,xleftmargin=.35\textwidth] 
SFX k0 Y  3	 # -kan
SFX k0 0  kan    . 
SFX k0 0  kanlah
SFX k0 0  kankah
	\end{lstlisting}
\medskip
	
	 Listing \ref{lst:sufiks} merupakan salah satu contoh dari akhiran yang terdapat pada kamus bahasa Indonesia. Berikut ini adalah penjelasan dari baris kode tersebut:
	
	\begin{itemize}
		\item SFX\\
		Menandakan bahwa kode tersebut merupakan akhiran
		
		\item k0\\
		Huruf k menunjukan akhiran yang dimulai dengan huruf k, dan angka 0 menunjukan bentuk akhiran asli
		
		\item 3\\
		Menandakan bahwa ada 3 jenis akhiran yang dapat digunakan, yaitu -kan, -kanlah dan -kankah
		
		\item -kan\\
		Menandakan bahwa kode tersebut merupakan akhiran -kan 
	\end{itemize}
	
	\item \textit{Konfiks} \\	
	\textit{Konfiks} adalah afiks tunggal yang terjadi dari dua unsur yang terpisah, misalnya ke-...-an dalam kemerdekaan.
	
	\begin{lstlisting}[caption={Potongan kode untuk konfiks}			\label{lst:konfiks},language=plaintext,xleftmargin=.35\textwidth] 
PFX KT Y 2	# keter- (keter-an)
PFX KT 0 keter/A1	[^r]
PFX KT 0 kete/A1	r

SFX Sa Y 1	# -an (se-an) 	+ o0
SFX Sa 0 an/S1o0A1 . 
	\end{lstlisting}
	
	Listing \ref{lst:konfiks} merupakan salah satu contoh dari konfiks yang terdapat pada kamus bahasa Indonesia. Berikut ini adalah penjelasan dari baris kode tersebut:
	
	\begin{itemize}
		\item Kolom pertama biasanya digunakan sebagai tanda untuk mengetahui apakah kode tersebut termasuk awalan atau akhiran. Namun pada konfiks tidak ada simbol khusus yang menandakan bahwa kode tersebut adalah sebuah konfiks. Pada baris 1 tertulis ”PFX”, yang berarti konfiks ini berakar pada awalan tertentu. Pada baris 5 tertulis ”SFX”, yang berarti konfiks ini berakar pada akhiran tertentu.
		
		\item Kolom ke-2 menunjukan simbol dari awalan atau akhiran yang digunakan. Pada baris 1, konfiks itu berasal dari awalan keter-, dapat dilihat bahwa huruf K berarti ”ke” dan huruf T adalah ”ter”. Pada baris 5, huruf S berarti awalan yang berasal dari huruf S dan huruf a berarti akhiran yang berasal dari huruf a.
		
		\item Kolom ke-4 menunjukan berapa banyak jumlah konfiks yang dapat digunakan. Pada baris 1 terdapat 2 jenis, sedangkan pada baris 5 terdapat 1 jenis saja.
		
		\item Kolom ke-5 menunjukan jenis konfiks yang digunakan. Baris 1-3 merupakan contoh dari konfiks keter-an, sedangkan baris 5-6 merupakan contoh dari konfiks se-an.
	\end{itemize}
	
\end{enumerate}