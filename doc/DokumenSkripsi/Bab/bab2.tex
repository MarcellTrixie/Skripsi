\lstdefinelanguage{plaintext}{
  sensitive=false,
  comment=[l]{//},
  morecomment=[s]{/*}{*/},
  identifierstyle=\color{black},
  morestring=[b]',
  morestring=[b]"
}

\lstset
{ 
    language=plaintext,
    basicstyle=\footnotesize,
    numbers=left,
    stepnumber=1,
    showstringspaces=false,
    tabsize=1,
    breaklines=true,
    breakatwhitespace=false,
    frame=leftline
}

%versi 2 (8-10-2016)
\chapter{Landasan Teori}
\label{chap:teori}

Pada bab 2 akan diuraikan tentang landasan teori yang membahas \textit{regular expression} dan \textit{library PDF Parser}.

\section{\textit{Regular Expression}}
\label{sec:regex} 
 
\textit{Regular expression} (\textit{regex})~\cite{jan:12:cookbook} adalah jenis pola teks tertentu yang dapat digunakan pada banyak aplikasi modern dan bahasa pemrograman. \textit{Regex} biasanya dimanfaatkan untuk memverifikasi kecocokan antara input dengan pola teks, untuk menemukan teks yang cocok dengan pola dalam teks yang lebih besar, untuk mengganti teks yang cocok dengan pola dengan teks lain atau menyusun ulang bit dari teks yang cocok dan untuk membagi sebuah blok teks menjadi beberapa subteks. \textit{Regex}sudah banyak digunakan dalam pencocokan pola, misalnya untuk validasi beberapa string seperti \textit{username} dan \textit{password}, alamat \textit{e-mail}, alamat \textit{IP} ataupun nomor telepon. Pemanfaatan \textit{regex} dengan baik, dapat menyederhanakan banyak tugas pemrograman dan pemrosesan teks dalam kehidupan sehari-hari.

Istilah \textit{regex} berasal dari teori matematika dan komputer sains, yang mencerminkan sifat ekspresi dalam matematika yang disebut keteraturan. Ekspresi tersebut dapat diimplementasikan dalam perangkat lunak, dengan menggunakan \textit{Deterministic Finite Automaton} (DFA). DFA adalah \textit{finite state machine} yang tidak menggunakan \textit{backtracking}.

Regex dapat digunakan dalam berbagai bahasa pemrograman, salah satunya yaitu, \textit{Perl Compatible Regular Expression} (PCRE). PCRE~\cite{pcre} adalah serangkaian fungsi yang menerapkan pencocokan pola \textit{regex} dengan menggunakan sintaks dan semantik yang sama dengan bahasa pemrograman \textit{Perl 5}, meskipun ada beberapa sedikit perbedaan. Pada saat ini, implementasi yang digunakan sesuai dengan \textit{Perl} versi 5.005.

\subsection{Metakarakter}

Metakarakter pada \textit{regex} dibedakan menjadi 2 jenis berdasarkan dari posisinya, yaitu metakarakter \textit{outside square brackets} dan metakarakter \textit{outside square brackets}. Meskipun ada beberapa simbol metakarakter yang sama, namun fungsinya agak berbeda. Pada metakarakter \textit{outside square brackets} terdapat 14 simbol, sedangkan metakarakter \textit{inside square brackets} terdapat 3 simbol. Rincian dari ke-2 metakarakter tersebut akan dijelaskan sebagai berikut:
	
\begin{table}[H]
	\caption {Tabel metakarakter \textit{outside square brackets}} \label{tab:metacharacters-outside}
	\begin{center}
		\begin{tabular}{|c|l|}
		\hline 
		Simbol & Nama \\ 
		\hline 
		$\backslash$ & Backslash \\ 
		\hline 
		$\hat{}$ & Circumflex Anchor\\ 
		\hline 
		\$ & Dollar Anchor\\ 
		\hline 
		. & Dot \\ 
		\hline 
		[ ] & Square Bracket \\ 
		\hline 
		$\vert$ & Vertical Bar \\ 
		\hline 
		( ) & Parenthesis \\ 
		\hline 
		? & Question Mark \\ 
		\hline 
		* & Asterisk \\ 
		\hline 
		+ & Plus \\ 
		\hline 
		$\lbrace$ $\rbrace$ & Curly Bracket \\ 
		\hline 
		\end{tabular} 
	\end{center}
\end{table}

Pada tabel \ref{tab:metacharacters-outside} telah disebutkan simbol-simbol yang digunakan pada metakarakter \textit{outside square brackets}. Berikut ini adalah penjelasan fungsi dari setiap metakarakter:

\begin{enumerate}
	\item \textit{Backslash} \\
	\textit{Backslash} yang berada di luar kelas karakter memiliki beberapa fungsi, yaitu:
	\begin{itemize}
		\item Membuat karakter lepas\\
		Apabila diikuti oleh karakter non-alfanumerik, metakarakter ini dapat menghilangkan makna khusus yang dimiliki oleh karakter tersebut. Misalnya pada saat ingin mencocokan karakter ''*'', dengan menggunakan metakarakter \textit{backslash} dapat dituliskan dengan ''$\backslash$*''. Jadi karakter \textit{asterisk} akan terbaca sebagai karakter bukan sebagai metakarakter.
				
		\item Menggunakan karakter yang tidak dapat ditulis dalam pola, seperti $\backslash$a (\textit{alarm}), $\backslash$cx (\textit{control-x}, $\backslash$e (\textit{escape}) dan lain-lain.
		
		\item Menentukan jenis karakter dalam pola, seperti $\backslash$d (angka desimal), $\backslash$D (non-angka desimal), $\backslash$w (karakter kata), $\backslash$W (non-karakter kata) dan lain-lain.
		
		\item Menggunakan pernyataan sederhana tertentu, seperti $\backslash$b (\textit{word boundary}), $\backslash$B (\textit{non-word boundary} dan lain-lain.
		
	\end{itemize}
	
	\item \textit{Anchor}\\
	\textit{Anchor} merupakan metakarakter yang terdiri dari simbol \textit{circumflex} ($\hat{}$) dan \textit{dollar} (\$). \textit{Circumflex} digunakan sebagai tanda awal yang memulai suatu teks string, sedangkan \textit{dollar} digunakan sebagai tanda akhir dari suatu teks pola.
	
	\item \textit{Dot} \\
	\textit{Dot} akan cocok dengan karakter apapun, keculai karakter \textit{newline}.
	
	\item \textit{Square brackets} \\
	\textit{Square brackets} terdiri dari pasangan ''['' dan '']'', memiliki fungsi untuk mendefinisikan kelas karakter. Kelas karakter akan berada di antara 2 karakter tersebut.	Contohnya kelas karakter numerik [0-9] sama dengan [0123456789].
	
	\item \textit{Vertical bar} \\
	\textit{Vertical bar} berfungsi untuk memisahkan beberapa kondisi pola alternatif yang berbeda. Sebagai contohnya untuk pola A | B, maka hasilnya akan mencocokan ''A'' atau ''B''.
	
	\item \textit{Parenthesis} \\
	\textit{Parenthesis} terdiri dari pasangan ''('' dan '')'', memiliki fungsi untuk mengelompokan subpola dalam \textit{regex}.	
	
	\item \textit{Quantifiers} \\
	\textit{Quantifiers} berfungsi untuk menunjukkan berapa banyak instance karakter, set karakter, atau kelas karakter yang harus dicocokkan. Pada metakarakter ini terdapat 3 jenis, yaitu:
	\begin{itemize}
		\item \textit{question mark} (?), dengan jumlah kuantifier $\lbrace$ 0,1 $\rbrace$ (0 hingga 1).
		\item \textit{asterisk} (*), dengan jumlah kuantifier $\lbrace$ 0, $\rbrace$ (0 atau lebih). 
		\item \textit{plus} (+), dengan jumlah kuantifier $\lbrace$ 1, $\rbrace$ (1 atau lebih). 
	\end{itemize}	 

	\item \textit{Curly Brackets} \\
	\textit{Curly Brackets} terdiri dari pasangan ''{'' dan ''}'', memiliki fungsi untuk mendefinisikan angka minimal dan maksimum dari kuantifiers yang akan digunakan. Misalnya pola a{1,5}, akan cocok dengan ''a'', ''aa'', ''aaa'', ''aaaa'' atau ''aaaaa''.
	
\end{enumerate}
		
Metakarakter \textit{inside square brackets} merupakan bagian dari kelas karakter. Pada bagian kelas  karakter ini hanya terdapat 3 macam metakarakter saja. Berikut adalah metakarakter yang digunakan:
		
\begin{table}[H]
	\caption {Tabel metakarakter \textit{inside square brackets}} \label{tab:metacharacters-inside}
	\begin{center}
		\begin{tabular}{|p{1.5cm} |p{2.5cm}|}
		\hline 
		Simbol & Nama \\ 
		\hline 
		$\backslash$ & Backslash \\ 
		\hline 
		$\hat{}$ & Circumflex \\ 
		\hline 
		- & Hyphen \\ 
		\hline
		\end{tabular}
	\end{center}
\end{table}

Pada tabel \ref{tab:metacharacters-inside} telah disebutkan macam-macam metakarakter \textit{inside square brackets}. Dari ke-3 metakarakter tersebut ada beberapa yang memiliki simbol yang sama dengan metakarakter \textit{outside square brackets}, namun fungsinya berbeda. Berikut adalah penjelasan dari fungsi metakarakter dari tabel \ref{tab:metacharacters-inside}

\begin{enumerate}
	\item \textit{Backslash} \\
	Metakarakter ini fungsinya sama dengan \textit{backslash} yang ada pada \textit{outside square brackets}. Namun dari ke-4 fungsi tersebut hanya 3 fungsi saja yang digunakan, yaitu membuat karakter lepas, Menggunakan karakter yang tidak dapat ditulis dalam pola dan menentukan jenis karakter dalam pola.
	
	\item \textit{Circumflex} \\
	Metakarakter ini memiliki fungsi yang berbeda dengan yang digunakan pada \textit{outside square brackets}. Fungsinya untuk membuat negasi, namun hanya berlaku untuk karakter pertamanya saja. Contohnya [ $\hat{}$ 0 ] akan cocok dengan semua karakter kecuali karakter 0.
	
	\item \textit{Hyphen} \\
	Metakarakter ini berfungsi untuk menentukan jangkauan dari sebuah karakter dalam kelas karakter, seperti [ 0-9 ] yang menandakan jangkauan karakter dari angka 0 hingga 9.
	
\end{enumerate}
	
\subsection{Kelas Karakter}

Kelas karakter adalah karakter yang memiliki atribut yang spesifik yang dikelompokan dalam sebuah kelas. Karakter tersebut dapat berbeda di setiap negara. Kelas karakter hanya valid digunakan pada \textit{regex} didalam tanda kurung siku pada \textit{bracket expression}. Perl mendukung notasi POSIX yang digunakan untuk kelas karakter. Dalam penggunaannya, kelas-kelas tersebut ditulis diantara ''[:'' dan '':]''. PCRE juga mendukung penggunaan notasi ini. Sebagai contoh untuk kelas alfanumerik, penulisannya yaitu [:alnum:]. Berikut ini akan dijelaskan macam-macam kelas karakter yang digunakan:

\begin{table}[H]
	\caption {Tabel kelas karakter} \label{tab:character classes}
	\begin{center}
		\begin{tabular}{|p{2 cm}|>{\raggedright} p{5 cm}| p{7.5 cm}|}
		\hline
		Kelas & Deskripsi & Keterangan \\ 
		\hline 
		alnum & Alfanumerik & Kelas yang berisi dengan karakter alfanumerik, meliputi angka dan huruf. Karakter-karakter yang termasuk yaitu, huruf a-Z atau A-Z dan angka 0-9. Simbol atau karakter khusus tidak termasuk dalam kelas ini. \newline \\ 
		\hline 
		alpha & Huruf & Kelas yang berisi dengan karakter huruf. Karakter-karakter yang termasuk yaitu, huruf a-Z atau A-Z.	Simbol atau karakter khusus tidak termasuk dalam kelas ini. \newline \\ 
		\hline 
		ascii & Kode karakter & Kelas yang merepresentasikan kode karakter dari 0-127. \newline \\ 
		\hline 
		blank & Spasi dan Tab & Karakter-karakter yang termasuk yaitu, TAB dan spasi. \newline \\ 
		\hline 
		cntrl & Karakter kontrol & Karakter kontrol adalah karakter yang tidak merepresentasikan simbol tetapi merepresentasikan character encoding. \newline \\ 
		\hline 
		digit & Angka desimal & Kelas yang berisi dengan karakter numerik. Karakter-karakter yang termasuk yaitu, angka 0-9. \newline \\ 
		\hline 
		\end{tabular} 
	\end{center}
\end{table}

\begin{table}[H]
	\caption {Tabel kelas karakter} \label{tab:character classes}
	\begin{center}
		\begin{tabular}{|p{2 cm}|>{\raggedright} p{5 cm}| p{7.5 cm}|}
		\hline
		Kelas & Deskripsi & Keterangan \\ 
		\hline 
		graph & Karakter cetak (kecuali spasi) & Kelas yang berisi karakter yang dapat dicetak dan tampak. Contoh karakter yang dapat dicetak namun tidak tampak adalah karakter spasi dan TAB. Jadi pada kelas ini karakter spasi dan tab tidak termasuk. \newline \\ 
		\hline 
		lower & Huruf kecil & Kelas yang berisi karakter dengan huruf kecil. \newline \\ 
		\hline 
		print & Karakter cetak (termasuk spasi) & Kelas yang berisi karakter yang dapat dicetak. Karakter yang termasuk kelas ini adalah spasi. \newline \\ 
		\hline 
		punct & Karakter cetak (kecuali alfanumerik) & Kelas yang berisi karakter yang dapat dicetak, namun tidak termasuk alfanumerik. \newline \\ 
		\hline 
		space & Ruang putih & Karakter yang termasuk kelas ini adalah spasi dan TAB. \newline \\ 
		\hline 
		upper & Huruf kapital & Kelas yang berisi karakter dengan huruf kapital. \newline \\ 
		\hline 
		word & Karakter ''word'' & Kelas yang berisi karakter kata. \newline \\ 
		\hline 
		xdigit & Heksadesimal & Kelas yang merepresentasikan bilangan hexadesimal, a-f atau A-F dan 0-9. \newline \\ 
		\hline 
		\end{tabular} 
	\end{center}
\end{table}

\section{PDF Parser}
\label{sec:pdfparser}

\textit{PDF Parser}~\cite{pdfparser} adalah sebuah \textit{PHP library} yang digunakan untuk mengekstrak data yang ada dalam sebuah file PDF. \textit{PDF Parser} dapat digunakan pada \textit{PHP} dengan versi 5.3 ke atas. Terdapat beberapa fitur yang dimiliki oleh \textit{library} ini. Berikut adalah fitur yang sudah dapat digunakan:

\begin{itemize}
	\item Memuat / mengurai objek dan header
	\item Menampilkan data meta, seperti nama penulis, deskripsi dan sebagainya
	\item Menampilkan isi dari teks PDF
	\item Dapat digunakan untuk kompresi PDF
	\item Mendukung \textit{MAC OS Roman charset encoding}
	\item Menangani \textit{encoding} heksa dan oktal pada teks
\end{itemize}

Dari fitur yang sudah ada, masih ada yang belum bisa ditangani oleh \textit{library} ini. \textit{PDF Parser} belum dapat mengekstrak dokumen yang diamankan. Selain itu, \textit{library} ini tidak dapat mendeteksi jenis teks yang dicetak miring, tebal dan bergaris bawah. Hingga saat ini, pengembangan \textit{library PDF parser} masih terus berjalan. 

\textit{PDF Parser} memiliki beberapa kelas yang menjalankan fungsional dari \textit{library ini}. Pada subbab berikut akan dijelaskan beberapa kelas dari \textit{library PDF Parser}.

\subsection{Kelas Parser}
Kelas ini berfungsi untuk mengatur ekstraksi file PDF. \textit{Method-method} yang terdapat pada kelas ini adalah sebagai berikut:

\begin{itemize}
	\item public function \_construct() \\
	Berfungsi untuk konstruktor kelas Parser.
	
	\item public function parseFile(\$filename) \\
	Berfungsi untuk mengembalikan isi konten dari file yang akan diekstrak. \newline
	Parameter: nama file yang akan diekstrak. \newline
	Kembalian: konten dari file yang akan diekstrak.
	
	\item public function parseContent(\$content) \\
	Berfungsi untuk mengekstrak konten pada dokumen. \newline
	Parameter: konten dari file PDF. \newline
	Kembalian: dokumen yang akan diekstrak.
	
	\item protected function parseTrailer(\$structure, \$document) \\
	Parameter: struktur dan dokumen \newline
	Kembalian: header dengan parameter trailer dan dokumen.
	
	\item protected function parseObject(\$id, \$structure, \$document) \\
	Parameter: id, struktur dan dokumen
	
	\item protected function parseHeader(\$structure, \$document) \\
	Berfungsi untuk mengekstrak header pada dokumen. \newline
	Parameter: struktur dan dokumen \newline	
	Kembalian: header dengan parameter trailer dan dokumen.
	
	\item protected function parseHeaderElement(\$type, \$value, \$document) \\
	Berfungsi untuk mengekstrak elemen-elemen yang ada pada header. \newline
	Parameter: tipe, value dan dokumen
\end{itemize}

\subsection{Page}
Kelas ini merepresentasikan halaman-halaman yang ada pada dokumen PDF. \textit{Method-method} yang terdapat pada kelas ini adalah sebagai berikut:

\begin{itemize}
	\item public function getFonts() \\
	Kembalian: mengembalikan array font.	
	
	\item public function getFont(\$id) \\
	Parameter: id dengan tipe data String. \newline
	Kembalian: megembalikan font.
	
	\item public function getXObjects() \\
	Kembalian: mengembalikan sebuah array PDFObject.
	
	\item public function getXObject(\$id) \\
	Parameter: id dengan tipe data String. \newline
	Kembalian: mengembalikan PDFObject.
	
	\item public function getText(Page \$page = null )\\
	Parameter: atribut \textit{page} dengan tipe data kelas Page. \newline
	Kembalian: mengembalikan isi teks.
		
	\item public function getTextArray(Page \$page = null) \\
	Parameter: atribut \textit{page} dengan tipe data kelas Page. \newline
	Kembalian: mengembalikan isi teks dalam sebuah array.
	
\end{itemize}

\section{Kamus Indonesia \textit{LibreOffice}}
\label{sec:kamusindo}

LibreOffice adalah sebuah paket aplikasi perkantoran berfitur lengkap yang tersedia secara gratis. LibreOffice menggunakan \textit{Open Document Format} (ODF) sebagai format aslinya untuk menyimpan dokumen. ODF menjadi format standar terbuka yang sedang diadopsi sebagai format file yang dibutuhkan untuk penerbitan dan penerimaan dokumen. LibreOffice juga dapat membuka dan menyimpan dokumen dalam format lainnya. Salah satunya format yang digunakan oleh beberapa versi dari Microsoft Office.

LibreOffice memiliki ekstensi untuk kamus Indonesia. Ekstensi ini sudah mengalami beberapa perkembangan, mulai dari versi 1.0 yang rilis pada tanggal 19 Mei 2012. Versi 1.0 merupakan hasil unggahan kembali dari ekstensi \textit{OpenOffice} yang terakhir diperbaharui pada tahun 2009. Pada tanggal 17 Mei 2014 versi 1.1 dirilis, pada versi ini LibreOffice versi 4.0 dapat menggunakan ekstensi ini. Selanjutnya, pada tanggal 15 Juli 2014 versi 2.0 dirilis. Pada versi yang paling baru ini digunakan metode baru dengan sirkumfiks yang kemudian mengubah daftar kata, sehingga memuat semua lema dari Kamus Besar Indonesia.

Ekstensi kamus Indonesia ini dapat diunduh secara gratis oleh para pengguna melalui halaman resmi dari \textit{LibreOffice}. Ekstensi yang dapat diunduh tersebut memiliki nama \textit{id\_id.oxt}. Dalam ekstensi tersebut, terdapat beberapa file yang berisi informasi tentang kamus Indonesia, yaitu \textit{id\_ID.aff} dan \textit{id\_ID.dic}.

\subsection{File \textit{id\_ID.dic}}

File ini berisi kata-kata yang akan digunakan dalam kamus bahasa Indonesia. File ini dapat dibuka dengan menggunakan aplikasi \textit{notepad} atau teks editor lainnya. Setelah dibuka menggunakan teks editor, akan terlihat 31129 baris yang berisi kata-kata yang ada dalam kamus bahasa Indonesia \textit{LibreOffice}. Untuk menjelaskan hal tersebut, akan diambil beberapa potong baris dari file tersebut.
	
	\begin{lstlisting}[caption={Potongan kode untuk file \textit{id\_ID.dic}}			\label{lst:idDic},language=plaintext,xleftmargin=.35\textwidth] 
31128
a
ab
aba
aba-aba
abad/i0
abadi/DkMkO0k0nl
abadiah
abah/DkMk
abah-abah
	\end{lstlisting}	
	
Pada listing \ref{lst:idDic}, baris pertama pada file ini menyatakan banyaknya kata yang ada pada kamus. Kata yang ada dalam kamus tersebut berjumlah 31128 buah. Baris ke-2 hingga ke-31128 merupakan kata-kata yang ada pada kamus. Namun ada beberapa kata tersebut ada yang terlihat berbeda, seperti pada baris 6, 7 dan 9. Di sebelah kiri dari kata tersebut terdapat garis miring yang disertai sebuah kode. Kode tersebut berarti bahwa kata tersebut dapat ditambahkan oleh imbuhan tertentu. Bagian ini akan dijelaskan lebih lanjut pada pembahasan file \textit{id\_ID.aff}.

\subsection{File \textit{id\_ID.aff}}

File ini berisi daftar awalan dan akhiran untuk kamus Indonesia. Awalan akan disimbolkan dengan huruf kapital, sedangkan akhiran akan disimbolkan dengan huruf kecil. Selain ke-2 hal tesebut, ada juga yang merupakan gabungan dari awalan dan akhiran. Berikut adalah penjelasan dari awalan dan akhiran:

\begin{enumerate}
	\item \textit{Prefiks} (Awalan) \\
	
	\begin{lstlisting}[caption={Potongan kode untuk prefiks}			\label{lst:prefiks},language=plaintext,xleftmargin=.35\textwidth] 
PFX B0 Y 2	# ber-
PFX B0 0 ber  	[^r]
PFX B0 0 be    	r
	\end{lstlisting}
	
	Listing \ref{lst:prefiks} merupakan salah satu contoh dari awalan yang terdapat pada kamus bahasa Indonesia. Berikut ini adalah penjelasan dari baris kode tersebut:	

	\begin{itemize}
		\item PFW\\
		Menandakan bahwa kode tersebut merupakan awalan
		
		\item B0\\
		Huruf B menunjukan awalan yang dimulai dengan huruf B, dan angka 0 menunjukan bentuk awalan asli
		
		\item 2\\
		Menandakan bahwa ada 2 jenis awalan yang dapat digunakan, yaitu ber- dan be-
		\item -ber\\
		Menandakan bahwa kode tersebut merupakan awalan ber-
	\end{itemize}	
		
	\item \textit{Sufiks} (Akhiran) \\	
	
	\begin{lstlisting}[caption={Potongan kode untuk sufiks}			\label{lst:sufiks},language=plaintext,xleftmargin=.35\textwidth] 
SFX k0 Y  3	 # -kan
SFX k0 0  kan    . 
SFX k0 0  kanlah
SFX k0 0  kankah
	\end{lstlisting}
	
	 Listing \ref{lst:sufiks} merupakan salah satu contoh dari akhiran yang terdapat pada kamus bahasa Indonesia. Berikut ini adalah penjelasan dari baris kode tersebut:
	
	\begin{itemize}
		\item SFX\\
		Menandakan bahwa kode tersebut merupakan akhiran
		
		\item k0\\
		Huruf k menunjukan akhiran yang dimulai dengan huruf k, dan angka 0 menunjukan bentuk akhiran asli
		
		\item 3\\
		Menandakan bahwa ada 3 jenis akhiran yang dapat digunakan, yaitu -kan, -kanlah dan -kankah
		
		\item -kan\\
		Menandakan bahwa kode tersebut merupakan akhiran -kan 
	\end{itemize}
	
\end{enumerate}