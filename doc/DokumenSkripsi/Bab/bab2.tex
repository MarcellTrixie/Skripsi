%versi 2 (8-10-2016)
\chapter{Landasan Teori}
\label{chap:teori}

Pada bab 2 akan diuraikan tentang landasan teori yang membahas \textit{regular expression},

\section{\textit{Regular Expression}}
\label{sec:regex} 
 
Pada subbab ini akan diuraikan tentang teori \textit{regular expression},

\subsection{Teori \textit{Regular Expression}}

\textit{Regular Expression} (\textit{Regex)} adalah jenis pola teks tertentu yang dapat digunakan pada banyak aplikasi modern dan bahasa pemrograman. \textit{Regex} terdiri dari kombinasi antara \textit{normalcharacter}, \textit{metacharacter} dan \textit{metasequences}. \textit{Normal character} mewakili karakter itu sendiri, sedangkan \textit{metacharacter} dan \textit{metasequences} merupakan karakter atau urutan karakter yang mewakili ide seperti kuantitas, lokasi, atau jenis karakter.

\textit{Regex}dapat digunakan untuk memverifikasi kecocokan antara input dengan pola teks, untuk menemukan teks yang cocok dengan pola dalam teks yang lebih besar, untuk mengganti teks yang cocok dengan pola dengan teks lain atau menyusun ulang bit dari teks yang cocok dan untuk membagi sebuah blok teks menjadi beberapa subteks.

\subsection{\textit{Regex Metacharacters, Modes, dan Construct}}

