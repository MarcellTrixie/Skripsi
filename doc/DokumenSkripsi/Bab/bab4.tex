\chapter{Perancangan}
\label{chap:Perancangan}

Pada bab ini dijelaskan mengenai perancangan perangkat lunak yang dibangun.

\section{Algoritma Pengecekan}

Hasil survei kesalahan-kesalahan dalam penulisan dokumen skripsi sudah disaring dan akan diimplementasikan dalam perangkat lunak. Pengecekan kesalahan tersebut akan dilakukan dengan menggunakan teknik \textit{pattern matching regex}. Berikut adalah rincian dari hasil survei yang dipilih beserta penyelesaiannya:

\begin{enumerate}
	\item Penulisan kata (PS-01) \\
	Untuk mendeteksi kesalahan penulisan kata, akan digunakan ekstensi kamus bahasa Indonesia \textit{LibreOffice}. Dengan menggunakan kamus tersebut, dapat meminimalisir kesalahan penulisan suatu kata. \newline
	Penyelesaian dengan \textit{regex}:
	
	\item Pemberian spasi sebelum dan sesudah tanda baca (PS-03) \\
	Kesalahan ini akan terdeteksi apabila tidak ada karakter spasi sebelum ataupun sesudah tanda baca.
	Penyelesaian dengan \textit{regex}:
	
	\item Awal kalimat tidak menggunakan huruf kapital (PS-05) \\
	Kesalahan ini akan terdeteksi apabila setelah tanda baca pada akhir kalimat, karakter pertama setelah spasi menggunakan huruf kecil.	
	Penyelesaian dengan \textit{regex}:
	
	\item Tidak ada spasi antar kata (PS-06) \\
	Hal ini dapat diselesaikan dengan menggunakan kamus bahasa Indonesia. Pada saat dua atau lebih kata tidak dipisahkan dengan spasi, maka kata tersebut tidak akan terdapat pada kamus. \newline
	Penyelesaian dengan \textit{regex}:
	
	\item Jumlah subbab dalam 1 bab tidak boleh hanya 1 (PS-09) \\
	Kesalahan ini dapat diselesaikan dengan mencari jumlah subbab yang ada dalam sebuah bab. \newline:
	Penyelesaian dengan \textit{regex}:
	
	\item Kalimat pengantar untuk setiap subbab (KAL-02) \\
	Kesalahan ini dapat dideteksi dengan melihat ada atau tidaknya kalimat setelah subbab dibuat. \newline
	Penyelesaian dengan \textit{regex}:
	
	\item Kelengkapan data skripsi (KAL-03) \\
	Kesalahan ini dapat terlihat pada halaman cover skripsi. Mahasiswa yang belum mengisi data skripsi, pada file PDFnya akan ditampilkan tulisan template pada data skripsinya. \newline
	Penyelesaian dengan \textit{regex}:
	
	\item Penggunaan kata ganti orang (VAN-03) \\
	Kesalahan ini dapat diatasi dengan memasukan kata-kata yang termasuk dalam kata ganti orang menjadi kata-kata yang tidak dapat digunakan. \newline
	Penyelesaian dengan \textit{regex}:
	
	\item Penulisan daftar referensi (NAT-01) \\
	Kesalahan dalam penulisan daftar referensi dapat dilihat dengan munculnya tanda ''[?]''. \newline
	Penyelesaian dengan \textit{regex}:
	
\end{enumerate}