\chapter{Kesimpulan dan Saran}
\label{chap:kesimpulan_dan_saran}

Pada bab ini berisi kesimpulan dari pembangunan aplikasi dan saran untuk pengembangan aplikasi ini.

\section{Kesimpulan}
\label{sec:kesimpulan}

Dari pengujian yang telah dilakukan, didapatkan kesimpulan-kesimpulan sebagai berikut:

\begin{enumerate}
	\item Kesalahan dalam dokumen skripsi diperiksa menggunakan pencocokan pola, dengan menggunakan \textit{regular expression}. Kesalahan yang dapat diperiksa pada perangkat lunak adalah kesalahan yang bersifat tekstual. Pola \textit{regular expression} tidak dapat memeriksa kesalahan yang bersifat kontekstual.
	
	\item Perangkat lunak pemeriksa kesalahaan dokumen skripsi berhasil dikembangkan dengan menggunakan bahasa pemrograman \textit{PHP}. Perangkat lunak menerapkan prinsip paradigma \textit{Object Oriented Programming}. Dalam pengembangannya, dibutuhkan sebuah \textit{library PdFParser} yang digunakan untuk melakukan ekstrak dokumen skripsi. Perangkat lunak akan menerima input nama file dokumen skripsi yang akan diperiksa, dan memberikan output berupa laporan kesalahan yang ditemukan.
	
	\item Hasil pengujian yang diperoleh, secara umum sudah memberikan hasil yang sesuai dengan yang diharapkan. Setiap fitur yang ada pada perangkat lunak dapat mencari dan melaporkan kesalahan yang ditemukan.
\end{enumerate}

\section{Saran}
\label{sec:saran}
Dari pengujian yang telah dilakukan, terdapat beberapa saran yang dapat digunakan sebagai pengembangan perangkat lunak:

\begin{enumerate}
	\item Menambahkan nomor halaman yang menjadi tempat ditemukannya kesalahan pada setiap fitur yang ada pada perangkat lunak.
		
	\item Mengembangkan fitur pemeriksa kesalahan kata, agar dapat memeriksa kata-kata yang menggunakan imbuhan.
	
	\item Membuat kamus yang dinamis untuk fitur pemeriksa kesalahan kata, agar dapat ditambahkan sesuai dengan kebutuhan.
	
	\item Mencari alternatif \textit{library PdfParser}, karena \textit{library} ini masih ada beberapa kekurangan.
	
	\item Menambahkan fitur-fitur yang belum ada pada perangkat lunak ini.
\end{enumerate}