\chapter{Kesimpulan dan Saran}
\label{chap:kesimpulan_dan_saran}

Pada bab ini berisi kesimpulan dari pembangunan aplikasi dan saran untuk pengembangan aplikasi ini.

\section{Kesimpulan}
\label{sec:kesimpulan}

Dari hasil penelitian yang dilakukan, didapatkan kesimpulan-kesimpulan sebagai berikut:

\begin{enumerate}
	\item Kesalahan yang ada pada dokumen skripsi berhasil diperiksa menggunakan metode pencocokan pola \textit{regular expression}. Pola akan memeriksa kesalahan yang bersifat tekstual pada dokumen skripsi. 
	
	\item Perangkat lunak pemeriksa kesalahaan dokumen skripsi berhasil dikembangkan dengan menggunakan bahasa pemrograman \textit{PHP}. Perangkat lunak menerapkan prinsip paradigma \textit{Object Oriented Programming}. Dalam pengembangannya, dibutuhkan sebuah \textit{library PdFParser} yang digunakan untuk melakukan ekstrak dokumen skripsi. Perangkat lunak akan menerima input nama file dokumen skripsi yang akan diperiksa, dan memberikan output berupa laporan kesalahan yang ditemukan.
\end{enumerate}

\section{Saran}
\label{sec:saran}
Dari hasil penelitian yang dilakukan, berikut adalah beberapa saran yang dapat digunakan sebagai pengembangan perangkat lunak:

\begin{enumerate}
	\item Mengembangkan fitur pemeriksa kesalahan kata, agar dapat memeriksa kata-kata yang menggunakan imbuhan.
	
	\item Mencari alternatif \textit{library PdfParser}, karena \textit{library} ini masih ada beberapa kekurangan.
	
	\item Menambahkan fitur-fitur yang belum ada pada perangkat lunak ini.
\end{enumerate}