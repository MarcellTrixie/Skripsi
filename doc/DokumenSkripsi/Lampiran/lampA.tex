%versi 3 (18-12-2016)
\chapter{Kode Program}
\label{lamp:A}

%terdapat 2 cara untuk memasukkan kode program
% 1. menggunakan perintah \lstinputlisting (kode program ditempatkan di folder yang sama dengan file ini)
% 2. menggunakan environment lstlisting (kode program dituliskan di dalam file ini)
% Perhatikan contoh yang diberikan!!
%
% untuk keduanya, ada parameter yang harus diisi:
% - language: bahasa dari kode program (pilihan: Java, C, C++, PHP, Matlab, C#, HTML, R, Python, SQL, dll)
% - caption: nama file dari kode program yang akan ditampilkan di dokumen akhir
%
% Perhatian: Abaikan warning tentang textasteriskcentered!!
%

\lstinputlisting[language=PHP, caption=SkripsiExtract.php]{./Lampiran/SkripsiExtract.php}

\lstinputlisting[language=PHP, caption=Checker.php]{./Lampiran/Checker.php} 

\lstinputlisting[language=PHP, caption=Main.php]{./Lampiran/Main.php}

\lstinputlisting[language=PHP, caption=KAL02\_PrefaceChecker.php]{./Lampiran/Checkers/KAL02_PrefaceChecker.php}

\lstinputlisting[language=PHP, caption=KAL03\_ThesisDataChecker.php]{./Lampiran/Checkers/KAL03_ThesisDataChecker.php}

\lstinputlisting[language=PHP, caption=NAT01\_ReferenceChecker.php]{./Lampiran/Checkers/NAT01_ReferenceChecker.php}

\lstinputlisting[language=PHP, caption=PS01\_TypoChecker.php]{./Lampiran/Checkers/PS01_TypoChecker.php}

\lstinputlisting[language=PHP, caption=PS03\_SpaceChecker.php]{./Lampiran/Checkers/PS03_SpaceChecker.php}

\lstinputlisting[language=PHP, caption=PS05\_CapitalLetterChecker.php]{./Lampiran/Checkers/PS05_CapitalLetterChecker.php}

\lstinputlisting[language=PHP, caption=PS09\_SubChapterChecker.php]{./Lampiran/Checkers/PS09_SubChapterChecker.php}

\lstinputlisting[language=PHP, caption=VAN03\_SubjectPronounChecker.php]{./Lampiran/Checkers/VAN03_SubjectPronounChecker.php}
