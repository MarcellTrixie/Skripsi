\documentclass[a4paper,twoside]{article}
\usepackage[T1]{fontenc}
\usepackage[bahasa]{babel}
\usepackage{graphicx}
\usepackage{graphics}
\usepackage{float}
\usepackage[cm]{fullpage}
\pagestyle{myheadings}
\usepackage{etoolbox}
\usepackage{setspace} 
\usepackage{lipsum} 
\setlength{\headsep}{30pt}
\usepackage[inner=2cm,outer=2.5cm,top=2.5cm,bottom=2cm]{geometry} %margin
% \pagestyle{empty}

\makeatletter
\renewcommand{\@maketitle} {\begin{center} {\LARGE \textbf{ \textsc{\@title}} \par} \bigskip {\large \textbf{\textsc{\@author}} }\end{center} }
\renewcommand{\thispagestyle}[1]{}
\markright{\textbf{\textsc{AIF402 \textemdash Rencana Kerja Skripsi \textemdash Sem. Genap 2019/2020}}}

\newcommand{\HRule}{\rule{\linewidth}{0.4mm}}
\renewcommand{\baselinestretch}{1}
\setlength{\parindent}{0 pt}
\setlength{\parskip}{6 pt}

\onehalfspacing
 
\begin{document}

\title{\@judultopik}
\author{\nama \textendash \@npm} 

%tulis nama dan NPM anda di sini:
\newcommand{\nama}{Marcell Trixie Alexander}
\newcommand{\@npm}{2014730003}
\newcommand{\@judultopik}{Aplikasi Pemeriksa Kesalahan Umum Dokumen Skripsi Teknik Informatika UNPAR} % Judul/topik anda
\newcommand{\jumpemb}{1} % Jumlah pembimbing, 1 atau 2
\newcommand{\tanggal}{20/08/2019}

% Dokumen hasil template ini harus dicetak bolak-balik !!!!

\maketitle

\pagenumbering{arabic}

\section{Deskripsi}
Skripsi merupakan karangan ilmiah yang wajib ditulis oleh mahasiswa sebagai bagian dari persyaratan akhir pendidikan akademiknya. Namun dalam penulisannya, peserta skripsi sering melakukan kesalahan kecil. Kesalahan sering terjadi dalam penggunaan imbuhan, kata keterangan, penulisan kata dan sebagainya. Hal-hal seperti ini seharusnya dapat diperiksa dan diminimalisir oleh diri sendiri. Pada saat bimbingan, waktu dosen pembimbing lebih baik dimanfaatkan untuk membahas konten dibanding memeriksa kesalahan-kesalahan tersebut.

Dari masalah tersebut dapat dibuat sebuah aplikasi untuk melakukan pemeriksaan pada dokumen skripsi. Kesalahan yang akan diperiksa berasal dari survei yang dilakukan kepada dosen-dosen Informatika Unpar. Hasil dari survei tersebut akan diseleksi untuk diimplementasikan ke dalam aplikasi. Aplikasi sederhana ini dapat dimanfaatkan oleh mahasiswa Informatika Unpar secara mandiri. Aplikasi ini dijalankan melalui melalui terminal \textit{command Windows}. Aplikasi menerima masukan berupa file \textit{PDF} skripsi dan menampilkan laporan yang berisi kesalahan-kesalahan yang ditemukan pada dokumen skripsi. 

\section{Rumusan Masalah}
Berdasarkan deskripsi topik yang sudah ditulis, dapat dirumuskan masalah sebagai berikut:
\begin{itemize}
	\item Apa saja kesalahan-kesalahan yang sering ditemukan dalam penulisan dokumen skripsi?
	\item Bagaimana cara membuat perangkat lunak yang dapat memeriksa kesalahan pada dokumen skripsi?
\end{itemize}

\section{Tujuan}
Tujuan dari penulisan topik ini adalah:
\begin{itemize}
	\item Dapat menemukan kesalahan-kesalahan yang sering terjadi dalam penulisan dokumen skripsi
	\item Dapat membangun perangkat lunak untuk memeriksa kesalahan yang ada pada dokumen skripsi
\end{itemize}

\section{Deskripsi Perangkat Lunak}
Perangkat lunak akhir yang akan dibuat memiliki fitur minimal sebagai berikut:
\begin{itemize}
	\item Aplikasi dapat menerima masukan berupa file PDF skripsi yang ingin diperiksa
	\item Aplikasi dapat melakukan ekstraksi teks PDF serta melakukan pemeriksaan
	\item Aplikasi dapat menampilkan laporan lengkap kesalahan yang ditemukan
\end{itemize}

\section{Detail Pengerjaan Skripsi}
Bagian-bagian pekerjaan skripsi ini adalah sebagai berikut:
\begin{enumerate}		
	\item Melakukan survei kepada dosen-dosen Informatika mengenai kesalahan-kesalahan penulisan yang ditemui dalam dokumen skripsi	
	\item Melakukan studi literatur \textit{Regular Expression} untuk mendeteksi kesalahan-kesalahan dalam file \textit{PDF} skripsi
	\item Mempelajari \textit{library PDF Parser} untuk mengestraksi file \textit{PDF} skripsi yang akan diperiksa
	\item Melakukan perancangan perangkat lunak
	\item Melakukan implementasi perancangan perangkat lunak
	\item Melakukan pengujian terhadap perancangan perangkat lunak
	\item Menulis dokumen skripsi
\end{enumerate}

\section{Rencana Kerja}
Penulis sudah menuntaskan mata kuliah skripsi 1, sehingga seluruh detail pengerjaan akan dilakukan pada mata kuliah skripsi 2. Rincian capaian yang direncanakan pada Skripsi 2 adalah sebagai berikut:
\begin{enumerate}
	\item Melakukan survei kepada dosen-dosen Informatika mengenai kesalahan-kesalahan penulisan yang ditemui dalam dokumen skripsi	
	\item Melakukan studi literatur \textit{Regular Expression} untuk mendeteksi kesalahan-kesalahan dalam file \textit{PDF} skripsi
	\item Mempelajari \textit{library PDF Parser} untuk mengestraksi file \textit{PDF} skripsi yang akan diperiksa
	\item Melakukan perancangan perangkat lunak
	\item Melakukan implementasi perancangan perangkat lunak
	\item Melakukan pengujian terhadap perancangan perangkat lunak
	\item Menulis dokumen skripsi
\end{enumerate}

\vspace{1cm}
\centering Bandung, \tanggal\\
\vspace{2cm} \nama \\ 
\vspace{1cm}

Menyetujui, \\
\ifdefstring{\jumpemb}{2}{
\vspace{1.5cm}
\begin{centering} Menyetujui,\\ \end{centering} \vspace{0.75cm}
\begin{minipage}[b]{0.45\linewidth}
% \centering Bandung, \makebox[0.5cm]{\hrulefill}/\makebox[0.5cm]{\hrulefill}/2013 \\
\vspace{2cm} Nama: \makebox[3cm]{\hrulefill}\\ Pembimbing Utama
\end{minipage} \hspace{0.5cm}
\begin{minipage}[b]{0.45\linewidth}
% \centering Bandung, \makebox[0.5cm]{\hrulefill}/\makebox[0.5cm]{\hrulefill}/2013\\
\vspace{2cm} Nama: \makebox[3cm]{\hrulefill}\\ Pembimbing Pendamping
\end{minipage}
\vspace{0.5cm}
}{
% \centering Bandung, \makebox[0.5cm]{\hrulefill}/\makebox[0.5cm]{\hrulefill}/2013\\
\vspace{2cm} Nama: \makebox[3cm]{\hrulefill}\\ Pembimbing Tunggal
}
\end{document}